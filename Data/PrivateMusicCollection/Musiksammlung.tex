%% Basierend auf einer TeXnicCenter-Vorlage von Tino Weinkauf.
%%%%%%%%%%%%%%%%%%%%%%%%%%%%%%%%%%%%%%%%%%%%%%%%%%%%%%%%%%%%%%

%%%%%%%%%%%%%%%%%%%%%%%%%%%%%%%%%%%%%%%%%%%%%%%%%%%%%%%%%%%%%
%% HEADER
%%%%%%%%%%%%%%%%%%%%%%%%%%%%%%%%%%%%%%%%%%%%%%%%%%%%%%%%%%%%%
\documentclass[a4paper,oneside,10pt]{report}
% Alternative Optionen:
%	Papiergröße: a4paper / a5paper / b5paper / letterpaper / legalpaper / executivepaper
% Duplex: oneside / twoside
% Grundlegende Fontgrößen: 10pt / 11pt / 12pt

%Hyperref Package für Links in der PDF
\usepackage[pdftex,
linktoc = all,			%Verlinkt sowohl text als auch Nummer im Inhaltsverzeichnis
colorlinks=true,    %Bei false macht er einen häßlcihen roten Rahmen um die Links
linkcolor=black,    %Linkfarbe im PDF schwarz machen. Für elektronische Version ändern.
citecolor=black,    %Linkfarbe im PDF schwarz machen. Für elektronische Version ändern.
bookmarks=true,     %Inhaltsverzeichnis auf der linken Seite beim öffnen der PDF
bookmarksopen=true,     	%Das Verzeichnis öffnen
bookmarksopenlevel=1,   	%Öffnungstiefe (chapter, section, subsection)
pagebackref=false,      	%Aus Literaturverzeichnissen etc. auf die entsprechende Seite springen- NEINEINEIN!!1elf
bookmarksnumbered=false, 	%Einträge im Verzeichnis nummerieren
pdfstartpage=1,         	%Startseite für die Anzeige festlegen
pdfpagemode=UseOutlines,
pdfstartview={XYZ null null 1.00}
]{hyperref} 

%% Deutsche Anpassungen %%%%%%%%%%%%%%%%%%%%%%%%%%%%%%%%%%%%%
\usepackage[english]{babel}
\usepackage[T1]{fontenc}
\usepackage[utf8]{inputenc}

\usepackage{lmodern} %Type1-Schriftart für nicht-englische Texte

%% Andere Packages %%%%%%%%%%%%%%%%%%%%%%%%%%%%%%%%%%%%%%%%%%
\usepackage{a4wide} %%Kleinere Seitenränder = mehr Text pro Zeile.
%% Zeilenabstand %%%%%%%%%%%%%%%%%%%%%%%%%%%%%%%%%%%%%%%%%%%%
\usepackage{setspace}
\singlespacing        %% 1-zeilig (Standard)
%\onehalfspacing       %% 1,5-zeilig
%\doublespacing        %% 2-zeilig

%% Packages für Grafiken & Abbildungen %%%%%%%%%%%%%%%%%%%%%%
\usepackage{graphicx} %%Zum Laden von Grafiken
%\usepackage{subfig} %%Teilabbildungen in einer Abbildung
%\usepackage{pst-all} %%PSTricks - nicht verwendbar mit pdfLaTeX

%% Beachten Sie:
%% Die Einbindung einer Grafik erfolgt mit \includegraphics{Dateiname}
%% bzw. über den Dialog im Einfügen-Menü.
%% 
%% Im Modus "LaTeX => PDF" können Sie u.a. folgende Grafikformate verwenden:
%%   .jpg  .png  .pdf  .mps
%% 
%% In den Modi "LaTeX => DVI", "LaTeX => PS" und "LaTeX => PS => PDF"
%% können Sie u.a. folgende Grafikformate verwenden:
%%   .eps  .ps  .bmp  .pict  .pntg


%% Packages für Formeln %%%%%%%%%%%%%%%%%%%%%%%%%%%%%%%%%%%%%
\usepackage{amsmath}
\usepackage{amsthm}
\usepackage{amsfonts}

\usepackage{caption}
\usepackage{float}
\restylefloat{figure}
\usepackage{here} 

\usepackage{titlesec}
\titleformat{\chapter}{\bfseries\Huge}{\thechapter.\quad}{0em}{}

%\usepackage{paralist}
\usepackage{enumitem}



%%%%%%%%%%%%%%%%%%%%%%%%%%%%%%%%%%%%%%%%%%%%%%%%%%%%%%%%%%%%%
%% DOKUMENT
%%%%%%%%%%%%%%%%%%%%%%%%%%%%%%%%%%%%%%%%%%%%%%%%%%%%%%%%%%%%%
\begin{document}
\setlength{\parindent}{0em}
\pagestyle{empty} %%Keine Kopf-/Fusszeilen auf den ersten Seiten.


%% Deckblatt %%%%%%%%%%%%%%%%%%%%%%%%%%%%%%%%%%%%%%%%%%%%%%%%
%% ==> Schreiben Sie hier Ihren Text oder fügen Sie eine externe Datei ein.

%% Die einfache Version:
\title{Private Music Collection}
\author{Johannes Schoder}
%\date{} %%Wenn kommentiert, wird das aktuelle Datum verwendet.
\maketitle

%% Die schönere Version:
%\input{deckblatt} %%Eine Datei 'deckblatt.tex' wird hierfür benötigt.
%% ==> TeXnicCenter liefert eine mögliche Deckblattdatei
%% ==> im Vorlagenarchiv mit (Datei | Neu von Vorlage...).

\setcounter{page}{109}

%% Inhaltsverzeichnis %%%%%%%%%%%%%%%%%%%%%%%%%%%%%%%%%%%%%%%
\tableofcontents %Inhaltsverzeichnis
\cleardoublepage %Das erste Kapitel soll auf einer ungeraden Seite beginnen.

\pagestyle{plain} %%Ab hier die Kopf-/Fusszeilen: headings / fancy / ...

%\setcounter{page}{116}

Album covers removed to prevent copyright infringements, replaced by:\\ http://www.publicdomainfiles.com/show\_file.php?id=13528072212039\\
\ \\
\ \\
Chapter \ref{metal} and Chapter \ref{modern} Contains: artists, album name, year, and favorite songs on the album (except for a few albums in Chapter \ref{modern} that are only listed without favorite tracks)\\
Chapter \ref{singles} and Chapter \ref{btracks} contain single purchased songs instead of albums.\\
Chapter \ref{fav} contains a list of overall favorite songs and albums.\\

%===========================================
% MELODEATH
%===========================================
\chapter{Metal}\label{metal}

\section{Melodic Death Metal}

\subsection{In Flames}

\begin{minipage}[t]{0.25\textwidth}\vspace{0pt}
\vspace{0pt}
\captionsetup{type=figure}
\includegraphics[width=\textwidth]{Images/cover.png}
\caption*{Lunar Strain \& Subterranean (1994)}
\end{minipage}
\begin{minipage}[t]{0.25\textwidth}\vspace{0pt}
\vspace{0pt}
\begin{itemize}[nosep,leftmargin=1em,labelwidth=*,align=left]
	\setlength{\itemsep}{0pt}
	\item Behind Space
	\item Lunar Strain
	\item Everlost (Pt. I \& II)
	\item In Flames
	\item Dreamscape
	\item Starforsaken
\end{itemize}
\end{minipage}
%=============
\begin{minipage}[t]{0.25\textwidth}\vspace{0pt}
\captionsetup{type=figure}
\includegraphics[width=\textwidth]{Images/cover.png}
\caption*{The Jester Race (1996)}
\end{minipage}
\begin{minipage}[t]{0.25\textwidth}\vspace{0pt}
\begin{itemize}[nosep,leftmargin=1em,labelwidth=*,align=left]
	\setlength{\itemsep}{0pt}
	\item Moonshield
	\item The Jester's Dance
	\item Lord Hypnos
	\item Acoustic Medley
	\item December Flower
\end{itemize}
\end{minipage}
%=============
\begin{minipage}[t]{0.25\textwidth}\vspace{0pt}
\captionsetup{type=figure}
\includegraphics[width=\textwidth]{Images/cover.png}
\caption*{Whoracle (1997)}
\end{minipage}
\begin{minipage}[t]{0.25\textwidth}\vspace{0pt}
\begin{itemize}[nosep,leftmargin=1em,labelwidth=*,align=left]
	\setlength{\itemsep}{0pt}
	\item Episode 666
	\item Gyroscope
	\item Whoracle
	\item Worlds Within The Margin
	\item Jester Script Transfigured
	\item Dialogue With The Stars
	\item Morphing Into Primal
	\item Jotun
\end{itemize}
\end{minipage}
%=============
\begin{minipage}[t]{0.25\textwidth}\vspace{0pt}
\captionsetup{type=figure}
\includegraphics[width=\textwidth]{Images/cover.png}
\caption*{Colony (1999)}
\end{minipage}
\begin{minipage}[t]{0.25\textwidth}\vspace{0pt}
\begin{itemize}[nosep,leftmargin=1em,labelwidth=*,align=left]
	\setlength{\itemsep}{0pt}
	\item The New World
	\item Ordinary Story
	\item Zombie Inc.
	\item Colony
	\item Pallar Anders Visa
	\item Resin
\end{itemize}
\end{minipage}
%=============
\begin{minipage}[t]{0.25\textwidth}\vspace{0pt}
\captionsetup{type=figure}
\includegraphics[width=\textwidth]{Images/cover.png}
\caption*{Clayman (2000)}
\end{minipage}
\begin{minipage}[t]{0.25\textwidth}\vspace{0pt}
\begin{itemize}[nosep,leftmargin=1em,labelwidth=*,align=left]
	\setlength{\itemsep}{0pt}
	\item Pinball Map
	\item Bullet Ride
	\item Only For The Weak
	\item Sattelites and Astronauts
	\item Suburban Me
	\item Clayman
	\item Square Nothing
\end{itemize}
\end{minipage}
%=============
\begin{minipage}[t]{0.25\textwidth}\vspace{0pt}
\captionsetup{type=figure}
\includegraphics[width=\textwidth]{Images/cover.png}
\caption*{Reroute To Remain (2002)}
\end{minipage}
\begin{minipage}[t]{0.25\textwidth}\vspace{0pt}
\begin{itemize}[nosep,leftmargin=1em,labelwidth=*,align=left]
	\setlength{\itemsep}{0pt}
	\item Trigger
	\item Cloud Connected
	\item Reroute To Remain
	\item Black \& White
	\item Metaphor
	\item Free Fall
\end{itemize}
\end{minipage}
%=============
\begin{minipage}[t]{0.25\textwidth}\vspace{0pt}
\captionsetup{type=figure}
\includegraphics[width=\textwidth]{Images/cover.png}
\caption*{Soundtrack To Your Escape (2004)}
\end{minipage}
\begin{minipage}[t]{0.25\textwidth}\vspace{0pt}
\begin{itemize}[nosep,leftmargin=1em,labelwidth=*,align=left]
	\setlength{\itemsep}{0pt}
	\item Dead Alone
	\item My Sweet Shadow
	\item The Quiet Place
\end{itemize}
\end{minipage}
%=============
\begin{minipage}[t]{0.25\textwidth}\vspace{0pt}
\captionsetup{type=figure}
\includegraphics[width=\textwidth]{Images/cover.png}
\caption*{Used And Abused (DVD) (2005)}
\end{minipage}
\begin{minipage}[t]{0.25\textwidth}\vspace{0pt}
\begin{itemize}[nosep,leftmargin=1em,labelwidth=*,align=left]
	\setlength{\itemsep}{0pt}
	\item Dead Alone
	\item My Sweet Shadow
	\item The Quiet Place
\end{itemize}
\end{minipage}
%=============
\begin{minipage}[t]{0.25\textwidth}\vspace{0pt}
\captionsetup{type=figure}
\includegraphics[width=\textwidth]{Images/cover.png}
\caption*{Come Clarity (2006)}
\end{minipage}
\begin{minipage}[t]{0.25\textwidth}\vspace{0pt}
\begin{itemize}[nosep,leftmargin=1em,labelwidth=*,align=left]
	\setlength{\itemsep}{0pt}
	\item Leeches
	\item Take This Life
	\item Come Clarity
	\item Scream
	\item Crawling Through Knives
	\item Our Infinite Struggle
\end{itemize}
\end{minipage}
%=============
\begin{minipage}[t]{0.25\textwidth}\vspace{0pt}
\captionsetup{type=figure}
\includegraphics[width=\textwidth]{Images/cover.png}
\caption*{A Sense Of Purpose (2008)}
\end{minipage}
\begin{minipage}[t]{0.25\textwidth}\vspace{0pt}
\begin{itemize}[nosep,leftmargin=1em,labelwidth=*,align=left]
	\setlength{\itemsep}{0pt}
	\item Alias
	\item The Chosen Pessimist
	\item Disconnected
	\item I'm The Highway
	\item The Mirror's Truth
	\item Move Through Me
\end{itemize}
\end{minipage}
%=============
\begin{minipage}[t]{0.25\textwidth}\vspace{0pt}
\captionsetup{type=figure}
\includegraphics[width=\textwidth]{Images/cover.png}
\caption*{Sounds Of A Playground Fading (2011)}
\end{minipage}
\begin{minipage}[t]{0.25\textwidth}\vspace{0pt}
\begin{itemize}[nosep,leftmargin=1em,labelwidth=*,align=left]
	\setlength{\itemsep}{0pt}
	\item Sounds Of A Playground Fading
	\item All For Me
	\item Deliver Us
	\item Ropes
	\item Enter Tragedy
	\item Where The Dead Ships Dwell
	\item The Puzzle
\end{itemize}
\end{minipage}
%=============
\begin{minipage}[t]{0.25\textwidth}\vspace{0pt}
\captionsetup{type=figure}
\includegraphics[width=\textwidth]{Images/cover.png}
\caption*{Siren Charms (2014)}
\end{minipage}
\begin{minipage}[t]{0.25\textwidth}\vspace{0pt}
\begin{itemize}[nosep,leftmargin=1em,labelwidth=*,align=left]
	\setlength{\itemsep}{0pt}
	\item Everything's Gone
	\item Paralyzed
	\item When The World Explodes
	\item Siren Charms
	\item Rusted Nail
\end{itemize}
\end{minipage}
%=============
\begin{minipage}[t]{0.25\textwidth}\vspace{0pt}
\captionsetup{type=figure}
\includegraphics[width=\textwidth]{Images/cover.png}
\caption*{Battles (2016)}
\end{minipage}
\begin{minipage}[t]{0.25\textwidth}\vspace{0pt}
\begin{itemize}[nosep,leftmargin=1em,labelwidth=*,align=left]
	\setlength{\itemsep}{0pt}
	\item The End
	\item Battles
	\item Us Against The World
	\item Wallflower
	\item Here Until Forever
	\item Save Me
\end{itemize}
\end{minipage}
%=============
\begin{minipage}[t]{0.25\textwidth}\vspace{0pt}
\captionsetup{type=figure}
\includegraphics[width=\textwidth]{Images/cover.png}
\caption*{Sounds From The Heart Of Gothenburg (2016)}
\end{minipage}
\begin{minipage}[t]{0.25\textwidth}\vspace{0pt}
\begin{itemize}[nosep,leftmargin=1em,labelwidth=*,align=left]
	\setlength{\itemsep}{0pt}
	\item The Chosen Pessimist
\end{itemize}
\end{minipage}
%=============
\begin{minipage}[t]{0.25\textwidth}\vspace{0pt}
\captionsetup{type=figure}
\includegraphics[width=\textwidth]{Images/cover.png}
\caption*{Down, Wicked \& No Good (2017)}
\end{minipage}
\begin{minipage}[t]{0.25\textwidth}\vspace{0pt}
\begin{itemize}[nosep,leftmargin=1em,labelwidth=*,align=left]
	\setlength{\itemsep}{0pt}
	\item Hurt
\end{itemize}
\end{minipage}
%=============
\begin{minipage}[t]{0.25\textwidth}\vspace{0pt}
\captionsetup{type=figure}
\includegraphics[width=\textwidth]{Images/cover.png}
\caption*{I, The Mask (2019)}
\end{minipage}
\begin{minipage}[t]{0.25\textwidth}\vspace{0pt}
\begin{itemize}[nosep,leftmargin=1em,labelwidth=*,align=left]
	\setlength{\itemsep}{0pt}
	\item I, The Mask
	\item I Am Above
\end{itemize}
\end{minipage}


\subsection{Dark Tranquillity}

\begin{minipage}[t]{0.25\textwidth}\vspace{0pt}
\captionsetup{type=figure}
\includegraphics[width=\textwidth]{Images/cover.png}
\caption*{The Gallery (1993)}
\end{minipage}
\begin{minipage}[t]{0.25\textwidth}\vspace{0pt}
\begin{itemize}[nosep,leftmargin=1em,labelwidth=*,align=left]
	\setlength{\itemsep}{0pt}
	\item Punish My Heaven
	\item Edenspring
	\item Lethe
	\item Mine Is The Grandeur
	\item ... Of Melancholy Burning
\end{itemize}
\end{minipage}
%=============
\begin{minipage}[t]{0.25\textwidth}\vspace{0pt}
\captionsetup{type=figure}
\includegraphics[width=\textwidth]{Images/cover.png}
\caption*{Fiction (2007)}
\end{minipage}
\begin{minipage}[t]{0.25\textwidth}\vspace{0pt}
\begin{itemize}[nosep,leftmargin=1em,labelwidth=*,align=left]
	\setlength{\itemsep}{0pt}
	\item The Lesser Faith
	\item Terminus
	\item Misery's Crown
	\item Focus Shift
	\item The Mundane And The Magic
\end{itemize}
\end{minipage}
%=============
\begin{minipage}[t]{0.25\textwidth}\vspace{0pt}
\captionsetup{type=figure}
\includegraphics[width=\textwidth]{Images/cover.png}
\caption*{Where Death Is Most Alive (2009)}
\end{minipage}
\begin{minipage}[t]{0.25\textwidth}\vspace{0pt}
\begin{itemize}[nosep,leftmargin=1em,labelwidth=*,align=left]
	\setlength{\itemsep}{0pt}
	\item Yesterworld/ Punisch My Heaven
	\item The Treason Wall
	\item Therein
	\item Final Resistance
	\item Focus Shift
	\item The Lesser Faith
	\item Edenspring
	\item Insanity's Crescendo
	\item Misery's Crown
\end{itemize}
\end{minipage}
%=============
\begin{minipage}[t]{0.25\textwidth}\vspace{0pt}
\captionsetup{type=figure}
\includegraphics[width=\textwidth]{Images/cover.png}
\caption*{Construct (2013)}
\end{minipage}
\begin{minipage}[t]{0.25\textwidth}\vspace{0pt}
\begin{itemize}[nosep,leftmargin=1em,labelwidth=*,align=left]
	\setlength{\itemsep}{0pt}
	\item For Broken Words
	\item The Science Of Noise
	\item Apathetic
\end{itemize}
\end{minipage}
%=============
\begin{minipage}[t]{0.25\textwidth}\vspace{0pt}
\captionsetup{type=figure}
\includegraphics[width=\textwidth]{Images/cover.png}
\caption*{Atoma (2016)}
\end{minipage}
\begin{minipage}[t]{0.25\textwidth}\vspace{0pt}
\begin{itemize}[nosep,leftmargin=1em,labelwidth=*,align=left]
	\setlength{\itemsep}{0pt}
	\item Forward Momentum
	\item Encircled
	\item Atoma
	\item The Pitiless
\end{itemize}

\end{minipage}
%=============
\begin{minipage}[t]{0.25\textwidth}\vspace{0pt}
\captionsetup{type=figure}
\includegraphics[width=\textwidth]{Images/cover.png}
\caption*{The Absolute (Single 2016)}
\end{minipage}
\begin{minipage}[t]{0.25\textwidth}\vspace{0pt}
\begin{itemize}[nosep,leftmargin=1em,labelwidth=*,align=left]
	\setlength{\itemsep}{0pt}
	\item The Absolute
	\item Time Out Of Place 
\end{itemize}
\end{minipage}

\subsection{Be'lakor}

%=============
\begin{minipage}[t]{0.25\textwidth}\vspace{0pt}
\captionsetup{type=figure}
\includegraphics[width=\textwidth]{Images/cover.png}
\caption*{Vessels (2016)}
\end{minipage}
\begin{minipage}[t]{0.25\textwidth}\vspace{0pt}
\begin{itemize}[nosep,leftmargin=1em,labelwidth=*,align=left]
	\setlength{\itemsep}{0pt}
	\item The Smoke Of Many Fires
	\item Grasping Light
\end{itemize}
\end{minipage}
%=============

\subsection{Parius}

%=============
\begin{minipage}[t]{0.25\textwidth}\vspace{0pt}
\captionsetup{type=figure}
\includegraphics[width=\textwidth]{Images/cover.png}
\caption*{The Eldritch Realm (2018)}
\end{minipage}
\begin{minipage}[t]{0.25\textwidth}\vspace{0pt}
\begin{itemize}[nosep,leftmargin=1em,labelwidth=*,align=left]
	\setlength{\itemsep}{0pt}
	\item Eldritch
\end{itemize}
\end{minipage}

\subsection{Amon Amarth}

%=============
\begin{minipage}[t]{0.25\textwidth}\vspace{0pt}
\captionsetup{type=figure}
\includegraphics[width=\textwidth]{Images/cover.png}
\caption*{Once Sent From The Golden Hall (1997)}
\end{minipage}
\begin{minipage}[t]{0.25\textwidth}\vspace{0pt}
\begin{itemize}[nosep,leftmargin=1em,labelwidth=*,align=left]
	\setlength{\itemsep}{0pt}
	\item Victorious March
\end{itemize}
\end{minipage}
%=============
\begin{minipage}[t]{0.25\textwidth}\vspace{0pt}
\captionsetup{type=figure}
\includegraphics[width=\textwidth]{Images/cover.png}
\caption*{The Avenger (1999)}
\end{minipage}
\begin{minipage}[t]{0.25\textwidth}\vspace{0pt}
\begin{itemize}[nosep,leftmargin=1em,labelwidth=*,align=left]
	\setlength{\itemsep}{0pt}
	\item The Last With Pagan Blood
\end{itemize}
\end{minipage}
%=============
\begin{minipage}[t]{0.25\textwidth}\vspace{0pt}
\captionsetup{type=figure}
\includegraphics[width=\textwidth]{Images/cover.png}
\caption*{Fate Of Norns (2004)}
\end{minipage}
\begin{minipage}[t]{0.25\textwidth}\vspace{0pt}
\begin{itemize}[nosep,leftmargin=1em,labelwidth=*,align=left]
	\setlength{\itemsep}{0pt}
	\item The Pursuit Of Vikings
	\item Valkyries Ride
\end{itemize}
\end{minipage}
%=============
\begin{minipage}[t]{0.25\textwidth}\vspace{0pt}
\captionsetup{type=figure}
\includegraphics[width=\textwidth]{Images/cover.png}
\caption*{With Oden On Our Side (2006)}
\end{minipage}
\begin{minipage}[t]{0.25\textwidth}\vspace{0pt}
\begin{itemize}[nosep,leftmargin=1em,labelwidth=*,align=left]
	\setlength{\itemsep}{0pt}
	\item Cry Of The Black Birds
	\item Runes To My Memory
	\item With Oden On Our Side
\end{itemize}
\end{minipage}
%=============
\begin{minipage}[t]{0.25\textwidth}\vspace{0pt}
\captionsetup{type=figure}
\includegraphics[width=\textwidth]{Images/cover.png}
\caption*{Twilight Of The Thunder God (2008)}
\end{minipage}
\begin{minipage}[t]{0.25\textwidth}\vspace{0pt}
\begin{itemize}[nosep,leftmargin=1em,labelwidth=*,align=left]
	\setlength{\itemsep}{0pt}
	\item Guardians Of Asgaard
	\item Tattered Banners And Bloody Flags
	\item Twilight Of The Thunder God
	\item Varyags Of Miklagaard
\end{itemize}
\end{minipage}
%=============
\begin{minipage}[t]{0.25\textwidth}\vspace{0pt}
\captionsetup{type=figure}
\includegraphics[width=\textwidth]{Images/cover.png}
\caption*{Surtur Rising (2011)}
\end{minipage}
\begin{minipage}[t]{0.25\textwidth}\vspace{0pt}
\begin{itemize}[nosep,leftmargin=1em,labelwidth=*,align=left]
	\setlength{\itemsep}{0pt}
	\item The Last Stand Of Frej
	\item Live Without Regrets
	\item Destroyer Of The Universe
\end{itemize}
\end{minipage}
%=============
\begin{minipage}[t]{0.25\textwidth}\vspace{0pt}
\captionsetup{type=figure}
\includegraphics[width=\textwidth]{Images/cover.png}
\caption*{Deceiver Of The Gods (2013)}
\end{minipage}
\begin{minipage}[t]{0.25\textwidth}\vspace{0pt}
\begin{itemize}[nosep,leftmargin=1em,labelwidth=*,align=left]
	\setlength{\itemsep}{0pt}
	\item As Loke Falls
	\item Deceiver Of The Gods
	\item We Shall Destroy
\end{itemize}
\end{minipage}
%=============
\begin{minipage}[t]{0.25\textwidth}\vspace{0pt}
\captionsetup{type=figure}
\includegraphics[width=\textwidth]{Images/cover.png}
\caption*{Jomsviking (2016)}
\end{minipage}
\begin{minipage}[t]{0.25\textwidth}\vspace{0pt}
\begin{itemize}[nosep,leftmargin=1em,labelwidth=*,align=left]
	\setlength{\itemsep}{0pt}
	\item Raise Your Horns
	\item One Thousand Burning Arrows
	\item The Way Of Vikings
	\item One Against All
\end{itemize}
\end{minipage}

\subsection{At The Gates}

%=============
\begin{minipage}[t]{0.25\textwidth}\vspace{0pt}
\captionsetup{type=figure}
\includegraphics[width=\textwidth]{Images/cover.png}
\caption*{Slaughter Of The Soul (1995)}
\end{minipage}
\begin{minipage}[t]{0.25\textwidth}\vspace{0pt}
\begin{itemize}[nosep,leftmargin=1em,labelwidth=*,align=left]
	\setlength{\itemsep}{0pt}
	\item Slaughter Of The Soul
	\item Blinded By Fear
	\item Under A Serpent Sun
	\item Cold
\end{itemize}
\end{minipage}
%=============
\begin{minipage}[t]{0.25\textwidth}\vspace{0pt}
\captionsetup{type=figure}
\includegraphics[width=\textwidth]{Images/cover.png}
\caption*{At War With Reality (2015)}
\end{minipage}
\begin{minipage}[t]{0.25\textwidth}\vspace{0pt}
\begin{itemize}[nosep,leftmargin=1em,labelwidth=*,align=left]
	\setlength{\itemsep}{0pt}
	\item At War With Reality
	\item The Head Of The Hydra
\end{itemize}
\end{minipage}
%=============
\begin{minipage}[t]{0.25\textwidth}\vspace{0pt}
\captionsetup{type=figure}
\includegraphics[width=\textwidth]{Images/cover.png}
\caption*{To Drink From The Night Itself (2018)}
\end{minipage}
\begin{minipage}[t]{0.25\textwidth}\vspace{0pt}
\begin{itemize}[nosep,leftmargin=1em,labelwidth=*,align=left]
	\setlength{\itemsep}{0pt}
	\item To Drink From The Night Itself
	\item The Mirror Black
\end{itemize}
\end{minipage}

\subsection{Orbit Culture}

%=============
\begin{minipage}[t]{0.25\textwidth}\vspace{0pt}
\captionsetup{type=figure}
\includegraphics[width=\textwidth]{Images/cover.png}
\caption*{Odyssey (2013)}
\end{minipage}
\begin{minipage}[t]{0.25\textwidth}\vspace{0pt}
\begin{itemize}[nosep,leftmargin=1em,labelwidth=*,align=left]
	\setlength{\itemsep}{0pt}
	\item Wildfire
	\item Odyssey
\end{itemize}
\end{minipage}
%=============
\begin{minipage}[t]{0.25\textwidth}\vspace{0pt}
\captionsetup{type=figure}
\includegraphics[width=\textwidth]{Images/cover.png}
\caption*{In Medias Res (2014)}
\end{minipage}
\begin{minipage}[t]{0.25\textwidth}\vspace{0pt}
\begin{itemize}[nosep,leftmargin=1em,labelwidth=*,align=left]
	\setlength{\itemsep}{0pt}
	\item Blacksphere
	\item Kalabalik
	\item Obscurity
\end{itemize}
\end{minipage}
%=============
\begin{minipage}[t]{0.25\textwidth}\vspace{0pt}
\captionsetup{type=figure}
\includegraphics[width=\textwidth]{Images/cover.png}
\caption*{Rasen (2016)}
\end{minipage}
\begin{minipage}[t]{0.25\textwidth}\vspace{0pt}
\begin{itemize}[nosep,leftmargin=1em,labelwidth=*,align=left]
	\setlength{\itemsep}{0pt}
	\item Sun Of All
	\item Svartport
	\item I, The Wolf
	\item Wings Of Dragons
	\item Rasen
	\item Dawn Of Light
	\item The Haste To The Pyre
	\item The Umbilical Chord
\end{itemize}
\end{minipage}
%=============
\begin{minipage}[t]{0.25\textwidth}\vspace{0pt}
\captionsetup{type=figure}
\includegraphics[width=\textwidth]{Images/cover.png}
\caption*{Redfog (2018)}
\end{minipage}
\begin{minipage}[t]{0.25\textwidth}\vspace{0pt}
\begin{itemize}[nosep,leftmargin=1em,labelwidth=*,align=left]
	\setlength{\itemsep}{0pt}
	\item See Through Me
	\item The Newborn One
\end{itemize}
\end{minipage}

\subsection{Raunchy}

%=============
\begin{minipage}[t]{0.25\textwidth}\vspace{0pt}
\captionsetup{type=figure}
\includegraphics[width=\textwidth]{Images/cover.png}
\caption*{Death Pop Romance (2006)}
\end{minipage}
\begin{minipage}[t]{0.25\textwidth}\vspace{0pt}
\begin{itemize}[nosep,leftmargin=1em,labelwidth=*,align=left]
	\setlength{\itemsep}{0pt}
	\item This Legend Forever
	\item Remembrance
	\item Persistance
\end{itemize}
\end{minipage}
%=============
\begin{minipage}[t]{0.25\textwidth}\vspace{0pt}
\captionsetup{type=figure}
\includegraphics[width=\textwidth]{Images/cover.png}
\caption*{Wasteland Discotheque (2014)}
\end{minipage}
\begin{minipage}[t]{0.25\textwidth}\vspace{0pt}
\begin{itemize}[nosep,leftmargin=1em,labelwidth=*,align=left]
	\setlength{\itemsep}{0pt}
	\item Somewhere Along The Road
	\item Somebody's Watching Me
\end{itemize}
\end{minipage}


\subsection{Numenorean}

%=============
\begin{minipage}[t]{0.25\textwidth}\vspace{0pt}
\captionsetup{type=figure}
\includegraphics[width=\textwidth]{Images/cover.png}
\caption*{Adore (2019)}
\end{minipage}
\begin{minipage}[t]{0.25\textwidth}\vspace{0pt}
\begin{itemize}[nosep,leftmargin=1em,labelwidth=*,align=left]
	\setlength{\itemsep}{0pt}
	\item Regret
	\item Coma
	\item Adore
\end{itemize}
\end{minipage}
%=============

\subsection{Omnium Gatherum}

%=============
\begin{minipage}[t]{0.25\textwidth}\vspace{0pt}
\captionsetup{type=figure}
\includegraphics[width=\textwidth]{Images/cover.png}
\caption*{Beyond (2013)}
\end{minipage}
\begin{minipage}[t]{0.25\textwidth}\vspace{0pt}
\begin{itemize}[nosep,leftmargin=1em,labelwidth=*,align=left]
	\setlength{\itemsep}{0pt}
	\item New Dynamic
	\item The Unknowing
\end{itemize}
\end{minipage}
%=============
\begin{minipage}[t]{0.25\textwidth}\vspace{0pt}
\captionsetup{type=figure}
\includegraphics[width=\textwidth]{Images/cover.png}
\caption*{Grey Havens (2016)}
\end{minipage}
\begin{minipage}[t]{0.25\textwidth}\vspace{0pt}
\begin{itemize}[nosep,leftmargin=1em,labelwidth=*,align=left]
	\setlength{\itemsep}{0pt}
	\item Ophidian Sunrise
	\item These Grey Havens
	\item The Pit
	\item Skyline	
\end{itemize}
\end{minipage}
%=============
\begin{minipage}[t]{0.25\textwidth}\vspace{0pt}
\captionsetup{type=figure}
\includegraphics[width=\textwidth]{Images/cover.png}
\caption*{The Burning Cold (2018)}
\end{minipage}
\begin{minipage}[t]{0.25\textwidth}\vspace{0pt}
\begin{itemize}[nosep,leftmargin=1em,labelwidth=*,align=left]
	\setlength{\itemsep}{0pt}
	\item Driven By Conflict
\end{itemize}
\end{minipage}

\subsection{Wolfheart}

%=============
\begin{minipage}[t]{0.25\textwidth}\vspace{0pt}
\captionsetup{type=figure}
\includegraphics[width=\textwidth]{Images/cover.png}
\caption*{Constellation Of The Black Light (2018)}
\end{minipage}
\begin{minipage}[t]{0.25\textwidth}\vspace{0pt}
\begin{itemize}[nosep,leftmargin=1em,labelwidth=*,align=left]
	\setlength{\itemsep}{0pt}
	\item Breakwater
	\item Warfare
\end{itemize}
\end{minipage}


\subsection{Gormathon}

%=============
\begin{minipage}[t]{0.25\textwidth}\vspace{0pt}
\captionsetup{type=figure}
\includegraphics[width=\textwidth]{Images/cover.png}
\caption*{Following The Beast (2014)}
\end{minipage}
\begin{minipage}[t]{0.25\textwidth}\vspace{0pt}
\begin{itemize}[nosep,leftmargin=1em,labelwidth=*,align=left]
	\setlength{\itemsep}{0pt}
	\item Remember
\end{itemize}
\end{minipage}

\subsection{Nekrogoblikon}

%=============
\begin{minipage}[t]{0.25\textwidth}\vspace{0pt}
\captionsetup{type=figure}
\includegraphics[width=\textwidth]{Images/cover.png}
\caption*{Welcome To Bunkers (2018)}
\end{minipage}
\begin{minipage}[t]{0.25\textwidth}\vspace{0pt}
\begin{itemize}[nosep,leftmargin=1em,labelwidth=*,align=left]
	\setlength{\itemsep}{0pt}
	\item The Many Faces Of Dr. Hubert Malbec
\end{itemize}
\end{minipage}

\subsection{Amorphis}

%=============
\begin{minipage}[t]{0.25\textwidth}\vspace{0pt}
\captionsetup{type=figure}
\includegraphics[width=\textwidth]{Images/cover.png}
\caption*{Under The Red Cloud (2015)}
\end{minipage}
\begin{minipage}[t]{0.25\textwidth}\vspace{0pt}
\begin{itemize}[nosep,leftmargin=1em,labelwidth=*,align=left]
	\setlength{\itemsep}{0pt}
	\item Bad Blood
	\item White Night
	\item Death Of A King
\end{itemize}
\end{minipage}

\subsection{Bloodred Hourglass}

%=============
\begin{minipage}[t]{0.25\textwidth}\vspace{0pt}
\captionsetup{type=figure}
\includegraphics[width=\textwidth]{Images/cover.png}
\caption*{Where The Oceans Burn (2015)}
\end{minipage}
\begin{minipage}[t]{0.25\textwidth}\vspace{0pt}
\begin{itemize}[nosep,leftmargin=1em,labelwidth=*,align=left]
	\setlength{\itemsep}{0pt}
	\item Where The Sinners Crawl
	\item Perdition
	\item Valkyrie
\end{itemize}
\end{minipage}
%=============
\begin{minipage}[t]{0.25\textwidth}\vspace{0pt}
\captionsetup{type=figure}
\includegraphics[width=\textwidth]{Images/cover.png}
\caption*{Heal (2017)}
\end{minipage}
\begin{minipage}[t]{0.25\textwidth}\vspace{0pt}
\begin{itemize}[nosep,leftmargin=1em,labelwidth=*,align=left]
	\setlength{\itemsep}{0pt}
	\item Quiet Complaint
\end{itemize}
\end{minipage}


\subsection{Insomnium}

%=============
\begin{minipage}[t]{0.25\textwidth}\vspace{0pt}
\captionsetup{type=figure}
\includegraphics[width=\textwidth]{Images/cover.png}
\caption*{One For Sorrow (2011)}
\end{minipage}
\begin{minipage}[t]{0.25\textwidth}\vspace{0pt}
\begin{itemize}[nosep,leftmargin=1em,labelwidth=*,align=left]
	\setlength{\itemsep}{0pt}
	\item One For Sorrow
	\item Inertia
	\item Through The Shadow
\end{itemize}
\end{minipage}
%=============
\begin{minipage}[t]{0.25\textwidth}\vspace{0pt}
\captionsetup{type=figure}
\includegraphics[width=\textwidth]{Images/cover.png}
\caption*{Weather The Storm (Single 2011)}
\end{minipage}
\begin{minipage}[t]{0.25\textwidth}\vspace{0pt}
\begin{itemize}[nosep,leftmargin=1em,labelwidth=*,align=left]
	\setlength{\itemsep}{0pt}
	\item Weather The Storm
\end{itemize}
\end{minipage}
%=============
\begin{minipage}[t]{0.25\textwidth}\vspace{0pt}
\captionsetup{type=figure}
\includegraphics[width=\textwidth]{Images/cover.png}
\caption*{Winter's Gate (2016)}
\end{minipage}
\begin{minipage}[t]{0.25\textwidth}\vspace{0pt}
\begin{itemize}[nosep,leftmargin=1em,labelwidth=*,align=left]
	\setlength{\itemsep}{0pt}
	\item Winter's Gate
\end{itemize}
\end{minipage}

\subsection{Soilwork}

%=============
\begin{minipage}[t]{0.25\textwidth}\vspace{0pt}
\captionsetup{type=figure}
\includegraphics[width=\textwidth]{Images/cover.png}
\caption*{The Panic Broadcast (2010)}
\end{minipage}
\begin{minipage}[t]{0.25\textwidth}\vspace{0pt}
\begin{itemize}[nosep,leftmargin=1em,labelwidth=*,align=left]
	\setlength{\itemsep}{0pt}
	\item Let This River Flow
\end{itemize}
\end{minipage}
%=============
\begin{minipage}[t]{0.25\textwidth}\vspace{0pt}
\captionsetup{type=figure}
\includegraphics[width=\textwidth]{Images/cover.png}
\caption*{The Ride Majestic (2015)}
\end{minipage}
\begin{minipage}[t]{0.25\textwidth}\vspace{0pt}
\begin{itemize}[nosep,leftmargin=1em,labelwidth=*,align=left]
	\setlength{\itemsep}{0pt}
	\item The Ride Majestic (Aspire Angelic)
	\item Death In General
	\item Whirl Of Pain
\end{itemize}
\end{minipage}


\subsection{Mors Principium Est}

%=============
\begin{minipage}[t]{0.25\textwidth}\vspace{0pt}
\captionsetup{type=figure}
\includegraphics[width=\textwidth]{Images/cover.png}
\caption*{Embers Of A Dying World (2017)}
\end{minipage}
\begin{minipage}[t]{0.25\textwidth}\vspace{0pt}
\begin{itemize}[nosep,leftmargin=1em,labelwidth=*,align=left]
	\setlength{\itemsep}{0pt}
	\item Masquerade
	\item Reclaim The Sun
\end{itemize}
\end{minipage}

\subsection{Parasite Inc.}

%=============
\begin{minipage}[t]{0.25\textwidth}\vspace{0pt}
\captionsetup{type=figure}
\includegraphics[width=\textwidth]{Images/cover.png}
\caption*{Time Tears Down (2013)}
\end{minipage}
\begin{minipage}[t]{0.25\textwidth}\vspace{0pt}
\begin{itemize}[nosep,leftmargin=1em,labelwidth=*,align=left]
	\setlength{\itemsep}{0pt}
	\item The Pulse Of The Dead
\end{itemize}
\end{minipage}

\subsection{Meadows End}

%=============
\begin{minipage}[t]{0.25\textwidth}\vspace{0pt}
\captionsetup{type=figure}
\includegraphics[width=\textwidth]{Images/cover.png}
\caption*{Sojourn (2016)}
\end{minipage}
\begin{minipage}[t]{0.25\textwidth}\vspace{0pt}
\begin{itemize}[nosep,leftmargin=1em,labelwidth=*,align=left]
	\setlength{\itemsep}{0pt}
	\item Heathens' Embrace
\end{itemize}
\end{minipage}

\subsection{Arch Enemy}

%=============
\begin{minipage}[t]{0.25\textwidth}\vspace{0pt}
\captionsetup{type=figure}
\includegraphics[width=\textwidth]{Images/cover.png}
\caption*{War Eternal (2014)}
\end{minipage}
\begin{minipage}[t]{0.25\textwidth}\vspace{0pt}
\begin{itemize}[nosep,leftmargin=1em,labelwidth=*,align=left]
	\setlength{\itemsep}{0pt}
	\item No More Regrets
	\item You Will Know My Name
	\item War Eternal
	\item Stolen Life
\end{itemize}
\end{minipage}
%=============
\begin{minipage}[t]{0.25\textwidth}\vspace{0pt}
\captionsetup{type=figure}
\includegraphics[width=\textwidth]{Images/cover.png}
\caption*{As The Stages Burn (2017)}
\end{minipage}
\begin{minipage}[t]{0.25\textwidth}\vspace{0pt}
\begin{itemize}[nosep,leftmargin=1em,labelwidth=*,align=left]
	\setlength{\itemsep}{0pt}
	\item Bloodstained Cross
	\item Nemesis
	\item Yesterday Is Dead And Gone
	\item Ravenous
\end{itemize}
\end{minipage}
%=============
\begin{minipage}[t]{0.25\textwidth}\vspace{0pt}
\captionsetup{type=figure}
\includegraphics[width=\textwidth]{Images/cover.png}
\caption*{Will To Power (2017)}
\end{minipage}
\begin{minipage}[t]{0.25\textwidth}\vspace{0pt}
\begin{itemize}[nosep,leftmargin=1em,labelwidth=*,align=left]
	\setlength{\itemsep}{0pt}
	\item The World Is Yours
	\item A Fight I Must Win
	\item Reason To Believe
\end{itemize}
\end{minipage}

\subsection{Children Of Bodom}

%=============
\begin{minipage}[t]{0.25\textwidth}\vspace{0pt}
\captionsetup{type=figure}
\includegraphics[width=\textwidth]{Images/cover.png}
\caption*{Skeletons In The Closet (2009)}
\end{minipage}
\begin{minipage}[t]{0.25\textwidth}\vspace{0pt}
\begin{itemize}[nosep,leftmargin=1em,labelwidth=*,align=left]
	\setlength{\itemsep}{0pt}
	\item Lookin' Out My Back Door
	\item She Is Beautiful
	\item Somebody Put Something In My Drink
\end{itemize}
\end{minipage}
%=============
\begin{minipage}[t]{0.25\textwidth}\vspace{0pt}
\captionsetup{type=figure}
\includegraphics[width=\textwidth]{Images/cover.png}
\caption*{Holiday At Lake Bodom (2015)}
\end{minipage}
\begin{minipage}[t]{0.25\textwidth}\vspace{0pt}
\begin{itemize}[nosep,leftmargin=1em,labelwidth=*,align=left]
	\setlength{\itemsep}{0pt}
	\item Needled 24\/7
	\item Are You Dead Yet
	\item Everytime I Die
	\item I'm Shipping Up To Boston
	\item Jessie's Girl
\end{itemize}
\end{minipage}

\subsection{Shylmagoghnar}

%=============
\begin{minipage}[t]{0.25\textwidth}\vspace{0pt}
\captionsetup{type=figure}
\includegraphics[width=\textwidth]{Images/cover.png}
\caption*{Emergence (2014)}
\end{minipage}
\begin{minipage}[t]{0.25\textwidth}\vspace{0pt}
\begin{itemize}[nosep,leftmargin=1em,labelwidth=*,align=left]
	\setlength{\itemsep}{0pt}
	\item I Am The Abyss
	\item Emergence
	\item Edin In Ashes
\end{itemize}
\end{minipage}
%=============
\begin{minipage}[t]{0.25\textwidth}\vspace{0pt}
\captionsetup{type=figure}
\includegraphics[width=\textwidth]{Images/cover.png}
\caption*{Transience (2018)}
\end{minipage}
\begin{minipage}[t]{0.25\textwidth}\vspace{0pt}
\begin{itemize}[nosep,leftmargin=1em,labelwidth=*,align=left]
	\setlength{\itemsep}{0pt}
	\item The Chosen Path
\end{itemize}
\end{minipage}

\subsection{In Mourning}

%=============
\begin{minipage}[t]{0.25\textwidth}\vspace{0pt}
\captionsetup{type=figure}
\includegraphics[width=\textwidth]{Images/cover.png}
\caption*{Afterglow (2016)}
\end{minipage}
\begin{minipage}[t]{0.25\textwidth}\vspace{0pt}
\begin{itemize}[nosep,leftmargin=1em,labelwidth=*,align=left]
	\setlength{\itemsep}{0pt}
	\item The Lighthouse Keeper
\end{itemize}
\end{minipage}

\subsection{Deadlock}

%=============
\begin{minipage}[t]{0.25\textwidth}\vspace{0pt}
\captionsetup{type=figure}
\includegraphics[width=\textwidth]{Images/cover.png}
\caption*{Hybris (2016)}
\end{minipage}
\begin{minipage}[t]{0.25\textwidth}\vspace{0pt}
\begin{itemize}[nosep,leftmargin=1em,labelwidth=*,align=left]
	\setlength{\itemsep}{0pt}
	\item Ein Deutsches Requiem
	\item Berserk
\end{itemize}
\end{minipage}

%===========================================
% TECHNICAL DEATH METAL
%===========================================
\cleardoublepage
\section{Technical Death Metal}

\subsection{Cattle Decapitation}

%=============
\begin{minipage}[t]{0.25\textwidth}\vspace{0pt}
\captionsetup{type=figure}
\includegraphics[width=\textwidth]{Images/cover.png}
\caption*{The Anthroprocene Extinction (2015)}
\end{minipage}
\begin{minipage}[t]{0.25\textwidth}\vspace{0pt}
\begin{itemize}[nosep,leftmargin=1em,labelwidth=*,align=left]
	\setlength{\itemsep}{0pt}
	\item Manufactured Extinct
\end{itemize}
\end{minipage}

\subsection{Kardashev}

%=============
\begin{minipage}[t]{0.25\textwidth}\vspace{0pt}
\captionsetup{type=figure}
\includegraphics[width=\textwidth]{Images/cover.png}
\caption*{The Almanac (Instrumental) (2018)}
\end{minipage}
\begin{minipage}[t]{0.25\textwidth}\vspace{0pt}
\begin{itemize}[nosep,leftmargin=1em,labelwidth=*,align=left]
	\setlength{\itemsep}{0pt}
	\item Beside Cliffs and Chasms 
\end{itemize}
\end{minipage}

\subsection{Cytotoxin}

%=============
\begin{minipage}[t]{0.25\textwidth}\vspace{0pt}
\captionsetup{type=figure}
\includegraphics[width=\textwidth]{Images/cover.png}
\caption*{Gammageddon (2017)}
\end{minipage}
\begin{minipage}[t]{0.25\textwidth}\vspace{0pt}
\begin{itemize}[nosep,leftmargin=1em,labelwidth=*,align=left]
	\setlength{\itemsep}{0pt}
	\item Chernopolis
\end{itemize}
\end{minipage}

\subsection{Nile}

%=============
\begin{minipage}[t]{0.25\textwidth}\vspace{0pt}
\captionsetup{type=figure}
\includegraphics[width=\textwidth]{Images/cover.png}
\caption*{What Should Not Be Unearthed (2015)}
\end{minipage}
\begin{minipage}[t]{0.25\textwidth}\vspace{0pt}
\begin{itemize}[nosep,leftmargin=1em,labelwidth=*,align=left]
	\setlength{\itemsep}{0pt}
	\item Call To Destruction
\end{itemize}
\end{minipage}

\subsection{Ophidius}

%=============
\begin{minipage}[t]{0.25\textwidth}\vspace{0pt}
\captionsetup{type=figure}
\includegraphics[width=\textwidth]{Images/cover.png}
\caption*{The Way Of The Voice (2016)}
\end{minipage}
\begin{minipage}[t]{0.25\textwidth}\vspace{0pt}
\begin{itemize}[nosep,leftmargin=1em,labelwidth=*,align=left]
	\setlength{\itemsep}{0pt}
	\item The Calling
	\item Fo Sivaas
\end{itemize}
\end{minipage}

\subsection{Obscura}

%=============
\begin{minipage}[t]{0.25\textwidth}\vspace{0pt}
\captionsetup{type=figure}
\includegraphics[width=\textwidth]{Images/cover.png}
\caption*{Omnivium (2011)}
\end{minipage}
\begin{minipage}[t]{0.25\textwidth}\vspace{0pt}
\begin{itemize}[nosep,leftmargin=1em,labelwidth=*,align=left]
	\setlength{\itemsep}{0pt}
	\item Vortex Omnivium
	\item Ocean Gateways
\end{itemize}
\end{minipage}

\subsection{Solipsismo}

%=============
\begin{minipage}[t]{0.25\textwidth}\vspace{0pt}
\captionsetup{type=figure}
\includegraphics[width=\textwidth]{Images/cover.png}
\caption*{Sangre Antigua (2017)}
\end{minipage}
\begin{minipage}[t]{0.25\textwidth}\vspace{0pt}
\begin{itemize}[nosep,leftmargin=1em,labelwidth=*,align=left]
	\setlength{\itemsep}{0pt}
	\item 8C2
	\item Reyes Y Dioses
	\item Violenta Transfusion
\end{itemize}
\end{minipage}

\subsection{Aephanemer}

%=============
\begin{minipage}[t]{0.25\textwidth}\vspace{0pt}
\captionsetup{type=figure}
\includegraphics[width=\textwidth]{Images/cover.png}
\caption*{Know Thyself (2014)}
\end{minipage}
\begin{minipage}[t]{0.25\textwidth}\vspace{0pt}
\begin{itemize}[nosep,leftmargin=1em,labelwidth=*,align=left]
	\setlength{\itemsep}{0pt}
	\item Alive
	\item Resilience
\end{itemize}
\end{minipage}
%=============
\begin{minipage}[t]{0.25\textwidth}\vspace{0pt}
\captionsetup{type=figure}
\includegraphics[width=\textwidth]{Images/cover.png}
\caption*{Memento Mori (2016)}
\end{minipage}
\begin{minipage}[t]{0.25\textwidth}\vspace{0pt}
\begin{itemize}[nosep,leftmargin=1em,labelwidth=*,align=left]
	\setlength{\itemsep}{0pt}
	\item Unstoppable
\end{itemize}
\end{minipage}

\subsection{Aether}

%=============
\begin{minipage}[t]{0.25\textwidth}\vspace{0pt}
\captionsetup{type=figure}
\includegraphics[width=\textwidth]{Images/cover.png}
\caption*{Tale Of Fire (2016)}
\end{minipage}
\begin{minipage}[t]{0.25\textwidth}\vspace{0pt}
\begin{itemize}[nosep,leftmargin=1em,labelwidth=*,align=left]
	\setlength{\itemsep}{0pt}
	\item Tale Of Fire
	\item Last Battle
\end{itemize}
\end{minipage}

\subsection{Aetheric}

%=============
\begin{minipage}[t]{0.25\textwidth}\vspace{0pt}
\captionsetup{type=figure}
\includegraphics[width=\textwidth]{Images/cover.png}
\caption*{Serpents Beneath The Shrine (2017)}
\end{minipage}
\begin{minipage}[t]{0.25\textwidth}\vspace{0pt}
\begin{itemize}[nosep,leftmargin=1em,labelwidth=*,align=left]
	\setlength{\itemsep}{0pt}
	\item By Death Posessed
\end{itemize}
\end{minipage}

\subsection{Aronius}

%=============
\begin{minipage}[t]{0.25\textwidth}\vspace{0pt}
\captionsetup{type=figure}
\includegraphics[width=\textwidth]{Images/cover.png}
\caption*{Truth In Perception (2014)}
\end{minipage}
\begin{minipage}[t]{0.25\textwidth}\vspace{0pt}
\begin{itemize}[nosep,leftmargin=1em,labelwidth=*,align=left]
	\setlength{\itemsep}{0pt}
	\item Disillusionment I
	\item Truth In Perception
\end{itemize}
\end{minipage}

\subsection{Fallujah}

%=============
\begin{minipage}[t]{0.25\textwidth}\vspace{0pt}
\captionsetup{type=figure}
\includegraphics[width=\textwidth]{Images/cover.png}
\caption*{Nomadic (2013)}
\end{minipage}
\begin{minipage}[t]{0.25\textwidth}\vspace{0pt}
\begin{itemize}[nosep,leftmargin=1em,labelwidth=*,align=left]
	\setlength{\itemsep}{0pt}
	\item The Dead Sea
\end{itemize}
\end{minipage}

\subsection{Irreversible Mechanism}

%=============
\begin{minipage}[t]{0.25\textwidth}\vspace{0pt}
\captionsetup{type=figure}
\includegraphics[width=\textwidth]{Images/cover.png}
\caption*{Infinite Fields (2015)}
\end{minipage}
\begin{minipage}[t]{0.25\textwidth}\vspace{0pt}
\begin{itemize}[nosep,leftmargin=1em,labelwidth=*,align=left]
	\setlength{\itemsep}{0pt}
	\item The Betrayer Of Time
\end{itemize}
\end{minipage}

\subsection{Psygnosis}

%=============
\begin{minipage}[t]{0.25\textwidth}\vspace{0pt}
\captionsetup{type=figure}
\includegraphics[width=\textwidth]{Images/cover.png}
\caption*{Neptune (2017)}
\end{minipage}
\begin{minipage}[t]{0.25\textwidth}\vspace{0pt}
\begin{itemize}[nosep,leftmargin=1em,labelwidth=*,align=left]
	\setlength{\itemsep}{0pt}
	\item Psygnosis Is Shit
\end{itemize}
\end{minipage}

\subsection{Transience}

%=============
\begin{minipage}[t]{0.25\textwidth}\vspace{0pt}
\captionsetup{type=figure}
\includegraphics[width=\textwidth]{Images/cover.png}
\caption*{Temple (2015)}
\end{minipage}
\begin{minipage}[t]{0.25\textwidth}\vspace{0pt}
\begin{itemize}[nosep,leftmargin=1em,labelwidth=*,align=left]
	\setlength{\itemsep}{0pt}
	\item Skirmish
\end{itemize}
\end{minipage}

%===========================================
% DEATH
%===========================================
\cleardoublepage
\section{Death Metal}

\subsection{Lamb Of God}

%=============
\begin{minipage}[t]{0.25\textwidth}\vspace{0pt}
\captionsetup{type=figure}
\includegraphics[width=\textwidth]{Images/cover.png}
\caption*{Sacrament (2006)}
\end{minipage}
\begin{minipage}[t]{0.25\textwidth}\vspace{0pt}
\begin{itemize}[nosep,leftmargin=1em,labelwidth=*,align=left]
	\setlength{\itemsep}{0pt}
	\item Redneck
	\item Walk With Me In Hell
\end{itemize}
\end{minipage}
%=============

\subsection{Gojira}

%=============
\begin{minipage}[t]{0.25\textwidth}\vspace{0pt}
\captionsetup{type=figure}
\includegraphics[width=\textwidth]{Images/cover.png}
\caption*{From mars to sirius (2005)}
\end{minipage}
\begin{minipage}[t]{0.25\textwidth}\vspace{0pt}
\begin{itemize}[nosep,leftmargin=1em,labelwidth=*,align=left]
	\setlength{\itemsep}{0pt}
	\item Global Warming
	\item Flying Whales
\end{itemize}
\end{minipage}
%=============
\begin{minipage}[t]{0.25\textwidth}\vspace{0pt}
\captionsetup{type=figure}
\includegraphics[width=\textwidth]{Images/cover.png}
\caption*{The Way of All Flesh  (2008)}
\end{minipage}
\begin{minipage}[t]{0.25\textwidth}\vspace{0pt}
\begin{itemize}[nosep,leftmargin=1em,labelwidth=*,align=left]
	\setlength{\itemsep}{0pt}
	\item Oroborus
	\item The Art of Dying
\end{itemize}
\end{minipage}
%=============
\begin{minipage}[t]{0.25\textwidth}\vspace{0pt}
\captionsetup{type=figure}
\includegraphics[width=\textwidth]{Images/cover.png}
\caption*{L' Enfant Sauvage (2012)}
\end{minipage}
\begin{minipage}[t]{0.25\textwidth}\vspace{0pt}
\begin{itemize}[nosep,leftmargin=1em,labelwidth=*,align=left]
	\setlength{\itemsep}{0pt}
	\item L'Enfant Sauvage
	\item The Gift Of Guilt
	\item Born In Winter
	\item Mouth Of Kala
\end{itemize}
\end{minipage}
%=============
\begin{minipage}[t]{0.25\textwidth}\vspace{0pt}
\captionsetup{type=figure}
\includegraphics[width=\textwidth]{Images/cover.png}
\caption*{Magma (2016)}
\end{minipage}
\begin{minipage}[t]{0.25\textwidth}\vspace{0pt}
\begin{itemize}[nosep,leftmargin=1em,labelwidth=*,align=left]
	\setlength{\itemsep}{0pt}
	\item Low Lands
	\item The Shooting Star
	\item Magma
\end{itemize}
\end{minipage}


\subsection{Deserted Fear}

%=============
\begin{minipage}[t]{0.25\textwidth}\vspace{0pt}
\captionsetup{type=figure}
\includegraphics[width=\textwidth]{Images/cover.png}
\caption*{Dead Shores Rising (2017)}
\end{minipage}
\begin{minipage}[t]{0.25\textwidth}\vspace{0pt}
\begin{itemize}[nosep,leftmargin=1em,labelwidth=*,align=left]
	\setlength{\itemsep}{0pt}
	\item The Fall Of Leaden Skies
	\item Open Their Gates
	\item Towards Humanity
\end{itemize}
\end{minipage}

\subsection{Death}

%=============
\begin{minipage}[t]{0.25\textwidth}\vspace{0pt}
\captionsetup{type=figure}
\includegraphics[width=\textwidth]{Images/cover.png}
\caption*{Symbolic (1995)}
\end{minipage}
\begin{minipage}[t]{0.25\textwidth}\vspace{0pt}
\begin{itemize}[nosep,leftmargin=1em,labelwidth=*,align=left]
	\setlength{\itemsep}{0pt}
	\item Without Judgement
	\item Symbolic
\end{itemize}
\end{minipage}

\subsection{Six Feet Under}

%=============
\begin{minipage}[t]{0.25\textwidth}\vspace{0pt}
\captionsetup{type=figure}
\includegraphics[width=\textwidth]{Images/cover.png}
\caption*{Graveyard Classics IV - The Number Of The Priest (2016)}
\end{minipage}
\begin{minipage}[t]{0.25\textwidth}\vspace{0pt}
\begin{itemize}[nosep,leftmargin=1em,labelwidth=*,align=left]
	\setlength{\itemsep}{0pt}
	\item Genocide
\end{itemize}
\end{minipage}

%===========================================
% DEATHCORE
%===========================================
\cleardoublepage
\section{Deathcore}

\subsection{Thy Art Is Murder}

%=============
\begin{minipage}[t]{0.25\textwidth}\vspace{0pt}
\captionsetup{type=figure}
\includegraphics[width=\textwidth]{Images/cover.png}
\caption*{The Depression Session (Split 2016)}
\end{minipage}
\begin{minipage}[t]{0.25\textwidth}\vspace{0pt}
\begin{itemize}[nosep,leftmargin=1em,labelwidth=*,align=left]
	\setlength{\itemsep}{0pt}
	\item They Will Know Another
	\item Du Hast
\end{itemize}
\end{minipage}

\subsection{Fit For An Autopsy}

%=============
\begin{minipage}[t]{0.25\textwidth}\vspace{0pt}
\captionsetup{type=figure}
\includegraphics[width=\textwidth]{Images/cover.png}
\caption*{The Depression Session (Split 2016)}
\end{minipage}
\begin{minipage}[t]{0.25\textwidth}\vspace{0pt}
\begin{itemize}[nosep,leftmargin=1em,labelwidth=*,align=left]
	\setlength{\itemsep}{0pt}
	\item Flatlining
	\item The Perfect Drug
\end{itemize}
\end{minipage}

\subsection{The Acacia Strain}

%=============
\begin{minipage}[t]{0.25\textwidth}\vspace{0pt}
\captionsetup{type=figure}
\includegraphics[width=\textwidth]{Images/cover.png}
\caption*{The Depression Session (Split 2016)}
\end{minipage}
\begin{minipage}[t]{0.25\textwidth}\vspace{0pt}
\begin{itemize}[nosep,leftmargin=1em,labelwidth=*,align=left]
	\setlength{\itemsep}{0pt}
	\item Sensory Deprivation
	\item Black Hole Sun
\end{itemize}
\end{minipage}
%=============
\begin{minipage}[t]{0.25\textwidth}\vspace{0pt}
\captionsetup{type=figure}
\includegraphics[width=\textwidth]{Images/cover.png}
\caption*{Gravebloom (2017)}
\end{minipage}
\begin{minipage}[t]{0.25\textwidth}\vspace{0pt}
\begin{itemize}[nosep,leftmargin=1em,labelwidth=*,align=left]
	\setlength{\itemsep}{0pt}
	\item Worthless
	\item Gravebloom
	\item Cold Gloom
\end{itemize}
\end{minipage}

\subsection{Whitechapel}

%=============
\begin{minipage}[t]{0.25\textwidth}\vspace{0pt}
\captionsetup{type=figure}
\includegraphics[width=\textwidth]{Images/cover.png}
\caption*{This Is Exile (2008)}
\end{minipage}
\begin{minipage}[t]{0.25\textwidth}\vspace{0pt}
\begin{itemize}[nosep,leftmargin=1em,labelwidth=*,align=left]
	\setlength{\itemsep}{0pt}
	\item Possession 
\end{itemize}
\end{minipage}

\subsection{Suicide Silence}

%=============
\begin{minipage}[t]{0.25\textwidth}\vspace{0pt}
\captionsetup{type=figure}
\includegraphics[width=\textwidth]{Images/cover.png}
\caption*{Suicide Silence (2017)}
\end{minipage}
\begin{minipage}[t]{0.25\textwidth}\vspace{0pt}
\begin{itemize}[nosep,leftmargin=1em,labelwidth=*,align=left]
	\setlength{\itemsep}{0pt}
	\item Don't Be Careful You Might Hurt Yourself
\end{itemize}
\end{minipage}

\subsection{Faith In Ruin}

%=============
\begin{minipage}[t]{0.25\textwidth}\vspace{0pt}
\captionsetup{type=figure}
\includegraphics[width=\textwidth]{Images/cover.png}
\caption*{Anathema (2016)}
\end{minipage}
\begin{minipage}[t]{0.25\textwidth}\vspace{0pt}
\begin{itemize}[nosep,leftmargin=1em,labelwidth=*,align=left]
	\setlength{\itemsep}{0pt}
	\item Newest Dark Power
	\item The Polygon
\end{itemize}
\end{minipage}

%
\newpage

\section{Metalcore}
%===========================================
% Metalcore
%===========================================

%\section{Metalcore}

\subsection{Heaven Shall Burn}

%=============
\begin{minipage}[t]{0.25\textwidth}
\captionsetup{type=figure}
\includegraphics[width=\textwidth]{Images/cover.png}
\caption*{Iconoclast I. - The Final Resistance (2008)}
\end{minipage}
\begin{minipage}[t]{0.25\textwidth}\vspace{0pt}
\begin{itemize}[nosep,leftmargin=1em,labelwidth=*,align=left]
	\setlength{\itemsep}{0pt}
	\item Endzeit
	\item Black Tears
	\item Atonement
\end{itemize}
\end{minipage}
%=============
\begin{minipage}[t]{0.25\textwidth}
\captionsetup{type=figure}
\includegraphics[width=\textwidth]{Images/cover.png}
\caption*{Invictus (2010)}
\end{minipage}
\begin{minipage}[t]{0.25\textwidth}\vspace{0pt}
\begin{itemize}[nosep,leftmargin=1em,labelwidth=*,align=left]
	\setlength{\itemsep}{0pt}
	\item Given in Death
\end{itemize}
\end{minipage}
%=============
\begin{minipage}[t]{0.25\textwidth}
\captionsetup{type=figure}
\includegraphics[width=\textwidth]{Images/cover.png}
\caption*{Veto (2013)}
\end{minipage}
\begin{minipage}[t]{0.25\textwidth}\vspace{0pt}
\begin{itemize}[nosep,leftmargin=1em,labelwidth=*,align=left]
	\setlength{\itemsep}{0pt}
	\item Beyond Redemption
	\item Hunters Will Be Hunted
	\item Godiva
	\item Fallen
\end{itemize}
\end{minipage}
%=============
\begin{minipage}[t]{0.25\textwidth}
\captionsetup{type=figure}
\includegraphics[width=\textwidth]{Images/cover.png}
\caption*{Wanderer (2016)}
\end{minipage}
\begin{minipage}[t]{0.25\textwidth}\vspace{0pt}
\begin{itemize}[nosep,leftmargin=1em,labelwidth=*,align=left]
	\setlength{\itemsep}{0pt}
	\item Passage Of The Crane
	\item The Cry Of Mankind
	\item Prey To God
	\item Corium
	\item Save Me
	\item They Shall Not Pass
\end{itemize}
\end{minipage}

\subsection{While She Sleeps}

%=============
\begin{minipage}[t]{0.25\textwidth}
\captionsetup{type=figure}
\includegraphics[width=\textwidth]{Images/cover.png}
\caption*{This Is The Six (2012)}
\end{minipage}
\begin{minipage}[t]{0.25\textwidth}\vspace{0pt}
\begin{itemize}[nosep,leftmargin=1em,labelwidth=*,align=left]
	\setlength{\itemsep}{0pt}
	\item Satisfied In Suffering
	\item The Chapel
	\item Seven Hills
	\item This Is The Six
	\item False Freedom
\end{itemize}
\end{minipage}
%=============
\begin{minipage}[t]{0.25\textwidth}\vspace{0pt}
\captionsetup{type=figure}
\includegraphics[width=\textwidth]{Images/cover.png}
\caption*{Brainwashed (2015)}
\end{minipage}
\begin{minipage}[t]{0.25\textwidth}\vspace{0pt}
\begin{itemize}[nosep,leftmargin=1em,labelwidth=*,align=left]
	\setlength{\itemsep}{0pt}
	\item Our Legacy
	\item Four Walls
\end{itemize}
\end{minipage}
%=============
\begin{minipage}[t]{0.25\textwidth}\vspace{0pt}
\captionsetup{type=figure}
\includegraphics[width=\textwidth]{Images/cover.png}
\caption*{You Are We (2017)}
\end{minipage}
\begin{minipage}[t]{0.25\textwidth}\vspace{0pt}
\begin{itemize}[nosep,leftmargin=1em,labelwidth=*,align=left]
	\setlength{\itemsep}{0pt}
	\item Hurricane
	\item Steal The Sun
	\item Revolt
\end{itemize}
\end{minipage}
%=============
\begin{minipage}[t]{0.25\textwidth}\vspace{0pt}
\captionsetup{type=figure}
\includegraphics[width=\textwidth]{Images/cover.png}
\caption*{So What (2017)}
\end{minipage}
\begin{minipage}[t]{0.25\textwidth}\vspace{0pt}
\begin{itemize}[nosep,leftmargin=1em,labelwidth=*,align=left]
	\setlength{\itemsep}{0pt}
	\item Elephant
	\item Anti-Social
\end{itemize}
\end{minipage}

\subsection{Architects}

%=============
\begin{minipage}[t]{0.25\textwidth}\vspace{0pt}
\captionsetup{type=figure}
\includegraphics[width=\textwidth]{Images/cover.png}
\caption*{Holy Hell (2018)}
\end{minipage}
\begin{minipage}[t]{0.25\textwidth}\vspace{0pt}
\begin{itemize}[nosep,leftmargin=1em,labelwidth=*,align=left]
	\setlength{\itemsep}{0pt}
	\item Holy Hell
	\item Doomsday
	\item A Wasted Hymn
\end{itemize}
\end{minipage}
%=============
\begin{minipage}[t]{0.25\textwidth}\vspace{0pt}
\captionsetup{type=figure}
\includegraphics[width=\textwidth]{Images/cover.png}
\caption*{Daybreaker (2012)}
\end{minipage}
\begin{minipage}[t]{0.25\textwidth}\vspace{0pt}
\begin{itemize}[nosep,leftmargin=1em,labelwidth=*,align=left]
	\setlength{\itemsep}{0pt}
	\item These Colours Don’t Run
	\item Even if you win, you're still a rat
	\item Blood Bank
	\item Cracks In The Earth
	\item Of Dust And Nations
	\item The Bitter End
\end{itemize}
\end{minipage}

\subsection{The Amity Affliction}

%=============
\begin{minipage}[t]{0.25\textwidth}\vspace{0pt}
\captionsetup{type=figure}
\includegraphics[width=\textwidth]{Images/cover.png}
\caption*{Let The Ocean Take Me (2014)}
\end{minipage}
\begin{minipage}[t]{0.25\textwidth}\vspace{0pt}
\begin{itemize}[nosep,leftmargin=1em,labelwidth=*,align=left]
	\setlength{\itemsep}{0pt}
	\item Pittsburgh
\end{itemize}
\end{minipage}
%=============
\begin{minipage}[t]{0.25\textwidth}\vspace{0pt}
\captionsetup{type=figure}
\includegraphics[width=\textwidth]{Images/cover.png}
\caption*{This Could Be Heartbreak (2016)}
\end{minipage}
\begin{minipage}[t]{0.25\textwidth}\vspace{0pt}
\begin{itemize}[nosep,leftmargin=1em,labelwidth=*,align=left]
	\setlength{\itemsep}{0pt}
	\item I Bring The Weather With Me
	\item This Could Be Heartbreak
\end{itemize}
\end{minipage}

\subsection{August Burns Red}

%=============
\begin{minipage}[t]{0.25\textwidth}\vspace{0pt}
\captionsetup{type=figure}
\includegraphics[width=\textwidth]{Images/cover.png}
\caption*{Sleddin' Hill (2012)}
\end{minipage}
\begin{minipage}[t]{0.25\textwidth}\vspace{0pt}
\begin{itemize}[nosep,leftmargin=1em,labelwidth=*,align=left]
	\setlength{\itemsep}{0pt}
	\item O Come, O Come, Emanuel 
\end{itemize}
\end{minipage}
%=============
\begin{minipage}[t]{0.25\textwidth}\vspace{0pt}
\captionsetup{type=figure}
\includegraphics[width=\textwidth]{Images/cover.png}
\caption*{Winter Wilderness (2018)}
\end{minipage}
\begin{minipage}[t]{0.25\textwidth}\vspace{0pt}
\begin{itemize}[nosep,leftmargin=1em,labelwidth=*,align=left]
	\setlength{\itemsep}{0pt}
	\item Winter Wilderness 
\end{itemize}
\end{minipage}


\subsection{Copia}

%=============
\begin{minipage}[t]{0.25\textwidth}\vspace{0pt}
\captionsetup{type=figure}
\includegraphics[width=\textwidth]{Images/cover.png}
\caption*{Epoch (2017)}
\end{minipage}
\begin{minipage}[t]{0.25\textwidth}\vspace{0pt}
\begin{itemize}[nosep,leftmargin=1em,labelwidth=*,align=left]
	\setlength{\itemsep}{0pt}
	\item Never Forget
\end{itemize}
\end{minipage}

\subsection{Moments}

%=============
\begin{minipage}[t]{0.25\textwidth}
\captionsetup{type=figure}
\includegraphics[width=\textwidth]{Images/cover.png}
\caption*{Clarity (EP) (2016)}
\end{minipage}
\begin{minipage}[t]{0.25\textwidth}\vspace{0pt}
\begin{itemize}[nosep,leftmargin=1em,labelwidth=*,align=left]
	\setlength{\itemsep}{0pt}
	\item Keepsake 
	\item Clarity
	\item Cardinal Closure
\end{itemize}
\end{minipage}

\subsection{Northlane}

%=============
\begin{minipage}[t]{0.25\textwidth}
\captionsetup{type=figure}
\includegraphics[width=\textwidth]{Images/cover.png}
\caption*{Node (2015)}
\end{minipage}
\begin{minipage}[t]{0.25\textwidth}\vspace{0pt}
\begin{itemize}[nosep,leftmargin=1em,labelwidth=*,align=left]
	\setlength{\itemsep}{0pt}
	\item Ohm
	\item Weightless
	\item Ra
	\item Soma
\end{itemize}
\end{minipage}

\subsection{Caliban}

%=============
\begin{minipage}[t]{0.25\textwidth}
\captionsetup{type=figure}
\includegraphics[width=\textwidth]{Images/cover.png}
\caption*{(2016)}
\end{minipage}
\begin{minipage}[t]{0.25\textwidth}\vspace{0pt}
\begin{itemize}[nosep,leftmargin=1em,labelwidth=*,align=left]
	\setlength{\itemsep}{0pt}
	\item Paralyzed
	\item brOKen
	\item Crystal Skies
\end{itemize}
\end{minipage}

\subsection{Annisokay}

%=============
\begin{minipage}[t]{0.25\textwidth}
\captionsetup{type=figure}
\includegraphics[width=\textwidth]{Images/cover.png}
\caption*{The Lucid Dreamer (2012)}
\end{minipage}
\begin{minipage}[t]{0.25\textwidth}\vspace{0pt}
\begin{itemize}[nosep,leftmargin=1em,labelwidth=*,align=left]
	\setlength{\itemsep}{0pt}
	\item Sky
	\item Monstercrazy
	\item Day To Day Tragedy
\end{itemize}
\end{minipage}

\subsection{Trivium}

%=============
\begin{minipage}[t]{0.25\textwidth}
\captionsetup{type=figure}
\includegraphics[width=\textwidth]{Images/cover.png}
\caption*{Silence In The Snow (2015)}
\end{minipage}
\begin{minipage}[t]{0.25\textwidth}\vspace{0pt}
\begin{itemize}[nosep,leftmargin=1em,labelwidth=*,align=left]
	\setlength{\itemsep}{0pt}
	\item Silence In The Snow
\end{itemize}
\end{minipage}

\subsection{Rise Of The Northstar}

%=============
\begin{minipage}[t]{0.25\textwidth}\vspace{0pt}
\captionsetup{type=figure}
\includegraphics[width=\textwidth]{Images/cover.png}
\caption*{Demonstrating My Saiya Style (2012)}
\end{minipage}
\begin{minipage}[t]{0.25\textwidth}\vspace{0pt}
\begin{itemize}[nosep,leftmargin=1em,labelwidth=*,align=left]
	\setlength{\itemsep}{0pt}
	\item Demonstrating My Saiya Style
	\item Home Is For The Heartless
\end{itemize}
\end{minipage}

\subsection{Parkway Drive}

%=============
\begin{minipage}[t]{0.25\textwidth}\vspace{0pt}
	\captionsetup{type=figure}
	\includegraphics[width=\textwidth]{Images/cover.png}
	\caption*{Deep Blue (2010)}
\end{minipage}
\begin{minipage}[t]{0.25\textwidth}\vspace{0pt}
	\begin{itemize}[nosep,leftmargin=1em,labelwidth=*,align=left]
		\setlength{\itemsep}{0pt}
		\item Wreckage
		\item Home Is For The Heartless
		\item Alone
	\end{itemize}
\end{minipage}
%=============
\begin{minipage}[t]{0.25\textwidth}\vspace{0pt}
\captionsetup{type=figure}
\includegraphics[width=\textwidth]{Images/cover.png}
\caption*{Atlas (2012)}
\end{minipage}
\begin{minipage}[t]{0.25\textwidth}\vspace{0pt}
\begin{itemize}[nosep,leftmargin=1em,labelwidth=*,align=left]
	\setlength{\itemsep}{0pt}
	\item Wild Eyes
\end{itemize}
\end{minipage}


\subsection{The Browning}

%=============
\begin{minipage}[t]{0.25\textwidth}
\captionsetup{type=figure}
\includegraphics[width=\textwidth]{Images/cover.png}
\caption*{Isolation (2016)}
\end{minipage}
\begin{minipage}[t]{0.25\textwidth}\vspace{0pt}
\begin{itemize}[nosep,leftmargin=1em,labelwidth=*,align=left]
	\setlength{\itemsep}{0pt}
	\item Disconnect
	\item Pure Evil
	\item Isolation
\end{itemize}
\end{minipage}

\subsection{Bring Me The Horizon}

%=============
\begin{minipage}[t]{0.25\textwidth}
\captionsetup{type=figure}
\includegraphics[width=\textwidth]{Images/cover.png}
\caption*{Sempiternal (2013)}
\end{minipage}
\begin{minipage}[t]{0.25\textwidth}\vspace{0pt}
\begin{itemize}[nosep,leftmargin=1em,labelwidth=*,align=left]
	\setlength{\itemsep}{0pt}
	\item Sempiternal
	\item Shadow Moses
\end{itemize}
\end{minipage}
%=============
\begin{minipage}[t]{0.25\textwidth}
\captionsetup{type=figure}
\includegraphics[width=\textwidth]{Images/cover.png}
\caption*{Thats The Spirit (2015)}
\end{minipage}
\begin{minipage}[t]{0.25\textwidth}\vspace{0pt}
\begin{itemize}[nosep,leftmargin=1em,labelwidth=*,align=left]
	\setlength{\itemsep}{0pt}
	\item Happy Song
	\item Doomed
\end{itemize}
\end{minipage}

%===========================================
% Folk Metal
%===========================================

\section{Folk Metal}

\subsection{Eluveitie}

%=============
\begin{minipage}[t]{0.25\textwidth}
\captionsetup{type=figure}
\includegraphics[width=\textwidth]{Images/cover.png}
\caption*{Slania \& Evocation - The Arcane Metal Hammer Edition (2009)}
\end{minipage}
\begin{minipage}[t]{0.25\textwidth}\vspace{0pt}
\begin{itemize}[nosep,leftmargin=1em,labelwidth=*,align=left]
	\setlength{\itemsep}{0pt}
	\item Gray Sublime Archon
	\item Inis Mona
	\item The Arcane Dominion
	\item Omnos
	\item Slania (Folk Medley)
\end{itemize}
\end{minipage}
%=============
\begin{minipage}[t]{0.25\textwidth}
\captionsetup{type=figure}
\includegraphics[width=\textwidth]{Images/cover.png}
\caption*{Everything Remains (As It Never Was) (2010)}
\end{minipage}
\begin{minipage}[t]{0.25\textwidth}\vspace{0pt}
\begin{itemize}[nosep,leftmargin=1em,labelwidth=*,align=left]
	\setlength{\itemsep}{0pt}
	\item Thousandfold
	\item Everything Remains As It Never Was
	\item Nil
	\item Kingdom Come Undone
\end{itemize}
\end{minipage}
%=============
\begin{minipage}[t]{0.25\textwidth}
\captionsetup{type=figure}
\includegraphics[width=\textwidth]{Images/cover.png}
\caption*{Origins (2014)}
\end{minipage}
\begin{minipage}[t]{0.25\textwidth}\vspace{0pt}
\begin{itemize}[nosep,leftmargin=1em,labelwidth=*,align=left]
	\setlength{\itemsep}{0pt}
	\item The Silver Sister
	\item Carry The Torch
	\item The Nameless
	\item From Darkness
	\item The Call Of The Mountains
	\item King
\end{itemize}
\end{minipage}
%=============
\begin{minipage}[t]{0.25\textwidth}
\captionsetup{type=figure}
\includegraphics[width=\textwidth]{Images/cover.png}
\caption*{Evocation II Pantheon (2017)}
\end{minipage}
\begin{minipage}[t]{0.25\textwidth}\vspace{0pt}
\begin{itemize}[nosep,leftmargin=1em,labelwidth=*,align=left]
	\setlength{\itemsep}{0pt}
	\item Catvrix
	\item Epona
	\item Lvgvs
	\item Antvmnos
	\item Artio
\end{itemize}
\end{minipage}

\subsection{Finntroll}

%=============
\begin{minipage}[t]{0.25\textwidth}\vspace{0pt}
\captionsetup{type=figure}
\includegraphics[width=\textwidth]{Images/cover.png}
\caption*{Nifelvind (2010)}
\end{minipage}
\begin{minipage}[t]{0.25\textwidth}\vspace{0pt}
\begin{itemize}[nosep,leftmargin=1em,labelwidth=*,align=left]
	\setlength{\itemsep}{0pt}
	\item Solsagan
	\item Under Bergets Rot
	\item Ett Norrskensdad
\end{itemize}
\end{minipage}

\subsection{Finsterforst}

%=============
\begin{minipage}[t]{0.25\textwidth}\vspace{0pt}
\captionsetup{type=figure}
\includegraphics[width=\textwidth]{Images/cover.png}
\caption*{Zerfall (2019)}
\end{minipage}
\begin{minipage}[t]{0.25\textwidth}\vspace{0pt}
\begin{itemize}[nosep,leftmargin=1em,labelwidth=*,align=left]
	\setlength{\itemsep}{0pt}
	\item Zerfall
	\item Weltenbrand
	\item Ecce Homo
\end{itemize}
\end{minipage}

\subsection{Korpiklaani}

%=============
\begin{minipage}[t]{0.25\textwidth}
\captionsetup{type=figure}
\includegraphics[width=\textwidth]{Images/cover.png}
\caption*{Voice Of Wilderness (2005)}
\end{minipage}
\begin{minipage}[t]{0.25\textwidth}\vspace{0pt}
\begin{itemize}[nosep,leftmargin=1em,labelwidth=*,align=left]
	\setlength{\itemsep}{0pt}
	\item Spirit Of The Forest
	\item Journey Man
	\item Beer Beer
\end{itemize}
\end{minipage}
%=============
\begin{minipage}[t]{0.25\textwidth}
\captionsetup{type=figure}
\includegraphics[width=\textwidth]{Images/cover.png}
\caption*{Noita (2015)}
\end{minipage}
\begin{minipage}[t]{0.25\textwidth}\vspace{0pt}
\begin{itemize}[nosep,leftmargin=1em,labelwidth=*,align=left]
	\setlength{\itemsep}{0pt}
	\item Ämmänhauta
	\item Lempo
	\item Pilli On Pajusta Tehty
	\item Sahti
\end{itemize}
\end{minipage}

\subsection{Skalmöld}

%=============
\begin{minipage}[t]{0.25\textwidth}
\captionsetup{type=figure}
\includegraphics[width=\textwidth]{Images/cover.png}
\caption*{Skalmöld \& Sinfoniuhljomsveit Islands (2013)}
\end{minipage}
\begin{minipage}[t]{0.25\textwidth}\vspace{0pt}
\begin{itemize}[nosep,leftmargin=1em,labelwidth=*,align=left]
	\setlength{\itemsep}{0pt}
	\item Hel
	\item Aras
	\item Midgardsormur
	\item Kvadning
\end{itemize}
\end{minipage}
%=============
\begin{minipage}[t]{0.25\textwidth}
\captionsetup{type=figure}
\includegraphics[width=\textwidth]{Images/cover.png}
\caption*{Vögguvisur Yggdrasils (2016)}
\end{minipage}
\begin{minipage}[t]{0.25\textwidth}\vspace{0pt}
\begin{itemize}[nosep,leftmargin=1em,labelwidth=*,align=left]
	\setlength{\itemsep}{0pt}
	\item Vanaheimur
	\item Alfheimur
\end{itemize}
\end{minipage}

\subsection{Equilibrium}

%=============
\begin{minipage}[t]{0.25\textwidth}
\captionsetup{type=figure}
\includegraphics[width=\textwidth]{Images/cover.png}
\caption*{Turis Fratyr (2005)}
\end{minipage}
\begin{minipage}[t]{0.25\textwidth}\vspace{0pt}
\begin{itemize}[nosep,leftmargin=1em,labelwidth=*,align=left]
	\setlength{\itemsep}{0pt}
	\item Der Sturm 
	\item Turis Fratyr
	\item Met
	\item Heimdalls Ruf
	\item Die Prophezeihung
	\item Nordheim
\end{itemize}
\end{minipage}
%=============
\begin{minipage}[t]{0.25\textwidth}
\captionsetup{type=figure}
\includegraphics[width=\textwidth]{Images/cover.png}
\caption*{Armageddon (2016)}
\end{minipage}
\begin{minipage}[t]{0.25\textwidth}\vspace{0pt}
\begin{itemize}[nosep,leftmargin=1em,labelwidth=*,align=left]
	\setlength{\itemsep}{0pt}
	\item Eternal Destination
	\item Prey
	\item Born To Be Epic
\end{itemize}
\end{minipage}

\subsection{Cellar Darling}

%=============
\begin{minipage}[t]{0.25\textwidth}\vspace{0pt}
\captionsetup{type=figure}
\includegraphics[width=\textwidth]{Images/cover.png}
\caption*{This Is The Sound (2017)}
\end{minipage}
\begin{minipage}[t]{0.25\textwidth}\vspace{0pt}
\begin{itemize}[nosep,leftmargin=1em,labelwidth=*,align=left]
	\setlength{\itemsep}{0pt}
	\item Avalanche
\end{itemize}
\end{minipage}

\subsection{Saltatio Mortis}

%=============
\begin{minipage}[t]{0.25\textwidth}\vspace{0pt}
\captionsetup{type=figure}
\includegraphics[width=\textwidth]{Images/cover.png}
\caption*{Wachstum Über Alles (2013)}
\end{minipage}
\begin{minipage}[t]{0.25\textwidth}\vspace{0pt}
\begin{itemize}[nosep,leftmargin=1em,labelwidth=*,align=left]
	\setlength{\itemsep}{0pt}
	\item Wachstum Über Alles
\end{itemize}
\end{minipage}

\subsection{Solstafir}

%=============
\begin{minipage}[t]{0.25\textwidth}
\captionsetup{type=figure}
\includegraphics[width=\textwidth]{Images/cover.png}
\caption*{Svartir Sandar (2011)}
\end{minipage}
\begin{minipage}[t]{0.25\textwidth}\vspace{0pt}
\begin{itemize}[nosep,leftmargin=1em,labelwidth=*,align=left]
	\setlength{\itemsep}{0pt}
	\item Fjara
\end{itemize}
\end{minipage}
%=============
\begin{minipage}[t]{0.25\textwidth}
\captionsetup{type=figure}
\includegraphics[width=\textwidth]{Images/cover.png}
\caption*{Otta (2014)}
\end{minipage}
\begin{minipage}[t]{0.25\textwidth}\vspace{0pt}
\begin{itemize}[nosep,leftmargin=1em,labelwidth=*,align=left]
	\setlength{\itemsep}{0pt}
	\item Lagnaetti
	\item Otta
\end{itemize}
\end{minipage}

\subsection{In Extremo}

%=============
\begin{minipage}[t]{0.25\textwidth}
\captionsetup{type=figure}
\includegraphics[width=\textwidth]{Images/cover.png}
\caption*{Quid Pro Quo (2016)}
\end{minipage}
\begin{minipage}[t]{0.25\textwidth}\vspace{0pt}
\begin{itemize}[nosep,leftmargin=1em,labelwidth=*,align=left]
	\setlength{\itemsep}{0pt}
	\item Lieb Vaterland, Magst Ruhig Sein
	\item Sternhagelvoll
	\item Pikse Palve
	\item Roter Stern
	\item Flaschenteufel
\end{itemize}
\end{minipage}

\subsection{Tyr}

%=============
\begin{minipage}[t]{0.25\textwidth}
\captionsetup{type=figure}
\includegraphics[width=\textwidth]{Images/cover.png}
\caption*{Valkyrja (2013)}
\end{minipage}
\begin{minipage}[t]{0.25\textwidth}\vspace{0pt}
\begin{itemize}[nosep,leftmargin=1em,labelwidth=*,align=left]
	\setlength{\itemsep}{0pt}
	\item Blood Of Heroes
	\item The Lay Of Our Love
\end{itemize}
\end{minipage}

\subsection{Arkona}

%=============
\begin{minipage}[t]{0.25\textwidth}
\captionsetup{type=figure}
\includegraphics[width=\textwidth]{Images/cover.png}
\caption*{Decade Of Glory (2013)}
\end{minipage}
\begin{minipage}[t]{0.25\textwidth}\vspace{0pt}
\begin{itemize}[nosep,leftmargin=1em,labelwidth=*,align=left]
	\setlength{\itemsep}{0pt}
	\item Yarilo
	\item Goi, Rode, Goi
	\item Slavsia, Rus
	\item Rus
	\item Liki Bessmertnykh Bogov
	\item Stenka Na Stenku
\end{itemize}
\end{minipage}
%=============
\begin{minipage}[t]{0.25\textwidth}
\captionsetup{type=figure}
\includegraphics[width=\textwidth]{Images/cover.png}
\caption*{Yav (2014)}
\end{minipage}
\begin{minipage}[t]{0.25\textwidth}\vspace{0pt}
\begin{itemize}[nosep,leftmargin=1em,labelwidth=*,align=left]
	\setlength{\itemsep}{0pt}
	\item Yav'
\end{itemize}
\end{minipage}

\subsection{Ensiferum}

%=============
\begin{minipage}[t]{0.25\textwidth}
\captionsetup{type=figure}
\includegraphics[width=\textwidth]{Images/cover.png}
\caption*{One Man Army (2015)}
\end{minipage}
\begin{minipage}[t]{0.25\textwidth}\vspace{0pt}
\begin{itemize}[nosep,leftmargin=1em,labelwidth=*,align=left]
	\setlength{\itemsep}{0pt}
	\item Two Of Spades
	\item Neito Pohjolan
	\item Axe Of Judgement
	\item Cry For The Earth Bounds
\end{itemize}
\end{minipage}

\subsection{Fejd}

%=============
\begin{minipage}[t]{0.25\textwidth}
\captionsetup{type=figure}
\includegraphics[width=\textwidth]{Images/cover.png}
\caption*{Eifur (2010)}
\end{minipage}
\begin{minipage}[t]{0.25\textwidth}\vspace{0pt}
\begin{itemize}[nosep,leftmargin=1em,labelwidth=*,align=left]
	\setlength{\itemsep}{0pt}
	\item Gryning
	\item Eifur
	\item Yggdrasil
\end{itemize}
\end{minipage}
%=============
\begin{minipage}[t]{0.25\textwidth}
\captionsetup{type=figure}
\includegraphics[width=\textwidth]{Images/cover.png}
\caption*{Trolldom (2016)}
\end{minipage}
\begin{minipage}[t]{0.25\textwidth}\vspace{0pt}
\begin{itemize}[nosep,leftmargin=1em,labelwidth=*,align=left]
	\setlength{\itemsep}{0pt}
	\item Bed För Din Själ
	\item Härjaren
\end{itemize}
\end{minipage}

\subsection{Heidevolk}

%=============
\begin{minipage}[t]{0.25\textwidth}
\captionsetup{type=figure}
\includegraphics[width=\textwidth]{Images/cover.png}
\caption*{Velua (2015)}
\end{minipage}
\begin{minipage}[t]{0.25\textwidth}\vspace{0pt}
\begin{itemize}[nosep,leftmargin=1em,labelwidth=*,align=left]
	\setlength{\itemsep}{0pt}
	\item Winter Woede
	\item Urth
\end{itemize}
\end{minipage}

\subsection{Alestorm}

%=============
\begin{minipage}[t]{0.25\textwidth}
\captionsetup{type=figure}
\includegraphics[width=\textwidth]{Images/cover.png}
\caption*{Live At The End Of The World (2013)}
\end{minipage}
\begin{minipage}[t]{0.25\textwidth}\vspace{0pt}
\begin{itemize}[nosep,leftmargin=1em,labelwidth=*,align=left]
	\setlength{\itemsep}{0pt}
	\item Shipwrecked
\end{itemize}
\end{minipage}

\subsection{Cellar Darling}

%=============
\begin{minipage}[t]{0.25\textwidth}
\captionsetup{type=figure}
\includegraphics[width=\textwidth]{Images/cover.png}
\caption*{Challenge (Single 2016)}
\end{minipage}
\begin{minipage}[t]{0.25\textwidth}\vspace{0pt}
\begin{itemize}[nosep,leftmargin=1em,labelwidth=*,align=left]
	\setlength{\itemsep}{0pt}
	\item Challenge Me
	\item Fire, Wind \& Earth
\end{itemize}
\end{minipage}

\subsection{Grai}

%=============
\begin{minipage}[t]{0.25\textwidth}
\captionsetup{type=figure}
\includegraphics[width=\textwidth]{Images/cover.png}
\caption*{In The Arms Of Mara (2014)}
\end{minipage}
\begin{minipage}[t]{0.25\textwidth}\vspace{0pt}
\begin{itemize}[nosep,leftmargin=1em,labelwidth=*,align=left]
	\setlength{\itemsep}{0pt}
	\item In The Arms Of Mara
\end{itemize}
\end{minipage}

\subsection{Helengard}

%=============
\begin{minipage}[t]{0.25\textwidth}
\captionsetup{type=figure}
\includegraphics[width=\textwidth]{Images/cover.png}
\caption*{Firebird (2017)}
\end{minipage}
\begin{minipage}[t]{0.25\textwidth}\vspace{0pt}
\begin{itemize}[nosep,leftmargin=1em,labelwidth=*,align=left]
	\setlength{\itemsep}{0pt}
	\item Fall Rue
	\item Summer Feast
\end{itemize}
\end{minipage}

%===========================================
% Doom
%===========================================

\section{Doom, Gothic, Stoner, etc.}

\subsection{Paradise Lost}

%=============
\begin{minipage}[t]{0.25\textwidth}
\captionsetup{type=figure}
\includegraphics[width=\textwidth]{Images/cover.png}
\caption*{Draconian Times (1995)}
\end{minipage}
\begin{minipage}[t]{0.25\textwidth}\vspace{0pt}
\begin{itemize}[nosep,leftmargin=1em,labelwidth=*,align=left]
	\setlength{\itemsep}{0pt}
	\item Enchantment
	\item Hallowed Land
\end{itemize}
\end{minipage}
%=============
\begin{minipage}[t]{0.25\textwidth}
\captionsetup{type=figure}
\includegraphics[width=\textwidth]{Images/cover.png}
\caption*{Faith Divides Us - Death Unites Us (2009)}
\end{minipage}
\begin{minipage}[t]{0.25\textwidth}\vspace{0pt}
\begin{itemize}[nosep,leftmargin=1em,labelwidth=*,align=left]
	\setlength{\itemsep}{0pt}
	\item Last Regret
	\item Faith Divides Us - Death Unites Us
\end{itemize}
\end{minipage}
%=============
\begin{minipage}[t]{0.25\textwidth}
\captionsetup{type=figure}
\includegraphics[width=\textwidth]{Images/cover.png}
\caption*{Tragic Illusion 25 (The Rarities) (2013)}
\end{minipage}
\begin{minipage}[t]{0.25\textwidth}\vspace{0pt}
\begin{itemize}[nosep,leftmargin=1em,labelwidth=*,align=left]
	\setlength{\itemsep}{0pt}
	\item Last Regret
	\item Faith Divides Us - Death Unites Us
\end{itemize}
\end{minipage}
%=============
\begin{minipage}[t]{0.25\textwidth}
\captionsetup{type=figure}
\includegraphics[width=\textwidth]{Images/cover.png}
\caption*{The Plague Within (2015)}
\end{minipage}
\begin{minipage}[t]{0.25\textwidth}\vspace{0pt}
\begin{itemize}[nosep,leftmargin=1em,labelwidth=*,align=left]
	\setlength{\itemsep}{0pt}
	\item Beneath Broken Earth
	\item No Hope In Sight
	\item Victim Of The Past
	\item Sacrifice The Flame
\end{itemize}
\end{minipage}

\subsection{Opeth}

%=============
\begin{minipage}[t]{0.25\textwidth}
\captionsetup{type=figure}
\includegraphics[width=\textwidth]{Images/cover.png}
\caption*{Blackwater Park (2001)}
\end{minipage}
\begin{minipage}[t]{0.25\textwidth}\vspace{0pt}
\begin{itemize}[nosep,leftmargin=1em,labelwidth=*,align=left]
	\setlength{\itemsep}{0pt}
	\item The Leaper Affinity
	\item Dirge For November
	\item The Drapery Falls
\end{itemize}
\end{minipage}
%=============
\begin{minipage}[t]{0.25\textwidth}
\captionsetup{type=figure}
\includegraphics[width=\textwidth]{Images/cover.png}
\caption*{The Roundhouse Tapes (2008)}
\end{minipage}
\begin{minipage}[t]{0.25\textwidth}\vspace{0pt}
\begin{itemize}[nosep,leftmargin=1em,labelwidth=*,align=left]
	\setlength{\itemsep}{0pt}
	\item Bleak
	\item Blackwater Park
	\item Night And The Silent Water
\end{itemize}
\end{minipage}

\subsection{Katatonia}

%=============
\begin{minipage}[t]{0.25\textwidth}
\captionsetup{type=figure}
\includegraphics[width=\textwidth]{Images/cover.png}
\caption*{Sanctitude (2015)}
\end{minipage}
\begin{minipage}[t]{0.25\textwidth}\vspace{0pt}
\begin{itemize}[nosep,leftmargin=1em,labelwidth=*,align=left]
	\setlength{\itemsep}{0pt}
	\item Teargas
	\item A Darkness Coming
	\item Idle Blood
	\item Undo You
\end{itemize}
\end{minipage}

\subsection{Alunah}

%=============
\begin{minipage}[t]{0.25\textwidth}\vspace{0pt}
\captionsetup{type=figure}
\includegraphics[width=\textwidth]{Images/cover.png}
\caption*{Awakening The Forest (2014)}
\end{minipage}
\begin{minipage}[t]{0.25\textwidth}\vspace{0pt}
\begin{itemize}[nosep,leftmargin=1em,labelwidth=*,align=left]
	\setlength{\itemsep}{0pt}
	\item Light Of Winter
\end{itemize}
\end{minipage}

\subsection{Pet The Preacher}

%=============
\begin{minipage}[t]{0.25\textwidth}\vspace{0pt}
\captionsetup{type=figure}
\includegraphics[width=\textwidth]{Images/cover.png}
\caption*{The Cave And The Sunlight (2014)}
\end{minipage}
\begin{minipage}[t]{0.25\textwidth}\vspace{0pt}
\begin{itemize}[nosep,leftmargin=1em,labelwidth=*,align=left]
	\setlength{\itemsep}{0pt}
	\item Let Your Dragon Fly
\end{itemize}
\end{minipage}

\subsection{Lacrimas Profundere}

%=============
\begin{minipage}[t]{0.25\textwidth}\vspace{0pt}
\captionsetup{type=figure}
\includegraphics[width=\textwidth]{Images/cover.png}
\caption*{Antiadore (2013)}
\end{minipage}
\begin{minipage}[t]{0.25\textwidth}\vspace{0pt}
\begin{itemize}[nosep,leftmargin=1em,labelwidth=*,align=left]
	\setlength{\itemsep}{0pt}
	\item Antiadore
\end{itemize}
\end{minipage}

%===========================================
% Black Metal
%===========================================

\section{Black Metal}

\subsection{Vindland}

%=============
\begin{minipage}[t]{0.25\textwidth}
\captionsetup{type=figure}
\includegraphics[width=\textwidth]{Images/cover.png}
\caption*{Hanter Savet (2016)}
\end{minipage}
\begin{minipage}[t]{0.25\textwidth}\vspace{0pt}
\begin{itemize}[nosep,leftmargin=1em,labelwidth=*,align=left]
	\setlength{\itemsep}{0pt}
	\item Morlusenn 
\end{itemize}
\end{minipage}

\subsection{Uada}

%=============
\begin{minipage}[t]{0.25\textwidth}
\captionsetup{type=figure}
\includegraphics[width=\textwidth]{Images/cover.png}
\caption*{Devoid Of Light (2016)}
\end{minipage}
\begin{minipage}[t]{0.25\textwidth}\vspace{0pt}
\begin{itemize}[nosep,leftmargin=1em,labelwidth=*,align=left]
	\setlength{\itemsep}{0pt}
	\item Devoid Of Light
	\item Black Autumn, White Spring
\end{itemize}
\end{minipage}

\subsection{Zeal And Ardor}

%=============
\begin{minipage}[t]{0.25\textwidth}
\captionsetup{type=figure}
\includegraphics[width=\textwidth]{Images/cover.png}
\caption*{Stranger Fruit (2018)}
\end{minipage}
\begin{minipage}[t]{0.25\textwidth}\vspace{0pt}
\begin{itemize}[nosep,leftmargin=1em,labelwidth=*,align=left]
	\setlength{\itemsep}{0pt}
	\item Built On Ashes
	\item Don't You Dare
	\item Row Row
\end{itemize}
\end{minipage}

\subsection{Der Weg Einer Freiheit}

%=============
\begin{minipage}[t]{0.25\textwidth}
\captionsetup{type=figure}
\includegraphics[width=\textwidth]{Images/cover.png}
\caption*{Finisterre (2017)}
\end{minipage}
\begin{minipage}[t]{0.25\textwidth}\vspace{0pt}
\begin{itemize}[nosep,leftmargin=1em,labelwidth=*,align=left]
	\setlength{\itemsep}{0pt}
	\item Aufbruch
\end{itemize}
\end{minipage}

\subsection{Ultar}

%=============
\begin{minipage}[t]{0.25\textwidth}
\captionsetup{type=figure}
\includegraphics[width=\textwidth]{Images/cover.png}
\caption*{Kadath (2016)}
\end{minipage}
\begin{minipage}[t]{0.25\textwidth}\vspace{0pt}
\begin{itemize}[nosep,leftmargin=1em,labelwidth=*,align=left]
	\setlength{\itemsep}{0pt}
	\item Nyarlathotep 
	\item Azathoth 
\end{itemize}
\end{minipage}

\subsection{Realm Of Wolves}

%=============
\begin{minipage}[t]{0.25\textwidth}
\captionsetup{type=figure}
\includegraphics[width=\textwidth]{Images/cover.png}
\caption*{Oblivion (2018)}
\end{minipage}
\begin{minipage}[t]{0.25\textwidth}\vspace{0pt}
\begin{itemize}[nosep,leftmargin=1em,labelwidth=*,align=left]
	\setlength{\itemsep}{0pt}
	\item Ignifer
\end{itemize}
\end{minipage}


\subsection{Harakiri For The Sky}

%=============
\begin{minipage}[t]{0.25\textwidth}
\captionsetup{type=figure}
\includegraphics[width=\textwidth]{Images/cover.png}
\caption*{III: Trauma (2016)}
\end{minipage}
\begin{minipage}[t]{0.25\textwidth}\vspace{0pt}
\begin{itemize}[nosep,leftmargin=1em,labelwidth=*,align=left]
	\setlength{\itemsep}{0pt}
	\item The Traces We Leave
\end{itemize}
\end{minipage}

\subsection{Ultha}

%=============
\begin{minipage}[t]{0.25\textwidth}
\captionsetup{type=figure}
\includegraphics[width=\textwidth]{Images/cover.png}
\caption*{The Inextricable Wandering (2018)}
\end{minipage}
\begin{minipage}[t]{0.25\textwidth}\vspace{0pt}
\begin{itemize}[nosep,leftmargin=1em,labelwidth=*,align=left]
	\setlength{\itemsep}{0pt}
	\item Cyanide Lips
\end{itemize}
\end{minipage}

\subsection{Mgła}

%=============
\begin{minipage}[t]{0.25\textwidth}
\captionsetup{type=figure}
\includegraphics[width=\textwidth]{Images/cover.png}
\caption*{Exercises in futility (2015)}
\end{minipage}
\begin{minipage}[t]{0.25\textwidth}\vspace{0pt}
\begin{itemize}[nosep,leftmargin=1em,labelwidth=*,align=left]
	\setlength{\itemsep}{0pt}
	\item Exercises in futility IV
\end{itemize}
\end{minipage}

\subsection{The Spirit}

%=============
\begin{minipage}[t]{0.25\textwidth}
\captionsetup{type=figure}
\includegraphics[width=\textwidth]{Images/cover.png}
\caption*{Sounds From The Vortex (2018)}
\end{minipage}
\begin{minipage}[t]{0.25\textwidth}\vspace{0pt}
\begin{itemize}[nosep,leftmargin=1em,labelwidth=*,align=left]
	\setlength{\itemsep}{0pt}
	\item The Clouds Of Damnation
\end{itemize}
\end{minipage}

\subsection{Carach Angren}

%=============
\begin{minipage}[t]{0.25\textwidth}
\captionsetup{type=figure}
\includegraphics[width=\textwidth]{Images/cover.png}
\caption*{This Is No Fairytale (2015)}
\end{minipage}
\begin{minipage}[t]{0.25\textwidth}\vspace{0pt}
\begin{itemize}[nosep,leftmargin=1em,labelwidth=*,align=left]
	\setlength{\itemsep}{0pt}
	\item When Crows Tick On Windows
	\item There's No Place Like Home
\end{itemize}
\end{minipage}

\subsection{Borknagar}

%=============
\begin{minipage}[t]{0.25\textwidth}
\captionsetup{type=figure}
\includegraphics[width=\textwidth]{Images/cover.png}
\caption*{Winter Thrice (2016)}
\end{minipage}
\begin{minipage}[t]{0.25\textwidth}\vspace{0pt}
\begin{itemize}[nosep,leftmargin=1em,labelwidth=*,align=left]
	\setlength{\itemsep}{0pt}
	\item The Rhymes Of The Mountain
\end{itemize}
\end{minipage}

\subsection{Agrypnie}

%=============
\begin{minipage}[t]{0.25\textwidth}
\captionsetup{type=figure}
\includegraphics[width=\textwidth]{Images/cover.png}
\caption*{Aetas Cineris (2013)}
\end{minipage}
\begin{minipage}[t]{0.25\textwidth}\vspace{0pt}
\begin{itemize}[nosep,leftmargin=1em,labelwidth=*,align=left]
	\setlength{\itemsep}{0pt}
	\item Erwachen
	\item Trümmer \/ Aetas Cineris
	\item Asche
\end{itemize}
\end{minipage}

\subsection{Behemoth}

%=============
\begin{minipage}[t]{0.25\textwidth}\vspace{0pt}
\captionsetup{type=figure}
\includegraphics[width=\textwidth]{Images/cover.png}
\caption*{Evangelion (2009)}
\end{minipage}
\begin{minipage}[t]{0.25\textwidth}\vspace{0pt}
\begin{itemize}[nosep,leftmargin=1em,labelwidth=*,align=left]
	\setlength{\itemsep}{0pt}
	\item Ov Fire And The Void
\end{itemize}
\end{minipage}
%=============
\begin{minipage}[t]{0.25\textwidth}\vspace{0pt}
\captionsetup{type=figure}
\includegraphics[width=\textwidth]{Images/cover.png}
\caption*{I Loved You At Your Darkest (2018)}
\end{minipage}
\begin{minipage}[t]{0.25\textwidth}\vspace{0pt}
\begin{itemize}[nosep,leftmargin=1em,labelwidth=*,align=left]
	\setlength{\itemsep}{0pt}
	\item Wolves ov Siberia
\end{itemize}
\end{minipage}

\subsection{Belzebubs}

%=============
\begin{minipage}[t]{0.25\textwidth}\vspace{0pt}
\captionsetup{type=figure}
\includegraphics[width=\textwidth]{Images/cover.png}
\caption*{Blackened Call (2018)}
\end{minipage}
\begin{minipage}[t]{0.25\textwidth}\vspace{0pt}
\begin{itemize}[nosep,leftmargin=1em,labelwidth=*,align=left]
	\setlength{\itemsep}{0pt}
	\item Blackened Call
	\item	Maleficarum - The Veil of the Moon Queen, Pt. I
\end{itemize}
\end{minipage}

\subsection{Celtic Frost}

%=============
\begin{minipage}[t]{0.25\textwidth}\vspace{0pt}
\captionsetup{type=figure}
\includegraphics[width=\textwidth]{Images/cover.png}
\caption*{Morbid Tales (1984)}
\end{minipage}
\begin{minipage}[t]{0.25\textwidth}\vspace{0pt}
\begin{itemize}[nosep,leftmargin=1em,labelwidth=*,align=left]
	\setlength{\itemsep}{0pt}
	\item Into The Crypts Of Rays
\end{itemize}
\end{minipage}

\subsection{Cradle Of Filth}

%=============
\begin{minipage}[t]{0.25\textwidth}\vspace{0pt}
\captionsetup{type=figure}
\includegraphics[width=\textwidth]{Images/cover.png}
\caption*{Hammer Of The Witches (2015)}
\end{minipage}
\begin{minipage}[t]{0.25\textwidth}\vspace{0pt}
\begin{itemize}[nosep,leftmargin=1em,labelwidth=*,align=left]
	\setlength{\itemsep}{0pt}
	\item Blackest Magick In Practice
\end{itemize}
\end{minipage}

\subsection{Enslaved}

%=============
\begin{minipage}[t]{0.25\textwidth}\vspace{0pt}
\captionsetup{type=figure}
\includegraphics[width=\textwidth]{Images/cover.png}
\caption*{Vertebrae (2008)}
\end{minipage}
\begin{minipage}[t]{0.25\textwidth}\vspace{0pt}
\begin{itemize}[nosep,leftmargin=1em,labelwidth=*,align=left]
	\setlength{\itemsep}{0pt}
	\item The Watcher
\end{itemize}
\end{minipage}

\subsection{Hypothermia}

%=============
\begin{minipage}[t]{0.25\textwidth}
\captionsetup{type=figure}
\includegraphics[width=\textwidth]{Images/cover.png}
\caption*{Skogens Hjaerta (2010)}
\end{minipage}
\begin{minipage}[t]{0.25\textwidth}\vspace{0pt}
\begin{itemize}[nosep,leftmargin=1em,labelwidth=*,align=left]
	\setlength{\itemsep}{0pt}
	\item Skogens Hjärta
\end{itemize}
\end{minipage}

\subsection{In Tenebriz}

%=============
\begin{minipage}[t]{0.25\textwidth}
\captionsetup{type=figure}
\includegraphics[width=\textwidth]{Images/cover.png}
\caption*{As The Spring Uncovers Pain (2017)}
\end{minipage}
\begin{minipage}[t]{0.25\textwidth}\vspace{0pt}
\begin{itemize}[nosep,leftmargin=1em,labelwidth=*,align=left]
	\setlength{\itemsep}{0pt}
	\item Pale Forest
	\item As The Spring Uncover Pain
\end{itemize}
\end{minipage}

\subsection{Moonspell}

%=============
\begin{minipage}[t]{0.25\textwidth}\vspace{0pt}
\captionsetup{type=figure}
\includegraphics[width=\textwidth]{Images/cover.png}
\caption*{Extinct (2015)}
\end{minipage}
\begin{minipage}[t]{0.25\textwidth}\vspace{0pt}
\begin{itemize}[nosep,leftmargin=1em,labelwidth=*,align=left]
	\setlength{\itemsep}{0pt}
	\item Extinct
	\item Breathe (Until We Are No More)
\end{itemize}
\end{minipage}

\subsection{Primordial}

%=============
\begin{minipage}[t]{0.25\textwidth}
\captionsetup{type=figure}
\includegraphics[width=\textwidth]{Images/cover.png}
\caption*{The Gathering Wilderness (2005)}
\end{minipage}
\begin{minipage}[t]{0.25\textwidth}\vspace{0pt}
\begin{itemize}[nosep,leftmargin=1em,labelwidth=*,align=left]
	\setlength{\itemsep}{0pt}
	\item The Coffin Ships
\end{itemize}
\end{minipage}

\subsection{Wode}

%=============
\begin{minipage}[t]{0.25\textwidth}
\captionsetup{type=figure}
\includegraphics[width=\textwidth]{Images/cover.png}
\caption*{Wode (2017)}
\end{minipage}
\begin{minipage}[t]{0.25\textwidth}\vspace{0pt}
\begin{itemize}[nosep,leftmargin=1em,labelwidth=*,align=left]
	\setlength{\itemsep}{0pt}
	\item Trails Of Smoke
	\item Plagues Of Insomnia
\end{itemize}
\end{minipage}

\subsection{Baise Ma Hache}

%=============
\begin{minipage}[t]{0.25\textwidth}
\captionsetup{type=figure}
\includegraphics[width=\textwidth]{Images/cover.png}
\caption*{F.E.R.T (2018)}
\end{minipage}
\begin{minipage}[t]{0.25\textwidth}\vspace{0pt}
\begin{itemize}[nosep,leftmargin=1em,labelwidth=*,align=left]
	\setlength{\itemsep}{0pt}
	\item Délivrance
\end{itemize}
\end{minipage}


%===========================================
% Alternative Metal
%===========================================

\section{Alternative Metal}

\subsection{Kvelertak}

%=============
\begin{minipage}[t]{0.25\textwidth}
\captionsetup{type=figure}
\includegraphics[width=\textwidth]{Images/cover.png}
\caption*{Kvelertak (2010)}
\end{minipage}
\begin{minipage}[t]{0.25\textwidth}\vspace{0pt}
\begin{itemize}[nosep,leftmargin=1em,labelwidth=*,align=left]
	\setlength{\itemsep}{0pt}
	\item Fossegrim
	\item Mjod
	\item Blodtorst
\end{itemize}
\end{minipage}
%=============
\begin{minipage}[t]{0.25\textwidth}
\captionsetup{type=figure}
\includegraphics[width=\textwidth]{Images/cover.png}
\caption*{Meir (2013)}
\end{minipage}
\begin{minipage}[t]{0.25\textwidth}\vspace{0pt}
\begin{itemize}[nosep,leftmargin=1em,labelwidth=*,align=left]
	\setlength{\itemsep}{0pt}
	\item Apenbaring
	\item Kvelertak
	\item Spring Fra Livet
	\item Trepan
	\item Nekrokosmos
\end{itemize}
\end{minipage}
%=============
\begin{minipage}[t]{0.25\textwidth}\vspace{0pt}
\captionsetup{type=figure}
\includegraphics[width=\textwidth]{Images/cover.png}
\caption*{Nattesferd (2016)}
\end{minipage}
\begin{minipage}[t]{0.25\textwidth}\vspace{0pt}
\begin{itemize}[nosep,leftmargin=1em,labelwidth=*,align=left]
	\setlength{\itemsep}{0pt}
	\item Heksebrann
\end{itemize}
\end{minipage}

\subsection{System Of A Down}

%=============
\begin{minipage}[t]{0.25\textwidth}
\captionsetup{type=figure}
\includegraphics[width=\textwidth]{Images/cover.png}
\caption*{System Of A Down (1998)}
\end{minipage}
\begin{minipage}[t]{0.25\textwidth}\vspace{0pt}
\begin{itemize}[nosep,leftmargin=1em,labelwidth=*,align=left]
	\setlength{\itemsep}{0pt}
	\item Sugar
	\item Soil
\end{itemize}
\end{minipage}
%=============
\begin{minipage}[t]{0.25\textwidth}
\captionsetup{type=figure}
\includegraphics[width=\textwidth]{Images/cover.png}
\caption*{Toxicity (2001)}
\end{minipage}
\begin{minipage}[t]{0.25\textwidth}\vspace{0pt}
\begin{itemize}[nosep,leftmargin=1em,labelwidth=*,align=left]
	\setlength{\itemsep}{0pt}
	\item Toxicity
	\item Chop Suey
	\item Aerials
	\item Deer Dance
	\item Needles
	\item Prison Song
\end{itemize}
\end{minipage}
%=============
\begin{minipage}[t]{0.25\textwidth}
\captionsetup{type=figure}
\includegraphics[width=\textwidth]{Images/cover.png}
\caption*{Steal This Album (2002)}
\end{minipage}
\begin{minipage}[t]{0.25\textwidth}\vspace{0pt}
\begin{itemize}[nosep,leftmargin=1em,labelwidth=*,align=left]
	\setlength{\itemsep}{0pt}
	\item Innervision
	\item I-E-A-I-A-I-O
	\item Pictures
	\item Highway Song
\end{itemize}
\end{minipage}
%=============
\begin{minipage}[t]{0.25\textwidth}
\captionsetup{type=figure}
\includegraphics[width=\textwidth]{Images/cover.png}
\caption*{Mezmerize (2006)}
\end{minipage}
\begin{minipage}[t]{0.25\textwidth}\vspace{0pt}
\begin{itemize}[nosep,leftmargin=1em,labelwidth=*,align=left]
	\setlength{\itemsep}{0pt}
	\item B.Y.O.B.
	\item Lost In Hollywood
	\item Old School Hollywood
	\item Sad Statue
	\item Radio/ Video
	\item Violent Pornography
\end{itemize}
\end{minipage}
%=============
\begin{minipage}[t]{0.25\textwidth}
\captionsetup{type=figure}
\includegraphics[width=\textwidth]{Images/cover.png}
\caption*{Hypnotize (2006)}
\end{minipage}
\begin{minipage}[t]{0.25\textwidth}\vspace{0pt}
\begin{itemize}[nosep,leftmargin=1em,labelwidth=*,align=left]
	\setlength{\itemsep}{0pt}
	\item Soldier Side
	\item Hypnotize
	\item Lonely Day
	\item She's Like Heroin
\end{itemize}
\end{minipage}
%=============
\begin{minipage}[t]{0.25\textwidth}
\captionsetup{type=figure}
\includegraphics[width=\textwidth]{Images/cover.png}
\caption*{Storaged Melodies}
\end{minipage}
\begin{minipage}[t]{0.25\textwidth}\vspace{0pt}
\begin{itemize}[nosep,leftmargin=1em,labelwidth=*,align=left]
	\setlength{\itemsep}{0pt}
	\item Feel Good
	\item Starlit Eyes
\end{itemize}
\end{minipage}

\subsection{Nothing More}

%=============
\begin{minipage}[t]{0.25\textwidth}\vspace{0pt}
\captionsetup{type=figure}
\includegraphics[width=\textwidth]{Images/cover.png}
\caption*{Nothing More (2014)}
\end{minipage}
\begin{minipage}[t]{0.25\textwidth}\vspace{0pt}
\begin{itemize}[nosep,leftmargin=1em,labelwidth=*,align=left]
	\setlength{\itemsep}{0pt}
	\item This Is The Time (Ballast)
	\item Jenny	
\end{itemize}
\end{minipage}

\subsection{Tempel}

%=============
\begin{minipage}[t]{0.25\textwidth}\vspace{0pt}
\captionsetup{type=figure}
\includegraphics[width=\textwidth]{Images/cover.png}
\caption*{Tempel (2018)}
\end{minipage}
\begin{minipage}[t]{0.25\textwidth}\vspace{0pt}
\begin{itemize}[nosep,leftmargin=1em,labelwidth=*,align=left]
	\setlength{\itemsep}{0pt}
	\item Fortress
\end{itemize}
\end{minipage}

\subsection{Alter Bridge}

%=============
\begin{minipage}[t]{0.25\textwidth}
\captionsetup{type=figure}
\includegraphics[width=\textwidth]{Images/cover.png}
\caption*{Fortress (2013)}
\end{minipage}
\begin{minipage}[t]{0.25\textwidth}\vspace{0pt}
\begin{itemize}[nosep,leftmargin=1em,labelwidth=*,align=left]
	\setlength{\itemsep}{0pt}
	\item Addicted To Pain
	\item Bleed It Dry
	\item Calm The Fire
	\item Lover
\end{itemize}
\end{minipage}

\subsection{Avenged Sevenfold}

%=============
\begin{minipage}[t]{0.25\textwidth}
\captionsetup{type=figure}
\includegraphics[width=\textwidth]{Images/cover.png}
\caption*{Nightmare (2010)}
\end{minipage}
\begin{minipage}[t]{0.25\textwidth}\vspace{0pt}
\begin{itemize}[nosep,leftmargin=1em,labelwidth=*,align=left]
	\setlength{\itemsep}{0pt}
	\item Nightmare
\end{itemize}
\end{minipage}
%=============
\begin{minipage}[t]{0.25\textwidth}
\captionsetup{type=figure}
\includegraphics[width=\textwidth]{Images/cover.png}
\caption*{The Stage (2016)}
\end{minipage}
\begin{minipage}[t]{0.25\textwidth}\vspace{0pt}
\begin{itemize}[nosep,leftmargin=1em,labelwidth=*,align=left]
	\setlength{\itemsep}{0pt}
	\item The Stage
	\item Exist
	\item Roman Sky
	\item God Damn
	\item Paradigm
\end{itemize}
\end{minipage}

\subsection{Diablo Blvd}

%=============
\begin{minipage}[t]{0.25\textwidth}\vspace{0pt}
\captionsetup{type=figure}
\includegraphics[width=\textwidth]{Images/cover.png}
\caption*{Zero Hour (2017)}
\end{minipage}
\begin{minipage}[t]{0.25\textwidth}\vspace{0pt}
\begin{itemize}[nosep,leftmargin=1em,labelwidth=*,align=left]
	\setlength{\itemsep}{0pt}
	\item Sing From The Gallows
	\item Life Amounts To Nothing
\end{itemize}
\end{minipage}

\subsection{Disturbed}

%=============
\begin{minipage}[t]{0.25\textwidth}
\captionsetup{type=figure}
\includegraphics[width=\textwidth]{Images/cover.png}
\caption*{The Sickness (2000)}
\end{minipage}
\begin{minipage}[t]{0.25\textwidth}\vspace{0pt}
\begin{itemize}[nosep,leftmargin=1em,labelwidth=*,align=left]
	\setlength{\itemsep}{0pt}
	\item Down With The Sickness
\end{itemize}
\end{minipage}
%=============
\begin{minipage}[t]{0.25\textwidth}
\captionsetup{type=figure}
\includegraphics[width=\textwidth]{Images/cover.png}
\caption*{Immortalized (2015)}
\end{minipage}
\begin{minipage}[t]{0.25\textwidth}\vspace{0pt}
\begin{itemize}[nosep,leftmargin=1em,labelwidth=*,align=left]
	\setlength{\itemsep}{0pt}
	\item The Light
	\item The Sound Of Silence
\end{itemize}
\end{minipage}

\subsection{Five Finger Death Punch}

%=============
\begin{minipage}[t]{0.25\textwidth}
\captionsetup{type=figure}
\includegraphics[width=\textwidth]{Images/cover.png}
\caption*{The Wrong Side of Heaven and the Righteous Side of Hell, Volume 1 (2013)}
\end{minipage}
\begin{minipage}[t]{0.25\textwidth}\vspace{0pt}
\begin{itemize}[nosep,leftmargin=1em,labelwidth=*,align=left]
	\setlength{\itemsep}{0pt}
	\item Wrong Side Of Heaven
	\item Watch You Bleed
	\item Lift Me Up
	\item Aywhere But Here
\end{itemize}
\end{minipage}
%=============
\begin{minipage}[t]{0.25\textwidth}
\captionsetup{type=figure}
\includegraphics[width=\textwidth]{Images/cover.png}
\caption*{Got Your Six (2015)}
\end{minipage}
\begin{minipage}[t]{0.25\textwidth}\vspace{0pt}
\begin{itemize}[nosep,leftmargin=1em,labelwidth=*,align=left]
	\setlength{\itemsep}{0pt}
	\item Jekyll And Hyde
	\item Got Your Six
	\item My Nemesis
	\item Wash It All Away
	\item Diggin' My Own Grave
	\item Boots And Blood
\end{itemize}
\end{minipage}

\subsection{Lindemann}

%=============
\begin{minipage}[t]{0.25\textwidth}
\captionsetup{type=figure}
\includegraphics[width=\textwidth]{Images/cover.png}
\caption*{Skills In Pills (2015)}
\end{minipage}
\begin{minipage}[t]{0.25\textwidth}\vspace{0pt}
\begin{itemize}[nosep,leftmargin=1em,labelwidth=*,align=left]
	\setlength{\itemsep}{0pt}
	\item Fish On
	\item Yukon
\end{itemize}
\end{minipage}
%=============
\begin{minipage}[t]{0.25\textwidth}
\captionsetup{type=figure}
\includegraphics[width=\textwidth]{Images/cover.png}
\caption*{Steh Auf (EP)(2019)}
\end{minipage}
\begin{minipage}[t]{0.25\textwidth}\vspace{0pt}
\begin{itemize}[nosep,leftmargin=1em,labelwidth=*,align=left]
	\setlength{\itemsep}{0pt}
	\item Steh Auf
\end{itemize}
\end{minipage}

\subsection{Linkin Park}

%=============
\begin{minipage}[t]{0.25\textwidth}
\captionsetup{type=figure}
\includegraphics[width=\textwidth]{Images/cover.png}
\caption*{Hybrid Theory (2000)}
\end{minipage}
\begin{minipage}[t]{0.25\textwidth}\vspace{0pt}
\begin{itemize}[nosep,leftmargin=1em,labelwidth=*,align=left]
	\setlength{\itemsep}{0pt}
	\item In The End
	\item Pushing Me Away
	\item One Step Closer
	\item Papercut
	\item With You
\end{itemize}
\end{minipage}
%=============
\begin{minipage}[t]{0.25\textwidth}\vspace{0pt}
\captionsetup{type=figure}
\includegraphics[width=\textwidth]{Images/cover.png}
\caption*{Minutes To Midnight (Deluxe) (2007)}
\end{minipage}
\begin{minipage}[t]{0.25\textwidth}\vspace{0pt}
\begin{itemize}[nosep,leftmargin=1em,labelwidth=*,align=left]
	\setlength{\itemsep}{0pt}
	\item Bleed It Out
	\item No More Sorrow
	\item Leave Out All The Rest
\end{itemize}
\end{minipage}
%=============
\begin{minipage}[t]{0.25\textwidth}
\captionsetup{type=figure}
\includegraphics[width=\textwidth]{Images/cover.png}
\caption*{New Divide (Single 2009)}
\end{minipage}
\begin{minipage}[t]{0.25\textwidth}\vspace{0pt}
\begin{itemize}[nosep,leftmargin=1em,labelwidth=*,align=left]
	\setlength{\itemsep}{0pt}
	\item New Divide
\end{itemize}
\end{minipage}

\subsection{Mastodon}

%=============
\begin{minipage}[t]{0.25\textwidth}
\captionsetup{type=figure}
\includegraphics[width=\textwidth]{Images/cover.png}
\caption*{Blood Mountain (2006)}
\end{minipage}
\begin{minipage}[t]{0.25\textwidth}\vspace{0pt}
\begin{itemize}[nosep,leftmargin=1em,labelwidth=*,align=left]
	\setlength{\itemsep}{0pt}
	\item Sleeping Giant
	\item Circle Of Cysquatch
\end{itemize}
\end{minipage}
%=============
\begin{minipage}[t]{0.25\textwidth}
\captionsetup{type=figure}
\includegraphics[width=\textwidth]{Images/cover.png}
\caption*{Once More \'Round The Sun (2014)}
\end{minipage}
\begin{minipage}[t]{0.25\textwidth}\vspace{0pt}
\begin{itemize}[nosep,leftmargin=1em,labelwidth=*,align=left]
	\setlength{\itemsep}{0pt}
	\item High Road
	\item The Motherload
	\item Asleep In The Deep
\end{itemize}
\end{minipage}

\subsection{Pain}

%=============
\begin{minipage}[t]{0.25\textwidth}
\captionsetup{type=figure}
\includegraphics[width=\textwidth]{Images/cover.png}
\caption*{Cynic Paradise (2008)}
\end{minipage}
\begin{minipage}[t]{0.25\textwidth}\vspace{0pt}
\begin{itemize}[nosep,leftmargin=1em,labelwidth=*,align=left]
	\setlength{\itemsep}{0pt}
	\item Follow Me
	\item Don't Care
	\item Have A Drink On Me
\end{itemize}
\end{minipage}
%=============
\begin{minipage}[t]{0.25\textwidth}
\captionsetup{type=figure}
\includegraphics[width=\textwidth]{Images/cover.png}
\caption*{Coming Home (2 Disk) (2016)}
\end{minipage}
\begin{minipage}[t]{0.25\textwidth}\vspace{0pt}
\begin{itemize}[nosep,leftmargin=1em,labelwidth=*,align=left]
	\setlength{\itemsep}{0pt}
	\item Call Me
	\item Coming Home
	\item Natural Born Idiot
	\item Shut Your Mouth 
	\item Same Old Song
	\item Dirty Woman
	\item The Great Pretender
\end{itemize}
\end{minipage}

\subsection{Rammstein}

%=============
\begin{minipage}[t]{0.25\textwidth}
\captionsetup{type=figure}
\includegraphics[width=\textwidth]{Images/cover.png}
\caption*{Videos 1995-2012 (2012)}
\end{minipage}
\begin{minipage}[t]{0.25\textwidth}\vspace{0pt}
\begin{itemize}[nosep,leftmargin=1em,labelwidth=*,align=left]
	\setlength{\itemsep}{0pt}
	\item Seemann
	\item Du Hast
	\item Haifisch
	\item Keine Lust
	\item Rosenrot
	\item Sonne
	\item Ich Tu Dir Weh
\end{itemize}
\end{minipage}

\subsection{Raubtier}

%=============
\begin{minipage}[t]{0.25\textwidth}
\captionsetup{type=figure}
\includegraphics[width=\textwidth]{Images/cover.png}
\caption*{Skriet Fran Vildmarken (2010)}
\end{minipage}
\begin{minipage}[t]{0.25\textwidth}\vspace{0pt}
\begin{itemize}[nosep,leftmargin=1em,labelwidth=*,align=left]
	\setlength{\itemsep}{0pt}
	\item Skriet Fran Vildmarken
	\item En Hjältes Väg
	\item Himmelsfärds-kommando
	\item Lebensgefahr
	\item Hulkovius Rex
	\item Achtung Panzer
	\item Världsherravälde
\end{itemize}
\end{minipage}
%=============
\begin{minipage}[t]{0.25\textwidth}
\captionsetup{type=figure}
\includegraphics[width=\textwidth]{Images/cover.png}
\caption*{Baersaerkagang (2016)}
\end{minipage}
\begin{minipage}[t]{0.25\textwidth}\vspace{0pt}
\begin{itemize}[nosep,leftmargin=1em,labelwidth=*,align=left]
	\setlength{\itemsep}{0pt}
	\item Bärsärkagang
	\item Genom Allt 
	\item Hymn
\end{itemize}
\end{minipage}

\subsection{Rest, Repose}

%=============
\begin{minipage}[t]{0.25\textwidth}
\captionsetup{type=figure}
\includegraphics[width=\textwidth]{Images/cover.png}
\caption*{Sleep City (EP 2015)}
\end{minipage}
\begin{minipage}[t]{0.25\textwidth}\vspace{0pt}
\begin{itemize}[nosep,leftmargin=1em,labelwidth=*,align=left]
	\setlength{\itemsep}{0pt}
	\item Sleep City
\end{itemize}
\end{minipage}

\subsection{Scars On Broadway}

%=============
\begin{minipage}[t]{0.25\textwidth}
\captionsetup{type=figure}
\includegraphics[width=\textwidth]{Images/cover.png}
\caption*{Scars On Broadway (2008)}
\end{minipage}
\begin{minipage}[t]{0.25\textwidth}\vspace{0pt}
\begin{itemize}[nosep,leftmargin=1em,labelwidth=*,align=left]
	\setlength{\itemsep}{0pt}
	\item Insane
	\item Funny
\end{itemize}
\end{minipage}
%=============
\begin{minipage}[t]{0.25\textwidth}\vspace{0pt}
\captionsetup{type=figure}
\includegraphics[width=\textwidth]{Images/cover.png}
\caption*{Dictator (2018)}
\end{minipage}
\begin{minipage}[t]{0.25\textwidth}\vspace{0pt}
\begin{itemize}[nosep,leftmargin=1em,labelwidth=*,align=left]
	\setlength{\itemsep}{0pt}
	\item Lives
	\item Guns Are Loaded
\end{itemize}
\end{minipage}

\subsection{Serj Tankian}

%=============
\begin{minipage}[t]{0.25\textwidth}
\captionsetup{type=figure}
\includegraphics[width=\textwidth]{Images/cover.png}
\caption*{Elect the Dead (2007)}
\end{minipage}
\begin{minipage}[t]{0.25\textwidth}\vspace{0pt}
\begin{itemize}[nosep,leftmargin=1em,labelwidth=*,align=left]
	\setlength{\itemsep}{0pt}
	\item Lie Lie Lie
	\item Saving Us
	\item Sky Is Over
	\item Empty Walls
	\item Feed Us
	\item Praise The Lord And Pass The Ammunition
\end{itemize}
\end{minipage}
%=============
\begin{minipage}[t]{0.25\textwidth}
\captionsetup{type=figure}
\includegraphics[width=\textwidth]{Images/cover.png}
\caption*{Elect the Dead Symphony (2010)}
\end{minipage}
\begin{minipage}[t]{0.25\textwidth}\vspace{0pt}
\begin{itemize}[nosep,leftmargin=1em,labelwidth=*,align=left]
	\setlength{\itemsep}{0pt}
	\item Lie Lie Lie
	\item Saving Us
	\item Sky Is Over
	\item Empty Walls
	\item Feed Us
\end{itemize}
\end{minipage}
%=============
\begin{minipage}[t]{0.25\textwidth}
\captionsetup{type=figure}
\includegraphics[width=\textwidth]{Images/cover.png}
\caption*{Imperfect Harmonies (2010)}
\end{minipage}
\begin{minipage}[t]{0.25\textwidth}\vspace{0pt}
\begin{itemize}[nosep,leftmargin=1em,labelwidth=*,align=left]
	\setlength{\itemsep}{0pt}
	\item Left Of Center
\end{itemize}
\end{minipage}
%=============
\begin{minipage}[t]{0.25\textwidth}
\captionsetup{type=figure}
\includegraphics[width=\textwidth]{Images/cover.png}
\caption*{Harakiri (2012)}
\end{minipage}
\begin{minipage}[t]{0.25\textwidth}\vspace{0pt}
\begin{itemize}[nosep,leftmargin=1em,labelwidth=*,align=left]
	\setlength{\itemsep}{0pt}
	\item Harakiri
\end{itemize}
\end{minipage}
%=============
\begin{minipage}[t]{0.25\textwidth}
\captionsetup{type=figure}
\includegraphics[width=\textwidth]{Images/cover.png}
\caption*{Orca (2013)}
\end{minipage}
\begin{minipage}[t]{0.25\textwidth}\vspace{0pt}
\begin{itemize}[nosep,leftmargin=1em,labelwidth=*,align=left]
	\setlength{\itemsep}{0pt}
	\item Act III - Delphinus Capensis
\end{itemize}
\end{minipage}

\subsection{Stepfather Fred}

%=============
\begin{minipage}[t]{0.25\textwidth}
\captionsetup{type=figure}
\includegraphics[width=\textwidth]{Images/cover.png}
\caption*{Hello Larry Brown? (2014)}
\end{minipage}
\begin{minipage}[t]{0.25\textwidth}\vspace{0pt}
\begin{itemize}[nosep,leftmargin=1em,labelwidth=*,align=left]
	\setlength{\itemsep}{0pt}
	\item Caroline
	\item Hello
	\item Fuck
\end{itemize}
\end{minipage}
%=============
\begin{minipage}[t]{0.25\textwidth}
\captionsetup{type=figure}
\includegraphics[width=\textwidth]{Images/cover.png}
\caption*{Unplugged and Handmade (2015)}
\end{minipage}
\begin{minipage}[t]{0.25\textwidth}\vspace{0pt}
\begin{itemize}[nosep,leftmargin=1em,labelwidth=*,align=left]
	\setlength{\itemsep}{0pt}
	\item Caroline
\end{itemize}
\end{minipage}

\subsection{Volbeat}

%=============
\begin{minipage}[t]{0.25\textwidth}
\captionsetup{type=figure}
\includegraphics[width=\textwidth]{Images/cover.png}
\caption*{The Strength \/ The Sound \/ The Songs (2005)}
\end{minipage}
\begin{minipage}[t]{0.25\textwidth}\vspace{0pt}
\begin{itemize}[nosep,leftmargin=1em,labelwidth=*,align=left]
	\setlength{\itemsep}{0pt}
	\item Always. Wu
	\item I Only Wanna Be With You
	\item Caroline \#1
	\item Caroline Leaving
	\item Rebel Monster
\end{itemize}
\end{minipage}


%
%===========================================
% Thresh Metal, Oldschool & others
%===========================================

\section{Thrash Metal, Oldschool \& others}

\subsubsection{ACDC}

%=============
\begin{minipage}[t]{0.25\textwidth}
\captionsetup{type=figure}
\includegraphics[width=\textwidth]{Images/cover.png}
\caption*{Live At Donington (1992)}
\end{minipage}
\begin{minipage}[t]{0.25\textwidth}\vspace{0pt}
\begin{itemize}[nosep,leftmargin=1em,labelwidth=*,align=left]
	\setlength{\itemsep}{0pt}
	\item Thunderstruck
	\item Hells Bells
	\item T.N.T.
	\item Highway To Hell
\end{itemize}
\end{minipage}

\subsubsection{Deep Purple}

%=============
\begin{minipage}[t]{0.25\textwidth}
\captionsetup{type=figure}
\includegraphics[width=\textwidth]{Images/cover.png}
\caption*{Perfect Strangers (1984)}
\end{minipage}
\begin{minipage}[t]{0.25\textwidth}\vspace{0pt}
\begin{itemize}[nosep,leftmargin=1em,labelwidth=*,align=left]
	\setlength{\itemsep}{0pt}
	\item Perfect Strangers 
	\item A Gypsy's Kiss
	\item Knocking At Your Back Door
\end{itemize}
\end{minipage}

\subsubsection{Dire Straits}

%=============
\begin{minipage}[t]{0.25\textwidth}
\captionsetup{type=figure}
\includegraphics[width=\textwidth]{Images/cover.png}
\caption*{Sultans Of Swing (1998)}
\end{minipage}
\begin{minipage}[t]{0.25\textwidth}\vspace{0pt}
\begin{itemize}[nosep,leftmargin=1em,labelwidth=*,align=left]
	\setlength{\itemsep}{0pt}
	\item Sultans Of Swing
	\item Lady Writer
\end{itemize}
\end{minipage}

\subsubsection{Airbourne}

%=============
\begin{minipage}[t]{0.25\textwidth}
\captionsetup{type=figure}
\includegraphics[width=\textwidth]{Images/cover.png}
\caption*{Black Dog Barking (2013)}
\end{minipage}
\begin{minipage}[t]{0.25\textwidth}\vspace{0pt}
\begin{itemize}[nosep,leftmargin=1em,labelwidth=*,align=left]
	\setlength{\itemsep}{0pt}
	\item Live It Up
	\item Back In The Game
	\item Ready To Rock
\end{itemize}
\end{minipage}

\subsubsection{Ghost}

%=============
\begin{minipage}[t]{0.25\textwidth}\vspace{0pt}
\captionsetup{type=figure}
\includegraphics[width=\textwidth]{Images/cover.png}
\caption*{Meliora + Popestar EP (2016 \& 2017)}
\end{minipage}
\begin{minipage}[t]{0.25\textwidth}\vspace{0pt}
\begin{itemize}[nosep,leftmargin=1em,labelwidth=*,align=left]
	\setlength{\itemsep}{0pt}
	\item He Is
	\item Square Hammer
	\item From The Pinnacle To The Pit
\end{itemize}
\end{minipage}

\subsubsection{Halestorm}

%=============
\begin{minipage}[t]{0.25\textwidth}\vspace{0pt}
\captionsetup{type=figure}
\includegraphics[width=\textwidth]{Images/cover.png}
\caption*{Into The Wild Life (2015)}
\end{minipage}
\begin{minipage}[t]{0.25\textwidth}\vspace{0pt}
\begin{itemize}[nosep,leftmargin=1em,labelwidth=*,align=left]
	\setlength{\itemsep}{0pt}
	\item Amen
	\item I Am The Fire
\end{itemize}
\end{minipage}

\subsubsection{Iron Maiden}

%=============
\begin{minipage}[t]{0.25\textwidth}
\captionsetup{type=figure}
\includegraphics[width=\textwidth]{Images/cover.png}
\caption*{From Fear to Eternity (2011)}
\end{minipage}
\begin{minipage}[t]{0.25\textwidth}\vspace{0pt}
\begin{itemize}[nosep,leftmargin=1em,labelwidth=*,align=left]
	\setlength{\itemsep}{0pt}
	\item Fear Of The Dark
	\item Blood Brothers
	\item Paschendale
\end{itemize}
\end{minipage}
%=============
\begin{minipage}[t]{0.25\textwidth}
\captionsetup{type=figure}
\includegraphics[width=\textwidth]{Images/cover.png}
\caption*{The Book of Souls (2015)}
\end{minipage}
\begin{minipage}[t]{0.25\textwidth}\vspace{0pt}
\begin{itemize}[nosep,leftmargin=1em,labelwidth=*,align=left]
	\setlength{\itemsep}{0pt}
	\item Death Or Glory
	\item The Book Of Souls
\end{itemize}
\end{minipage}

\subsubsection{Judas Priest}

%=============
\begin{minipage}[t]{0.25\textwidth}\vspace{0pt}
	\captionsetup{type=figure}
	\includegraphics[width=\textwidth]{Images/cover.png}
	\caption*{Turbo (1986)}
\end{minipage}
\begin{minipage}[t]{0.25\textwidth}\vspace{0pt}
	\begin{itemize}[nosep,leftmargin=1em,labelwidth=*,align=left]
		\setlength{\itemsep}{0pt}
		\item Turbo Lover
	\end{itemize}
\end{minipage}
%=============
%=============
\begin{minipage}[t]{0.25\textwidth}\vspace{0pt}
\captionsetup{type=figure}
\includegraphics[width=\textwidth]{Images/cover.png}
\caption*{Painkiller (1990)}
\end{minipage}
\begin{minipage}[t]{0.25\textwidth}\vspace{0pt}
\begin{itemize}[nosep,leftmargin=1em,labelwidth=*,align=left]
	\setlength{\itemsep}{0pt}
	\item Painkiller
	\item Night Crawler
\end{itemize}
\end{minipage}
%=============
\begin{minipage}[t]{0.25\textwidth}\vspace{0pt}
\captionsetup{type=figure}
\includegraphics[width=\textwidth]{Images/cover.png}
\caption*{Single Cuts (2011)}
\end{minipage}
\begin{minipage}[t]{0.25\textwidth}\vspace{0pt}
\begin{itemize}[nosep,leftmargin=1em,labelwidth=*,align=left]
	\setlength{\itemsep}{0pt}
	\item Painkiller
	\item Before The Dawn
\end{itemize}
\end{minipage}
%=============
\begin{minipage}[t]{0.25\textwidth}\vspace{0pt}
\captionsetup{type=figure}
\includegraphics[width=\textwidth]{Images/cover.png}
\caption*{Firepower (2018)}
\end{minipage}
\begin{minipage}[t]{0.25\textwidth}\vspace{0pt}
\begin{itemize}[nosep,leftmargin=1em,labelwidth=*,align=left]
	\setlength{\itemsep}{0pt}
	\item Lightning Strike
\end{itemize}
\end{minipage}

\subsubsection{Kreator}

%=============
\begin{minipage}[t]{0.25\textwidth}
\captionsetup{type=figure}
\includegraphics[width=\textwidth]{Images/cover.png}
\caption*{Phantom Antichrist (2012)}
\end{minipage}
\begin{minipage}[t]{0.25\textwidth}\vspace{0pt}
\begin{itemize}[nosep,leftmargin=1em,labelwidth=*,align=left]
	\setlength{\itemsep}{0pt}
	\item Your Heaven, My Hell
	\item Phantom Antichrist
	\item Civilisation Collapse
\end{itemize}
\end{minipage}

\subsubsection{Metallica}

%=============
\begin{minipage}[t]{0.25\textwidth}
\captionsetup{type=figure}
\includegraphics[width=\textwidth]{Images/cover.png}
\caption*{Metallica (1991)}
\end{minipage}
\begin{minipage}[t]{0.25\textwidth}\vspace{0pt}
\begin{itemize}[nosep,leftmargin=1em,labelwidth=*,align=left]
	\setlength{\itemsep}{0pt}
	\item Enter Sandman
	\item The Unforgiven
	\item Nothing Else Matters
\end{itemize}
\end{minipage}
%=============
\begin{minipage}[t]{0.25\textwidth}
\captionsetup{type=figure}
\includegraphics[width=\textwidth]{Images/cover.png}
\caption*{S \& M (1999)}
\end{minipage}
\begin{minipage}[t]{0.25\textwidth}\vspace{0pt}
\begin{itemize}[nosep,leftmargin=1em,labelwidth=*,align=left]
	\setlength{\itemsep}{0pt}
	\item Master Of Puppets
	\item The Call Of The Ktulu
	\item Nothing Else Matters
	\item One
\end{itemize}
\end{minipage}
%=============
\begin{minipage}[t]{0.25\textwidth}
\captionsetup{type=figure}
\includegraphics[width=\textwidth]{Images/cover.png}
\caption*{Death Magnetic (2008)}
\end{minipage}
\begin{minipage}[t]{0.25\textwidth}\vspace{0pt}
\begin{itemize}[nosep,leftmargin=1em,labelwidth=*,align=left]
	\setlength{\itemsep}{0pt}
	\item The Day That Never Comes
	\item The Unforgiven III.
	\item My Apocalypse
\end{itemize}
\end{minipage}


\subsubsection{Motörhead}

%=============
\begin{minipage}[t]{0.25\textwidth}
\captionsetup{type=figure}
\includegraphics[width=\textwidth]{Images/cover.png}
\caption*{You'll Get Yours (2010)}
\end{minipage}
\begin{minipage}[t]{0.25\textwidth}\vspace{0pt}
\begin{itemize}[nosep,leftmargin=1em,labelwidth=*,align=left]
	\setlength{\itemsep}{0pt}
	\item Ace Of Spades
\end{itemize}
\end{minipage}

\subsubsection{Pantera}

%=============
\begin{minipage}[t]{0.25\textwidth}
\captionsetup{type=figure}
\includegraphics[width=\textwidth]{Images/cover.png}
\caption*{Cowboys from Hell (1990)}
\end{minipage}
\begin{minipage}[t]{0.25\textwidth}\vspace{0pt}
\begin{itemize}[nosep,leftmargin=1em,labelwidth=*,align=left]
	\setlength{\itemsep}{0pt}
	\item Cemetery Gates
\end{itemize}
\end{minipage}

\subsubsection{Saxon}

%=============
\begin{minipage}[t]{0.25\textwidth}
\captionsetup{type=figure}
\includegraphics[width=\textwidth]{Images/cover.png}
\caption*{Crusader (1984)}
\end{minipage}
\begin{minipage}[t]{0.25\textwidth}\vspace{0pt}
\begin{itemize}[nosep,leftmargin=1em,labelwidth=*,align=left]
	\setlength{\itemsep}{0pt}
	\item Crusader
	\item A Little Bit Of What You Fancy
	\item Just Let Me Rock
	\item Do It All For You
\end{itemize}
\end{minipage}

\subsubsection{Van Halen}

%=============
\begin{minipage}[t]{0.25\textwidth}
\captionsetup{type=figure}
\includegraphics[width=\textwidth]{Images/cover.png}
\caption*{Diver Down (1982)}
\end{minipage}
\begin{minipage}[t]{0.25\textwidth}\vspace{0pt}
\begin{itemize}[nosep,leftmargin=1em,labelwidth=*,align=left]
	\setlength{\itemsep}{0pt}
	\item (Oh) Pretty Woman
	\item Dancing In The Streets
	\item Little Guitars
	\item Cathedral
	\item Hang 'em High
\end{itemize}
\end{minipage}

%===========================================
% Rock, Punk, Ska etc. 
%===========================================

\section{Rock, Punk, Ska etc.}

\subsubsection{Anathema}

%=============
\begin{minipage}[t]{0.25\textwidth}
\captionsetup{type=figure}
\includegraphics[width=\textwidth]{Images/cover.png}
\caption*{Internal Landscapes - The Best Of 2008-2018 (2018)}
\end{minipage}
\begin{minipage}[t]{0.25\textwidth}\vspace{0pt}
\begin{itemize}[nosep,leftmargin=1em,labelwidth=*,align=left]
	\setlength{\itemsep}{0pt}
	\item Springfield
\end{itemize}
\end{minipage}

\subsubsection{Billy Talent}

%=============
\begin{minipage}[t]{0.25\textwidth}
\captionsetup{type=figure}
\includegraphics[width=\textwidth]{Images/cover.png}
\caption*{Billy Talent II (2006)}
\end{minipage}
\begin{minipage}[t]{0.25\textwidth}\vspace{0pt}
\begin{itemize}[nosep,leftmargin=1em,labelwidth=*,align=left]
	\setlength{\itemsep}{0pt}
	\item Devil In A Midnight Mass
	\item Fallen Leaves
	\item Red Flag
	\item Surrender 
	\item Worker Bees
\end{itemize}
\end{minipage}
%=============
\begin{minipage}[t]{0.25\textwidth}
\captionsetup{type=figure}
\includegraphics[width=\textwidth]{Images/cover.png}
\caption*{Dead Silence (2012)}
\end{minipage}
\begin{minipage}[t]{0.25\textwidth}\vspace{0pt}
\begin{itemize}[nosep,leftmargin=1em,labelwidth=*,align=left]
	\setlength{\itemsep}{0pt}
	\item Viking Death March
	\item Lonely Road To Absolution
	\item Surprise Surprise
	\item Hanging By A Thread
\end{itemize}
\end{minipage}

\subsubsection{Green Day}

%=============
\begin{minipage}[t]{0.25\textwidth}
\captionsetup{type=figure}
\includegraphics[width=\textwidth]{Images/cover.png}
\caption*{American Idiot (2004)}
\end{minipage}
\begin{minipage}[t]{0.25\textwidth}\vspace{0pt}
\begin{itemize}[nosep,leftmargin=1em,labelwidth=*,align=left]
	\setlength{\itemsep}{0pt}
	\item Wake Me Up When September Ends
	\item American Idiot
	\item Boulevard Of Broken Dreams
\end{itemize}
\end{minipage}

\subsubsection{Sum 41}

%=============
\begin{minipage}[t]{0.25\textwidth}
\captionsetup{type=figure}
\includegraphics[width=\textwidth]{Images/cover.png}
\caption*{Chuck (2004)}
\end{minipage}
\begin{minipage}[t]{0.25\textwidth}\vspace{0pt}
\begin{itemize}[nosep,leftmargin=1em,labelwidth=*,align=left]
	\setlength{\itemsep}{0pt}
	\item Pieces
\end{itemize}
\end{minipage}

\subsubsection{The Offspring}

%=============
\begin{minipage}[t]{0.25\textwidth}
\captionsetup{type=figure}
\includegraphics[width=\textwidth]{Images/cover.png}
\caption*{Rise and Fall, Rage and Grace  (2004)}
\end{minipage}
\begin{minipage}[t]{0.25\textwidth}\vspace{0pt}
\begin{itemize}[nosep,leftmargin=1em,labelwidth=*,align=left]
	\setlength{\itemsep}{0pt}
	\item You're Gonna Go Far Kid
	\item Kristy, Are You Doing Okay 
\end{itemize}
\end{minipage}

\subsubsection{Rise Against}

%=============
\begin{minipage}[t]{0.25\textwidth}
\captionsetup{type=figure}
\includegraphics[width=\textwidth]{Images/cover.png}
\caption*{Siren Song of the Counter Culture (2004)}
\end{minipage}
\begin{minipage}[t]{0.25\textwidth}\vspace{0pt}
\begin{itemize}[nosep,leftmargin=1em,labelwidth=*,align=left]
	\setlength{\itemsep}{0pt}
	\item Paper Wings
	\item Blood To Bleed
\end{itemize}
\end{minipage}
%=============
\begin{minipage}[t]{0.25\textwidth}\vspace{0pt}
\captionsetup{type=figure}
\includegraphics[width=\textwidth]{Images/cover.png}
\caption*{The Sufferer \& The Witness (2006)}
\end{minipage}
\begin{minipage}[t]{0.25\textwidth}\vspace{0pt}
\begin{itemize}[nosep,leftmargin=1em,labelwidth=*,align=left]
	\setlength{\itemsep}{0pt}
	\item Prayer Of The Refugee
	\item Under The Knife
\end{itemize}
\end{minipage}
%=============
\begin{minipage}[t]{0.25\textwidth}
\captionsetup{type=figure}
\includegraphics[width=\textwidth]{Images/cover.png}
\caption*{Appeal to Reason (2008)}
\end{minipage}
\begin{minipage}[t]{0.25\textwidth}\vspace{0pt}
\begin{itemize}[nosep,leftmargin=1em,labelwidth=*,align=left]
	\setlength{\itemsep}{0pt}
	\item Long Forgotten Sons
	\item Savior
	\item Hero Of War
\end{itemize}
\end{minipage}
%=============
\begin{minipage}[t]{0.25\textwidth}
\captionsetup{type=figure}
\includegraphics[width=\textwidth]{Images/cover.png}
\caption*{Endgame (2011)}
\end{minipage}
\begin{minipage}[t]{0.25\textwidth}\vspace{0pt}
\begin{itemize}[nosep,leftmargin=1em,labelwidth=*,align=left]
	\setlength{\itemsep}{0pt}
	\item Make It Stop (September’s Children)
	\item Satellite
	\item Wait For Me
	\item Architects
\end{itemize}
\end{minipage}
%=============
\begin{minipage}[t]{0.25\textwidth}
\captionsetup{type=figure}
\includegraphics[width=\textwidth]{Images/cover.png}
\caption*{Long Forgotten Songs (2013)}
\end{minipage}
\begin{minipage}[t]{0.25\textwidth}\vspace{0pt}
\begin{itemize}[nosep,leftmargin=1em,labelwidth=*,align=left]
	\setlength{\itemsep}{0pt}
	\item Death Blossom
	\item Everchanging
\end{itemize}
\end{minipage}
%=============
\begin{minipage}[t]{0.25\textwidth}\vspace{0pt}
\captionsetup{type=figure}
\includegraphics[width=\textwidth]{Images/cover.png}
\caption*{The Black Market (2014)}
\end{minipage}
\begin{minipage}[t]{0.25\textwidth}\vspace{0pt}
\begin{itemize}[nosep,leftmargin=1em,labelwidth=*,align=left]
	\setlength{\itemsep}{0pt}
	\item I Don't Want To Be Here Anymore
	\item People Live Here
\end{itemize}
\end{minipage}
%=============
\begin{minipage}[t]{0.25\textwidth}
\captionsetup{type=figure}
\includegraphics[width=\textwidth]{Images/cover.png}
\caption*{Wolves (2017)}
\end{minipage}
\begin{minipage}[t]{0.25\textwidth}\vspace{0pt}
\begin{itemize}[nosep,leftmargin=1em,labelwidth=*,align=left]
	\setlength{\itemsep}{0pt}
	\item The Violence
\end{itemize}
\end{minipage}

\subsubsection{Anti-Flag}

%=============
\begin{minipage}[t]{0.25\textwidth}\vspace{0pt}
\captionsetup{type=figure}
\includegraphics[width=\textwidth]{Images/cover.png}
\caption*{The General Strike (2012)}
\end{minipage}
\begin{minipage}[t]{0.25\textwidth}\vspace{0pt}
\begin{itemize}[nosep,leftmargin=1em,labelwidth=*,align=left]
	\setlength{\itemsep}{0pt}
	\item Broken Bones
\end{itemize}
\end{minipage}
%=============
\begin{minipage}[t]{0.25\textwidth}\vspace{0pt}
\captionsetup{type=figure}
\includegraphics[width=\textwidth]{Images/cover.png}
\caption*{American Fall (2017)}
\end{minipage}
\begin{minipage}[t]{0.25\textwidth}\vspace{0pt}
\begin{itemize}[nosep,leftmargin=1em,labelwidth=*,align=left]
	\setlength{\itemsep}{0pt}
	\item American Attraction
	\item When The Wall Falls
	\item Digital Blackout
\end{itemize}
\end{minipage}

\subsubsection{Dropkick Murphys}

%=============
\begin{minipage}[t]{0.25\textwidth}\vspace{0pt}
\captionsetup{type=figure}
\includegraphics[width=\textwidth]{Images/cover.png}
\caption*{The Meanest Of Times (2007)}
\end{minipage}
\begin{minipage}[t]{0.25\textwidth}\vspace{0pt}
\begin{itemize}[nosep,leftmargin=1em,labelwidth=*,align=left]
	\setlength{\itemsep}{0pt}
	\item The State Of Massachusetts
\end{itemize}
\end{minipage}
%=============

\subsubsection{Buster Shuffle}

%=============
\begin{minipage}[t]{0.25\textwidth}
\captionsetup{type=figure}
\includegraphics[width=\textwidth]{Images/cover.png}
\caption*{Our Night Out (2009)}
\end{minipage}
\begin{minipage}[t]{0.25\textwidth}\vspace{0pt}
\begin{itemize}[nosep,leftmargin=1em,labelwidth=*,align=left]
	\setlength{\itemsep}{0pt}
	\item At The Bank 
	\item Me, Myself \& I
	\item I'm Into You
\end{itemize}
\end{minipage}

\subsubsection{Pvris}

%=============
\begin{minipage}[t]{0.25\textwidth}
\captionsetup{type=figure}
\includegraphics[width=\textwidth]{Images/cover.png}
\caption*{White Noise (2014)}
\end{minipage}
\begin{minipage}[t]{0.25\textwidth}\vspace{0pt}
\begin{itemize}[nosep,leftmargin=1em,labelwidth=*,align=left]
	\setlength{\itemsep}{0pt}
	\item St. Patrick 
	\item My House
\end{itemize}
\end{minipage}

\subsubsection{Audioslave}

%=============
\begin{minipage}[t]{0.25\textwidth}
\captionsetup{type=figure}
\includegraphics[width=\textwidth]{Images/cover.png}
\caption*{Audioslave (2002)}
\end{minipage}
\begin{minipage}[t]{0.25\textwidth}\vspace{0pt}
\begin{itemize}[nosep,leftmargin=1em,labelwidth=*,align=left]
	\setlength{\itemsep}{0pt}
	\item Like A Stone
\end{itemize}
\end{minipage}

\subsubsection{Stone Sour}

%=============
\begin{minipage}[t]{0.25\textwidth}
\captionsetup{type=figure}
\includegraphics[width=\textwidth]{Images/cover.png}
\caption*{Audio Secrecy (2010)}
\end{minipage}
\begin{minipage}[t]{0.25\textwidth}\vspace{0pt}
\begin{itemize}[nosep,leftmargin=1em,labelwidth=*,align=left]
	\setlength{\itemsep}{0pt}
	\item Say You'll Haunt Me
\end{itemize}
\end{minipage}

\subsubsection{Red Hot Chili Peppers}

%=============
\begin{minipage}[t]{0.25\textwidth}
\captionsetup{type=figure}
\includegraphics[width=\textwidth]{Images/cover.png}
\caption*{Greatest Hits (2003)}
\end{minipage}
\begin{minipage}[t]{0.25\textwidth}\vspace{0pt}
\begin{itemize}[nosep,leftmargin=1em,labelwidth=*,align=left]
	\setlength{\itemsep}{0pt}
	\item Californication 
	\item By The Way
	\item Scar Tissue
\end{itemize}
\end{minipage}

\subsubsection{Various}

%=============
\begin{minipage}[t]{0.25\textwidth}\vspace{0pt}
\captionsetup{type=figure}
\includegraphics[width=\textwidth]{Images/cover.png}
\caption*{Raid The Arcade - Armada Book Inspired Soundtrack (2018)}
\end{minipage}
\begin{minipage}[t]{0.25\textwidth}\vspace{0pt}
\begin{itemize}[nosep,leftmargin=1em,labelwidth=*,align=left]
	\setlength{\itemsep}{0pt}
	\item T.N.T.
	\item Black Betty 
	\item Another One Bites The Dust
\end{itemize}
\end{minipage}
%=============

\subsubsection{Pink}

%=============
\begin{minipage}[t]{0.25\textwidth}
\captionsetup{type=figure}
\includegraphics[width=\textwidth]{Images/cover.png}
\caption*{Sober (Single 2008)}
\end{minipage}
\begin{minipage}[t]{0.25\textwidth}\vspace{0pt}
\begin{itemize}[nosep,leftmargin=1em,labelwidth=*,align=left]
	\setlength{\itemsep}{0pt}
	\item Sober
\end{itemize}
\end{minipage}

\subsubsection{Ledger}

%=============
\begin{minipage}[t]{0.25\textwidth}\vspace{0pt}
\captionsetup{type=figure}
\includegraphics[width=\textwidth]{Images/cover.png}
\caption*{Ledger EP (2018)}
\end{minipage}
\begin{minipage}[t]{0.25\textwidth}\vspace{0pt}
\begin{itemize}[nosep,leftmargin=1em,labelwidth=*,align=left]
	\setlength{\itemsep}{0pt}
	\item Not Dead Yet
\end{itemize}
\end{minipage}
%=============

%===========================================
% Athmospheric & Nonmetal
%===========================================

\section{Atmospheric \& Nonmetal}

\subsubsection{Wardruna}

%=============
\begin{minipage}[t]{0.25\textwidth}
\captionsetup{type=figure}
\includegraphics[width=\textwidth]{Images/cover.png}
\caption*{Runaljod - Gap Var Ginnunga (2009)}
\end{minipage}
\begin{minipage}[t]{0.25\textwidth}\vspace{0pt}
\begin{itemize}[nosep,leftmargin=1em,labelwidth=*,align=left]
	\setlength{\itemsep}{0pt}
	\item Heimta Thurs
	\item Hagall
	\item Kauna
\end{itemize}
\end{minipage}
%=============
\begin{minipage}[t]{0.25\textwidth}
\captionsetup{type=figure}
\includegraphics[width=\textwidth]{Images/cover.png}
\caption*{Runaljod - Yggdrasil (2013)}
\end{minipage}
\begin{minipage}[t]{0.25\textwidth}\vspace{0pt}
\begin{itemize}[nosep,leftmargin=1em,labelwidth=*,align=left]
	\setlength{\itemsep}{0pt}
	\item Rotlaust Tre Fell
	\item Solringen
	\item Sowelu
	\item Helvegen
\end{itemize}
\end{minipage}
%=============
\begin{minipage}[t]{0.25\textwidth}
\captionsetup{type=figure}
\includegraphics[width=\textwidth]{Images/cover.png}
\caption*{Runaljod - Ragnarok (2016)}
\end{minipage}
\begin{minipage}[t]{0.25\textwidth}\vspace{0pt}
\begin{itemize}[nosep,leftmargin=1em,labelwidth=*,align=left]
	\setlength{\itemsep}{0pt}
	\item Wunjo
	\item Raido
	\item Odal 
	\item UruR
	\item Perto
\end{itemize}
\end{minipage}

\subsubsection{Einar Selvik \& Ivar Bjørnson}

%=============
\begin{minipage}[t]{0.25\textwidth}\vspace{0pt}
\captionsetup{type=figure}
\includegraphics[width=\textwidth]{Images/cover.png}
\caption*{Hugsja (2018)}
\end{minipage}
\begin{minipage}[t]{0.25\textwidth}\vspace{0pt}
\begin{itemize}[nosep,leftmargin=1em,labelwidth=*,align=left]
	\setlength{\itemsep}{0pt}
	\item WulthuR
	\item Nordvegen
\end{itemize}
\end{minipage}

\subsubsection{If These Trees Could Talk}

%=============
\begin{minipage}[t]{0.25\textwidth}
\captionsetup{type=figure}
\includegraphics[width=\textwidth]{Images/cover.png}
\caption*{The Bones of a Dying World (2016)}
\end{minipage}
\begin{minipage}[t]{0.25\textwidth}\vspace{0pt}
\begin{itemize}[nosep,leftmargin=1em,labelwidth=*,align=left]
	\setlength{\itemsep}{0pt}
	\item The Giving Tree
	\item Berlin
	\item Solstice
\end{itemize}
\end{minipage}

\subsubsection{A Light In The Dark}

%=============
\begin{minipage}[t]{0.25\textwidth}
\captionsetup{type=figure}
\includegraphics[width=\textwidth]{Images/cover.png}
\caption*{Vanished (2016)}
\end{minipage}
\begin{minipage}[t]{0.25\textwidth}\vspace{0pt}
\begin{itemize}[nosep,leftmargin=1em,labelwidth=*,align=left]
	\setlength{\itemsep}{0pt}
	\item I Tried To Forget
	\item Vanished
\end{itemize}
\end{minipage}
%=============
\begin{minipage}[t]{0.25\textwidth}
\captionsetup{type=figure}
\includegraphics[width=\textwidth]{Images/cover.png}
\caption*{Imperfect (Split 2017)}
\end{minipage}
\begin{minipage}[t]{0.25\textwidth}\vspace{0pt}
\begin{itemize}[nosep,leftmargin=1em,labelwidth=*,align=left]
	\setlength{\itemsep}{0pt}
	\item Uncertain
\end{itemize}
\end{minipage}

\subsubsection{Kauan}

%=============
\begin{minipage}[t]{0.25\textwidth}
\captionsetup{type=figure}
\includegraphics[width=\textwidth]{Images/cover.png}
\caption*{Aava tuulen maa (2009)}
\end{minipage}
\begin{minipage}[t]{0.25\textwidth}\vspace{0pt}
\begin{itemize}[nosep,leftmargin=1em,labelwidth=*,align=left]
	\setlength{\itemsep}{0pt}
	\item Valveuni
	\item Fohn
\end{itemize}
\end{minipage}
%=============
\begin{minipage}[t]{0.25\textwidth}
\captionsetup{type=figure}
\includegraphics[width=\textwidth]{Images/cover.png}
\caption*{Live Pirut \& Sorni Nai (2017)}
\end{minipage}
\begin{minipage}[t]{0.25\textwidth}\vspace{0pt}
\begin{itemize}[nosep,leftmargin=1em,labelwidth=*,align=left]
	\setlength{\itemsep}{0pt}
	\item Pirut
	\item Sorni Nai
\end{itemize}
\end{minipage}

\subsubsection{Skyforest}

%=============
\begin{minipage}[t]{0.25\textwidth}
\captionsetup{type=figure}
\includegraphics[width=\textwidth]{Images/cover.png}
\caption*{Unity (2016)}
\end{minipage}
\begin{minipage}[t]{0.25\textwidth}\vspace{0pt}
\begin{itemize}[nosep,leftmargin=1em,labelwidth=*,align=left]
	\setlength{\itemsep}{0pt}
	\item Autumnal Embrace
	\item A Graceful Spirit
	\item Reminiscence
\end{itemize}
\end{minipage}

\subsubsection{Ulvesang}

%=============
\begin{minipage}[t]{0.25\textwidth}
\captionsetup{type=figure}
\includegraphics[width=\textwidth]{Images/cover.png}
\caption*{Ulvesang (2015)}
\end{minipage}
\begin{minipage}[t]{0.25\textwidth}\vspace{0pt}
\begin{itemize}[nosep,leftmargin=1em,labelwidth=*,align=left]
	\setlength{\itemsep}{0pt}
	\item Litherpoan
	\item Two Rivers
\end{itemize}
\end{minipage}

%===========================================
% Buckethead
%===========================================

\section{Buckethead}

%=============
\begin{minipage}[t]{0.25\textwidth}
\captionsetup{type=figure}
\includegraphics[width=\textwidth]{Images/cover.png}
\caption*{March of the Slunks (2012)}
\end{minipage}
\begin{minipage}[t]{0.25\textwidth}\vspace{0pt}
\begin{itemize}[nosep,leftmargin=1em,labelwidth=*,align=left]
	\setlength{\itemsep}{0pt}
	\item Magellan's Maze
\end{itemize}
\end{minipage}
%=============
\begin{minipage}[t]{0.25\textwidth}
\captionsetup{type=figure}
\includegraphics[width=\textwidth]{Images/cover.png}
\caption*{Hold Me Forever (2014)}
\end{minipage}
\begin{minipage}[t]{0.25\textwidth}\vspace{0pt}
\begin{itemize}[nosep,leftmargin=1em,labelwidth=*,align=left]
	\setlength{\itemsep}{0pt}
	\item N
	\item Y
	\item C
\end{itemize}
\end{minipage}
%=============
\begin{minipage}[t]{0.25\textwidth}
\captionsetup{type=figure}
\includegraphics[width=\textwidth]{Images/cover.png}
\caption*{Lightboard (2016)}
\end{minipage}
\begin{minipage}[t]{0.25\textwidth}\vspace{0pt}
\begin{itemize}[nosep,leftmargin=1em,labelwidth=*,align=left]
	\setlength{\itemsep}{0pt}
	\item Lightboard
\end{itemize}
\end{minipage}

%===========================================
% Symphonic Metal
%===========================================

\section{Symphonic Metal}

\subsubsection{Nightwish}

%=============
\begin{minipage}[t]{0.25\textwidth}
\captionsetup{type=figure}
\includegraphics[width=\textwidth]{Images/cover.png}
\caption*{From Wishes To Eternity (2001)}
\end{minipage}
\begin{minipage}[t]{0.25\textwidth}\vspace{0pt}
\begin{itemize}[nosep,leftmargin=1em,labelwidth=*,align=left]
	\setlength{\itemsep}{0pt}
	\item Walking In The Air
	\item Wanderlust
	\item Elvenpath
\end{itemize}
\end{minipage}
%=============
\begin{minipage}[t]{0.25\textwidth}
\captionsetup{type=figure}
\includegraphics[width=\textwidth]{Images/cover.png}
\caption*{Once (2004)}
\end{minipage}
\begin{minipage}[t]{0.25\textwidth}\vspace{0pt}
\begin{itemize}[nosep,leftmargin=1em,labelwidth=*,align=left]
	\setlength{\itemsep}{0pt}
	\item Wish I Had An Angel
	\item Nemo
	\item Kuolema Tekee Taiteilijan
	\item Planet Hell
\end{itemize}
\end{minipage}
%=============
\begin{minipage}[t]{0.25\textwidth}
\captionsetup{type=figure}
\includegraphics[width=\textwidth]{Images/cover.png}
\caption*{Highest Hopes (2005)}
\end{minipage}
\begin{minipage}[t]{0.25\textwidth}\vspace{0pt}
\begin{itemize}[nosep,leftmargin=1em,labelwidth=*,align=left]
	\setlength{\itemsep}{0pt}
	\item Sleeping Sun
	\item Bless The Child
	\item Wishmaster
\end{itemize}
\end{minipage}
%=============
\begin{minipage}[t]{0.25\textwidth}
\captionsetup{type=figure}
\includegraphics[width=\textwidth]{Images/cover.png}
\caption*{End Of An Era (2006)}
\end{minipage}
\begin{minipage}[t]{0.25\textwidth}\vspace{0pt}
\begin{itemize}[nosep,leftmargin=1em,labelwidth=*,align=left]
	\setlength{\itemsep}{0pt}
	\item High Hopes
	\item Slaying The Dreamer
	\item Over The Hills And Far Away
	\item Ever Dream 
	\item Dark Chest Of Wonders
\end{itemize}
\end{minipage}

\subsubsection{Epica}

%=============
\begin{minipage}[t]{0.25\textwidth}
\captionsetup{type=figure}
\includegraphics[width=\textwidth]{Images/cover.png}
\caption*{The Divine Conspiracy (2007)}
\end{minipage}
\begin{minipage}[t]{0.25\textwidth}\vspace{0pt}
\begin{itemize}[nosep,leftmargin=1em,labelwidth=*,align=left]
	\setlength{\itemsep}{0pt}
	\item Never Enough
	\item The Divine Conspiracy
\end{itemize}
\end{minipage}
%=============
\begin{minipage}[t]{0.25\textwidth}
\captionsetup{type=figure}
\includegraphics[width=\textwidth]{Images/cover.png}
\caption*{Retrospect - 10th Anniversary (2013)}
\end{minipage}
\begin{minipage}[t]{0.25\textwidth}\vspace{0pt}
\begin{itemize}[nosep,leftmargin=1em,labelwidth=*,align=left]
	\setlength{\itemsep}{0pt}
	\item Introspect
	\item Quietus
	\item Sensorium
	\item Serenade Of Self-Destruction
	\item Storm The Sorrow
\end{itemize}
\end{minipage}
%=============
\begin{minipage}[t]{0.25\textwidth}
\captionsetup{type=figure}
\includegraphics[width=\textwidth]{Images/cover.png}
\caption*{The Quantum Enigma (2014)}
\end{minipage}
\begin{minipage}[t]{0.25\textwidth}\vspace{0pt}
\begin{itemize}[nosep,leftmargin=1em,labelwidth=*,align=left]
	\setlength{\itemsep}{0pt}
	\item The Essence Of Silence
	\item Canvas Of Life
	\item Natural Corruption
	\item Victims Of Contingency
	\item The Second Stone
\end{itemize}
\end{minipage}

\subsubsection{Within Temptation}

%=============
\begin{minipage}[t]{0.25\textwidth}
\captionsetup{type=figure}
\includegraphics[width=\textwidth]{Images/cover.png}
\caption*{Let Us Burn (2014)}
\end{minipage}
\begin{minipage}[t]{0.25\textwidth}\vspace{0pt}
\begin{itemize}[nosep,leftmargin=1em,labelwidth=*,align=left]
	\setlength{\itemsep}{0pt}
	\item Candles
	\item Iron
	\item Ice Queen
	\item Faster
	\item Sinead
	\item Stand My Ground
	\item And We Run
\end{itemize}
\end{minipage}

\subsubsection{EvaneScence}

%=============
\begin{minipage}[t]{0.25\textwidth}
\captionsetup{type=figure}
\includegraphics[width=\textwidth]{Images/cover.png}
\caption*{Fallen (2003)}
\end{minipage}
\begin{minipage}[t]{0.25\textwidth}\vspace{0pt}
\begin{itemize}[nosep,leftmargin=1em,labelwidth=*,align=left]
	\setlength{\itemsep}{0pt}
	\item My Immortal 
	\item Bring Me To Life
	\item Going Under
	\item Everybody's Fool
	\item Hello
\end{itemize}
\end{minipage}

\subsubsection{Beyond The Black}

%=============
\begin{minipage}[t]{0.25\textwidth}
\captionsetup{type=figure}
\includegraphics[width=\textwidth]{Images/cover.png}
\caption*{Lost in Forever (2016)}
\end{minipage}
\begin{minipage}[t]{0.25\textwidth}\vspace{0pt}
\begin{itemize}[nosep,leftmargin=1em,labelwidth=*,align=left]
	\setlength{\itemsep}{0pt}
	\item Lost In Forever
	\item Night Will Fade
\end{itemize}
\end{minipage}


\subsubsection{Blackbriar}

%=============
\begin{minipage}[t]{0.25\textwidth}
\captionsetup{type=figure}
\includegraphics[width=\textwidth]{Images/cover.png}
\caption*{We'd Rather Burn (2018)}
\end{minipage}
\begin{minipage}[t]{0.25\textwidth}\vspace{0pt}
\begin{itemize}[nosep,leftmargin=1em,labelwidth=*,align=left]
	\setlength{\itemsep}{0pt}
	\item I'd Rather Burn
\end{itemize}
\end{minipage}

\subsubsection{Delain}

%=============
\begin{minipage}[t]{0.25\textwidth}
\captionsetup{type=figure}
\includegraphics[width=\textwidth]{Images/cover.png}
\caption*{Interlude (2013)}
\end{minipage}
\begin{minipage}[t]{0.25\textwidth}\vspace{0pt}
\begin{itemize}[nosep,leftmargin=1em,labelwidth=*,align=left]
	\setlength{\itemsep}{0pt}
	\item Breathe On Me
	\item Such A Shame
	\item Are You Done With Me
	\item We Are The Others
\end{itemize}
\end{minipage}
%=============
\begin{minipage}[t]{0.25\textwidth}\vspace{0pt}
\captionsetup{type=figure}
\includegraphics[width=\textwidth]{Images/cover.png}
\caption*{A Decade Of Delain  (2017)}
\end{minipage}
\begin{minipage}[t]{0.25\textwidth}\vspace{0pt}
\begin{itemize}[nosep,leftmargin=1em,labelwidth=*,align=left]
	\setlength{\itemsep}{0pt}
	\item Fire With Fire 
\end{itemize}
\end{minipage}

\subsubsection{Indica}

%=============
\begin{minipage}[t]{0.25\textwidth}
\captionsetup{type=figure}
\includegraphics[width=\textwidth]{Images/cover.png}
\caption*{A Way Away (2010)}
\end{minipage}
\begin{minipage}[t]{0.25\textwidth}\vspace{0pt}
\begin{itemize}[nosep,leftmargin=1em,labelwidth=*,align=left]
	\setlength{\itemsep}{0pt}
	\item Precious Dark
	\item In Passing
	\item Scissor, Paper, Rock
\end{itemize}
\end{minipage}

\subsubsection{Wildpath}

%=============
\begin{minipage}[t]{0.25\textwidth}
\captionsetup{type=figure}
\includegraphics[width=\textwidth]{Images/cover.png}
\caption*{Disclosure (2015)}
\end{minipage}
\begin{minipage}[t]{0.25\textwidth}\vspace{0pt}
\begin{itemize}[nosep,leftmargin=1em,labelwidth=*,align=left]
	\setlength{\itemsep}{0pt}
	\item Hollow
	\item Disclosure
\end{itemize}
\end{minipage}

\subsubsection{Apocalyptica}

%=============
\begin{minipage}[t]{0.25\textwidth}
\captionsetup{type=figure}
\includegraphics[width=\textwidth]{Images/cover.png}
\caption*{Plays Metallica  (1996)}
\end{minipage}
\begin{minipage}[t]{0.25\textwidth}\vspace{0pt}
\begin{itemize}[nosep,leftmargin=1em,labelwidth=*,align=left]
	\setlength{\itemsep}{0pt}
	\item Bittersweet
\end{itemize}
\end{minipage}
%=============
\begin{minipage}[t]{0.25\textwidth}
\captionsetup{type=figure}
\includegraphics[width=\textwidth]{Images/cover.png}
\caption*{Apocalyptica (2005)}
\end{minipage}
\begin{minipage}[t]{0.25\textwidth}\vspace{0pt}
\begin{itemize}[nosep,leftmargin=1em,labelwidth=*,align=left]
	\setlength{\itemsep}{0pt}
	\item The Unforgiven
\end{itemize}
\end{minipage}

%===========================================
% Power Metal
%===========================================

\section{Power Metal}

\subsubsection{Blind Guardian}

%=============
\begin{minipage}[t]{0.25\textwidth}
\captionsetup{type=figure}
\includegraphics[width=\textwidth]{Images/cover.png}
\caption*{Somewhere Far Beyond (1992)}
\end{minipage}
\begin{minipage}[t]{0.25\textwidth}\vspace{0pt}
\begin{itemize}[nosep,leftmargin=1em,labelwidth=*,align=left]
	\setlength{\itemsep}{0pt}
	\item The Bards Song (In The Forest)
	\item Ashes To Ashes
	\item The Quest For Tanelorn
\end{itemize}
\end{minipage}
%=============
\begin{minipage}[t]{0.25\textwidth}
\captionsetup{type=figure}
\includegraphics[width=\textwidth]{Images/cover.png}
\caption*{Imaginations Through the Looking Glass (2004)}
\end{minipage}
\begin{minipage}[t]{0.25\textwidth}\vspace{0pt}
\begin{itemize}[nosep,leftmargin=1em,labelwidth=*,align=left]
	\setlength{\itemsep}{0pt}
	\item A Past \& Future Secret
	\item Bright Eyes
	\item Imaginations From The Other Side
	\item I'm Alive
	\item Mordred's Song
	\item The Last Candle
	\item And Then There Was Silence
\end{itemize}
\end{minipage}
%=============
\begin{minipage}[t]{0.25\textwidth}
\captionsetup{type=figure}
\includegraphics[width=\textwidth]{Images/cover.png}
\caption*{At the Edge of Time (2010)}
\end{minipage}
\begin{minipage}[t]{0.25\textwidth}\vspace{0pt}
\begin{itemize}[nosep,leftmargin=1em,labelwidth=*,align=left]
	\setlength{\itemsep}{0pt}
	\item Sacred Worlds
	\item Tanelorn (Into The Void)
	\item Ride Into Obesession
\end{itemize}
\end{minipage}
%=============
\begin{minipage}[t]{0.25\textwidth}
\captionsetup{type=figure}
\includegraphics[width=\textwidth]{Images/cover.png}
\caption*{Memories of a Time to Come (2012)}
\end{minipage}
\begin{minipage}[t]{0.25\textwidth}\vspace{0pt}
\begin{itemize}[nosep,leftmargin=1em,labelwidth=*,align=left]
	\setlength{\itemsep}{0pt}
	\item And Then There Was Silence
	\item Valhalla
	\item Somewhere Far Beyond
	\item Mirror Mirror
\end{itemize}
\end{minipage}
%=============
\begin{minipage}[t]{0.25\textwidth}
\captionsetup{type=figure}
\includegraphics[width=\textwidth]{Images/cover.png}
\caption*{Beyond the Red Mirror (2015)}
\end{minipage}
\begin{minipage}[t]{0.25\textwidth}\vspace{0pt}
\begin{itemize}[nosep,leftmargin=1em,labelwidth=*,align=left]
	\setlength{\itemsep}{0pt}
	\item Distant Memories
	\item Miracle Machine
	\item Grand Parade
\end{itemize}
\end{minipage}

\subsubsection{Sabaton}

%=============
\begin{minipage}[t]{0.25\textwidth}
\captionsetup{type=figure}
\includegraphics[width=\textwidth]{Images/cover.png}
\caption*{The Art of War (2008)}
\end{minipage}
\begin{minipage}[t]{0.25\textwidth}\vspace{0pt}
\begin{itemize}[nosep,leftmargin=1em,labelwidth=*,align=left]
	\setlength{\itemsep}{0pt}
	\item Swedish Pagans
	\item 40:1
	\item The Art Of War
	\item Ghost Division
\end{itemize}
\end{minipage}
%=============
\begin{minipage}[t]{0.25\textwidth}
\captionsetup{type=figure}
\includegraphics[width=\textwidth]{Images/cover.png}
\caption*{Coat of Arms (2010)}
\end{minipage}
\begin{minipage}[t]{0.25\textwidth}\vspace{0pt}
\begin{itemize}[nosep,leftmargin=1em,labelwidth=*,align=left]
	\setlength{\itemsep}{0pt}
	\item Uprising
	\item Coat Of Arms
\end{itemize}
\end{minipage}
%=============
\begin{minipage}[t]{0.25\textwidth}
\captionsetup{type=figure}
\includegraphics[width=\textwidth]{Images/cover.png}
\caption*{Heroes (2014)}
\end{minipage}
\begin{minipage}[t]{0.25\textwidth}\vspace{0pt}
\begin{itemize}[nosep,leftmargin=1em,labelwidth=*,align=left]
	\setlength{\itemsep}{0pt}
	\item To Hell And Back
	\item Night Witches
	\item Resist And Bite
\end{itemize}
\end{minipage}
%=============
\begin{minipage}[t]{0.25\textwidth}\vspace{0pt}
\captionsetup{type=figure}
\includegraphics[width=\textwidth]{Images/cover.png}
\caption*{The Last Stand (2016)}
\end{minipage}
\begin{minipage}[t]{0.25\textwidth}\vspace{0pt}
\begin{itemize}[nosep,leftmargin=1em,labelwidth=*,align=left]
	\setlength{\itemsep}{0pt}
	\item Sparta
	\item Shiroyama
\end{itemize}
\end{minipage}
%=============
\begin{minipage}[t]{0.25\textwidth}\vspace{0pt}
\captionsetup{type=figure}
\includegraphics[width=\textwidth]{Images/cover.png}
\caption*{The Great War (2019)}
\end{minipage}
\begin{minipage}[t]{0.25\textwidth}\vspace{0pt}
\begin{itemize}[nosep,leftmargin=1em,labelwidth=*,align=left]
	\setlength{\itemsep}{0pt}
	\item Attack Of The Dead Man
	\item The Red Baron
\end{itemize}
\end{minipage}



\subsubsection{Helloween}

%=============
\begin{minipage}[t]{0.25\textwidth}
\captionsetup{type=figure}
\includegraphics[width=\textwidth]{Images/cover.png}
\caption*{Unarmed (2010)}
\end{minipage}
\begin{minipage}[t]{0.25\textwidth}\vspace{0pt}
\begin{itemize}[nosep,leftmargin=1em,labelwidth=*,align=left]
	\setlength{\itemsep}{0pt}
	\item Future World 
	\item If I Could Fly
	\item Perfect Gentleman 
	\item Dr. Stein
\end{itemize}
\end{minipage}
%=============
\begin{minipage}[t]{0.25\textwidth}
\captionsetup{type=figure}
\includegraphics[width=\textwidth]{Images/cover.png}
\caption*{Ride the Sky (2016)}
\end{minipage}
\begin{minipage}[t]{0.25\textwidth}\vspace{0pt}
\begin{itemize}[nosep,leftmargin=1em,labelwidth=*,align=left]
	\setlength{\itemsep}{0pt}
	\item Future World 
	\item I Want Out
	\item Dr. Stein
\end{itemize}
\end{minipage}

\subsubsection{Powerwolf}

%=============
\begin{minipage}[t]{0.25\textwidth}
\captionsetup{type=figure}
\includegraphics[width=\textwidth]{Images/cover.png}
\caption*{Blessed \& Possessed (2015)}
\end{minipage}
\begin{minipage}[t]{0.25\textwidth}\vspace{0pt}
\begin{itemize}[nosep,leftmargin=1em,labelwidth=*,align=left]
	\setlength{\itemsep}{0pt}
	\item Blessed And Possessed
	\item Armata Strigoi
	\item Sacramental Sister
\end{itemize}
\end{minipage}

\subsubsection{Dragonforce}

%=============
\begin{minipage}[t]{0.25\textwidth}
\captionsetup{type=figure}
\includegraphics[width=\textwidth]{Images/cover.png}
\caption*{Inhuman Rampage (2005)}
\end{minipage}
\begin{minipage}[t]{0.25\textwidth}\vspace{0pt}
\begin{itemize}[nosep,leftmargin=1em,labelwidth=*,align=left]
	\setlength{\itemsep}{0pt}
	\item Through The Fire And The Flames
\end{itemize}
\end{minipage}

\subsubsection{Gloryhammer}

%=============
\begin{minipage}[t]{0.25\textwidth}
\captionsetup{type=figure}
\includegraphics[width=\textwidth]{Images/cover.png}
\caption*{Space 1992: Rise of the Chaos Wizards (2015)}
\end{minipage}
\begin{minipage}[t]{0.25\textwidth}\vspace{0pt}
\begin{itemize}[nosep,leftmargin=1em,labelwidth=*,align=left]
	\setlength{\itemsep}{0pt}
	\item Universe On Fire
	\item Legend Of The Astral Hammer
\end{itemize}
\end{minipage}

\subsubsection{Hammerfall}

%=============
\begin{minipage}[t]{0.25\textwidth}
\captionsetup{type=figure}
\includegraphics[width=\textwidth]{Images/cover.png}
\caption*{No Sacrifice, No Victory (2009)}
\end{minipage}
\begin{minipage}[t]{0.25\textwidth}\vspace{0pt}
\begin{itemize}[nosep,leftmargin=1em,labelwidth=*,align=left]
	\setlength{\itemsep}{0pt}
	\item Any Means Necessary
	\item My Shanora
\end{itemize}
\end{minipage}

\subsubsection{Pentakill}

%=============
\begin{minipage}[t]{0.25\textwidth}
\captionsetup{type=figure}
\includegraphics[width=\textwidth]{Images/cover.png}
\caption*{Smite And Ignite (2014)}
\end{minipage}
\begin{minipage}[t]{0.25\textwidth}\vspace{0pt}
\begin{itemize}[nosep,leftmargin=1em,labelwidth=*,align=left]
	\setlength{\itemsep}{0pt}
	\item Deathfire Grasp
\end{itemize}
\end{minipage}
%=============
\begin{minipage}[t]{0.25\textwidth}
\captionsetup{type=figure}
\includegraphics[width=\textwidth]{Images/cover.png}
\caption*{II: Grasp of the Undying (2017)}
\end{minipage}
\begin{minipage}[t]{0.25\textwidth}\vspace{0pt}
\begin{itemize}[nosep,leftmargin=1em,labelwidth=*,align=left]
	\setlength{\itemsep}{0pt}
	\item Mortal Reminder
\end{itemize}
\end{minipage}

%===========================================
% Progressive Metal
%===========================================

\section{Progressive Metal}

\subsubsection{Animals As Leaders}

%=============
\begin{minipage}[t]{0.25\textwidth}
\captionsetup{type=figure}
\includegraphics[width=\textwidth]{Images/cover.png}
\caption*{Weightless (2011)}
\end{minipage}
\begin{minipage}[t]{0.25\textwidth}\vspace{0pt}
\begin{itemize}[nosep,leftmargin=1em,labelwidth=*,align=left]
	\setlength{\itemsep}{0pt}
	\item Weightless
	\item New Eden
\end{itemize}
\end{minipage}

\subsubsection{Earthside}

%=============
\begin{minipage}[t]{0.25\textwidth}
\captionsetup{type=figure}
\includegraphics[width=\textwidth]{Images/cover.png}
\caption*{A Dream in Static (2015)}
\end{minipage}
\begin{minipage}[t]{0.25\textwidth}\vspace{0pt}
\begin{itemize}[nosep,leftmargin=1em,labelwidth=*,align=left]
	\setlength{\itemsep}{0pt}
	\item A Dream In Static
	\item Skyline
	\item The Closest I've Come
\end{itemize}
\end{minipage}

\subsubsection{Polyphia}

%=============
\begin{minipage}[t]{0.25\textwidth}
\captionsetup{type=figure}
\includegraphics[width=\textwidth]{Images/cover.png}
\caption*{Renaissance (2016)}
\end{minipage}
\begin{minipage}[t]{0.25\textwidth}\vspace{0pt}
\begin{itemize}[nosep,leftmargin=1em,labelwidth=*,align=left]
	\setlength{\itemsep}{0pt}
	\item Culture Shock
	\item Euphoria
	\item Bittersweet
\end{itemize}
\end{minipage}

\subsubsection{Dream Theater}

%=============
\begin{minipage}[t]{0.25\textwidth}
\captionsetup{type=figure}
\includegraphics[width=\textwidth]{Images/cover.png}
\caption*{Dream Theater (2013)}
\end{minipage}
\begin{minipage}[t]{0.25\textwidth}\vspace{0pt}
\begin{itemize}[nosep,leftmargin=1em,labelwidth=*,align=left]
	\setlength{\itemsep}{0pt}
	\item Along For The Ride
	\item The Enemy Inside
\end{itemize}
\end{minipage}

\subsubsection{Sarah Longfield}

%=============
\begin{minipage}[t]{0.25\textwidth}
\captionsetup{type=figure}
\includegraphics[width=\textwidth]{Images/cover.png}
\caption*{Par Avion (2012)}
\end{minipage}
\begin{minipage}[t]{0.25\textwidth}\vspace{0pt}
\begin{itemize}[nosep,leftmargin=1em,labelwidth=*,align=left]
	\setlength{\itemsep}{0pt}
	\item Sea
\end{itemize}
\end{minipage}

\subsubsection{Master Boot Record}

%=============
\begin{minipage}[t]{0.25\textwidth}
\captionsetup{type=figure}
\includegraphics[width=\textwidth]{Images/cover.png}
\caption*{C:\\>COPY \*.\* A: \/V () (2017)}
\end{minipage}
\begin{minipage}[t]{0.25\textwidth}\vspace{0pt}
\begin{itemize}[nosep,leftmargin=1em,labelwidth=*,align=left]
	\setlength{\itemsep}{0pt}
	\item DEV.NFO
\end{itemize}
\end{minipage}

%===========================================
% Bandcamp Discoveries
%===========================================

\section{Bandcamp Discoveries \& Sampler}

\subsubsection{Distant Dream}

%=============
\begin{minipage}[t]{0.25\textwidth}
\captionsetup{type=figure}
\includegraphics[width=\textwidth]{Images/cover.png}
\caption*{It All Starts From Pieces (2017)}
\end{minipage}
\begin{minipage}[t]{0.25\textwidth}\vspace{0pt}
\begin{itemize}[nosep,leftmargin=1em,labelwidth=*,align=left]
	\setlength{\itemsep}{0pt}
	\item Timeless Colors
	\item A Touch Of The Sky
\end{itemize}
\end{minipage}

\subsubsection{Shadow Universe}

%=============
\begin{minipage}[t]{0.25\textwidth}
\captionsetup{type=figure}
\includegraphics[width=\textwidth]{Images/cover.png}
\caption*{The Unspeakable World (2017)}
\end{minipage}
\begin{minipage}[t]{0.25\textwidth}\vspace{0pt}
\begin{itemize}[nosep,leftmargin=1em,labelwidth=*,align=left]
	\setlength{\itemsep}{0pt}
	\item Pulsar
\end{itemize}
\end{minipage}

\subsubsection{Cloudkicker}

%=============
\begin{minipage}[t]{0.25\textwidth}
\captionsetup{type=figure}
\includegraphics[width=\textwidth]{Images/cover.png}
\caption*{Let Yourself Be Huge (2011)}
\end{minipage}
\begin{minipage}[t]{0.25\textwidth}\vspace{0pt}
\begin{itemize}[nosep,leftmargin=1em,labelwidth=*,align=left]
	\setlength{\itemsep}{0pt}
	\item You And Yours
	\item It's Inside Me, And I'm Inside It
\end{itemize}
\end{minipage}

\subsubsection{Earth Science}

%=============
\begin{minipage}[t]{0.25\textwidth}
\captionsetup{type=figure}
\includegraphics[width=\textwidth]{Images/cover.png}
\caption*{Flares (EP) (2011)}
\end{minipage}
\begin{minipage}[t]{0.25\textwidth}\vspace{0pt}
\begin{itemize}[nosep,leftmargin=1em,labelwidth=*,align=left]
	\setlength{\itemsep}{0pt}
	\item Whiskey Tango Foxtrott
\end{itemize}
\end{minipage}

\subsubsection{Hope For Heroes}

%=============
\begin{minipage}[t]{0.25\textwidth}
\captionsetup{type=figure}
\includegraphics[width=\textwidth]{Images/cover.png}
\caption*{Turnaround (2014)}
\end{minipage}
\begin{minipage}[t]{0.25\textwidth}\vspace{0pt}
\begin{itemize}[nosep,leftmargin=1em,labelwidth=*,align=left]
	\setlength{\itemsep}{0pt}
	\item The Room
	\item Turnaround
\end{itemize}
\end{minipage}

\subsubsection{Where The Good Way Lies}

%=============
\begin{minipage}[t]{0.25\textwidth}
\captionsetup{type=figure}
\includegraphics[width=\textwidth]{Images/cover.png}
\caption*{Nineteen Fourteen (2016)}
\end{minipage}
\begin{minipage}[t]{0.25\textwidth}\vspace{0pt}
\begin{itemize}[nosep,leftmargin=1em,labelwidth=*,align=left]
	\setlength{\itemsep}{0pt}
	\item Shadow March
\end{itemize}
\end{minipage}

\subsubsection{Dirtwire}

%=============
\begin{minipage}[t]{0.25\textwidth}
\captionsetup{type=figure}
\includegraphics[width=\textwidth]{Images/cover.png}
\caption*{Dirtwire (2012)}
\end{minipage}
\begin{minipage}[t]{0.25\textwidth}\vspace{0pt}
\begin{itemize}[nosep,leftmargin=1em,labelwidth=*,align=left]
	\setlength{\itemsep}{0pt}
	\item Hunter's Harp
	\item Amphibian Circuits
\end{itemize}
\end{minipage}

\subsubsection{Lulacruza}

%=============
\begin{minipage}[t]{0.25\textwidth}
\captionsetup{type=figure}
\includegraphics[width=\textwidth]{Images/cover.png}
\caption*{Orcas (2015)}
\end{minipage}
\begin{minipage}[t]{0.25\textwidth}\vspace{0pt}
\begin{itemize}[nosep,leftmargin=1em,labelwidth=*,align=left]
	\setlength{\itemsep}{0pt}
	\item Una Resuena
	\item Lagunita
\end{itemize}
\end{minipage}

\subsubsection{Doublestone}

%=============
\begin{minipage}[t]{0.25\textwidth}
\captionsetup{type=figure}
\includegraphics[width=\textwidth]{Images/cover.png}
\caption*{Wingmakers (2013)}
\end{minipage}
\begin{minipage}[t]{0.25\textwidth}\vspace{0pt}
\begin{itemize}[nosep,leftmargin=1em,labelwidth=*,align=left]
	\setlength{\itemsep}{0pt}
	\item Wingmakers
\end{itemize}
\end{minipage}

\subsubsection{Thenightyouleft}

%=============
\begin{minipage}[t]{0.25\textwidth}\vspace{0pt}
\captionsetup{type=figure}
\includegraphics[width=\textwidth]{Images/cover.png}
\caption*{The Woods (2015)}
\end{minipage}
\begin{minipage}[t]{0.25\textwidth}\vspace{0pt}
\begin{itemize}[nosep,leftmargin=1em,labelwidth=*,align=left]
	\setlength{\itemsep}{0pt}
	\item Of A Demon In My View
\end{itemize}
\end{minipage}

\subsubsection{Relapse Records}

%=============
\begin{minipage}[t]{0.25\textwidth}\vspace{0pt}
\captionsetup{type=figure}
\includegraphics[width=\textwidth]{Images/cover.png}
\caption*{Relapse Sampler (2015)}
\end{minipage}
\begin{minipage}[t]{0.25\textwidth}\vspace{0pt}
\begin{itemize}[nosep,leftmargin=1em,labelwidth=*,align=left]
	\setlength{\itemsep}{0pt}
	\item Myrkur - Mordet
\end{itemize}
\end{minipage}
%=============
\begin{minipage}[t]{0.25\textwidth}\vspace{0pt}
\captionsetup{type=figure}
\includegraphics[width=\textwidth]{Images/cover.png}
\caption*{Relapse Sampler (2016)}
\end{minipage}
\begin{minipage}[t]{0.25\textwidth}\vspace{0pt}
\begin{itemize}[nosep,leftmargin=1em,labelwidth=*,align=left]
	\setlength{\itemsep}{0pt}
	\item Myrkur - Onde Born 
\end{itemize}
\end{minipage}
%=============
\begin{minipage}[t]{0.25\textwidth}\vspace{0pt}
\captionsetup{type=figure}
\includegraphics[width=\textwidth]{Images/cover.png}
\caption*{Relapse Sampler (2017)}
\end{minipage}
\begin{minipage}[t]{0.25\textwidth}\vspace{0pt}
\begin{itemize}[nosep,leftmargin=1em,labelwidth=*,align=left]
	\setlength{\itemsep}{0pt}
	\item Myrkur - Ulvinde
\end{itemize}
\end{minipage}

\subsubsection{Willowtip}

%=============
\begin{minipage}[t]{0.25\textwidth}\vspace{0pt}
\captionsetup{type=figure}
\includegraphics[width=\textwidth]{Images/cover.png}
\caption*{Willowtip Sampler (2018)}
\end{minipage}
\begin{minipage}[t]{0.25\textwidth}\vspace{0pt}
\begin{itemize}[nosep,leftmargin=1em,labelwidth=*,align=left]
	\setlength{\itemsep}{0pt}
	\item Necrophagist - Mutilate The Stillborn
\end{itemize}
\end{minipage}
%=============

\subsubsection{Naturmacht Productions}

%=============
\begin{minipage}[t]{0.25\textwidth}\vspace{0pt}
\captionsetup{type=figure}
\includegraphics[width=\textwidth]{Images/cover.png}
\caption*{Naturmacht Compilation Vol. I (2009)}
\end{minipage}
\begin{minipage}[t]{0.25\textwidth}\vspace{0pt}
\begin{itemize}[nosep,leftmargin=1em,labelwidth=*,align=left]
	\setlength{\itemsep}{0pt}
	\item Agael - Legend
\end{itemize}
\end{minipage}
%=============
\begin{minipage}[t]{0.25\textwidth}\vspace{0pt}
\captionsetup{type=figure}
\includegraphics[width=\textwidth]{Images/cover.png}
\caption*{Naturmacht Compilation Vol. III (2010)}
\end{minipage}
\begin{minipage}[t]{0.25\textwidth}\vspace{0pt}
\begin{itemize}[nosep,leftmargin=1em,labelwidth=*,align=left]
	\setlength{\itemsep}{0pt}
	\item Cold Empire - Of Woods and Trees
\end{itemize}
\end{minipage}
%=============
\begin{minipage}[t]{0.25\textwidth}\vspace{0pt}
\captionsetup{type=figure}
\includegraphics[width=\textwidth]{Images/cover.png}
\caption*{Naturmacht Compilation Vol. III (2012)}
\end{minipage}
\begin{minipage}[t]{0.25\textwidth}\vspace{0pt}
\begin{itemize}[nosep,leftmargin=1em,labelwidth=*,align=left]
	\setlength{\itemsep}{0pt}
	\item Æðra - Horizon
\end{itemize}
\end{minipage}

\subsubsection{Neckbeard Deathcamp}

%=============
\begin{minipage}[t]{0.25\textwidth}\vspace{0pt}
\captionsetup{type=figure}
\includegraphics[width=\textwidth]{Images/cover.png}
\caption*{White Nationalism is for Basement Dwelling Losers (2018)}
\end{minipage}
\begin{minipage}[t]{0.25\textwidth}\vspace{0pt}
\begin{itemize}[nosep,leftmargin=1em,labelwidth=*,align=left]
	\setlength{\itemsep}{0pt}
	\item The Fetishization ov Asian Women Despite a Demand for a Pure White Race
\end{itemize}
\end{minipage}

\subsubsection{Rys}

%=============
\begin{minipage}[t]{0.25\textwidth}\vspace{0pt}
\captionsetup{type=figure}
\includegraphics[width=\textwidth]{Images/cover.png}
\caption*{Legacy (2017)}
\end{minipage}
\begin{minipage}[t]{0.25\textwidth}\vspace{0pt}
\begin{itemize}[nosep,leftmargin=1em,labelwidth=*,align=left]
	\setlength{\itemsep}{0pt}
	\item Legacy
	\item Unease
\end{itemize}
\end{minipage}


\subsubsection{The Circle Pit Compilation}

%=============
\begin{minipage}[t]{0.25\textwidth}\vspace{0pt}
\captionsetup{type=figure}
\includegraphics[width=\textwidth]{Images/cover.png}
\caption*{The Circle Pit Compilation II - Part One (2018)}
\end{minipage}
\begin{minipage}[t]{0.25\textwidth}\vspace{0pt}
\begin{itemize}[nosep,leftmargin=1em,labelwidth=*,align=left]
	\setlength{\itemsep}{0pt}
	\item Humanity's Last Breath - Harm
	\item Orbit Culture - Saw
\end{itemize}
\end{minipage}
%=============
\begin{minipage}[t]{0.25\textwidth}\vspace{0pt}
\captionsetup{type=figure}
\includegraphics[width=\textwidth]{Images/cover.png}
\caption*{The Circle Pit Compilation II - Part Two (2018)}
\end{minipage}
\begin{minipage}[t]{0.25\textwidth}\vspace{0pt}
\begin{itemize}[nosep,leftmargin=1em,labelwidth=*,align=left]
	\setlength{\itemsep}{0pt}
	\item For Giants - Big Sky
\end{itemize}
\end{minipage}
%=============
\begin{minipage}[t]{0.25\textwidth}\vspace{0pt}
\captionsetup{type=figure}
\includegraphics[width=\textwidth]{Images/cover.png}
\caption*{The Circle Pit Compilation II - Part Three (2018)}
\end{minipage}
\begin{minipage}[t]{0.25\textwidth}\vspace{0pt}
\begin{itemize}[nosep,leftmargin=1em,labelwidth=*,align=left]
	\setlength{\itemsep}{0pt}
	\item Winter's Gate - The Exile
\end{itemize}
\end{minipage}
%=============
\begin{minipage}[t]{0.25\textwidth}\vspace{0pt}
\captionsetup{type=figure}
\includegraphics[width=\textwidth]{Images/cover.png}
\caption*{The Circle Pit Compilation II - Part Four (2018)}
\end{minipage}
\begin{minipage}[t]{0.25\textwidth}\vspace{0pt}
\begin{itemize}[nosep,leftmargin=1em,labelwidth=*,align=left]
	\setlength{\itemsep}{0pt}
	\item Voidspawn - Pyrrhic
\end{itemize}
\end{minipage}

%===========================================
% Noise
%===========================================

\section{Noise}

\subsubsection{Frontierer}

%=============
\begin{minipage}[t]{0.25\textwidth}
\captionsetup{type=figure}
\includegraphics[width=\textwidth]{Images/cover.png}
\caption*{Orange Mathematics (2015)}
\end{minipage}
\begin{minipage}[t]{0.25\textwidth}\vspace{0pt}
\begin{itemize}[nosep,leftmargin=1em,labelwidth=*,align=left]
	\setlength{\itemsep}{0pt}
	\item Cascading Dialects
\end{itemize}
\end{minipage}

%

%===========================================
% MELODEATH
%===========================================
\cleardoublepage

\chapter{Non-Metal}\label{modern}

Most of the albums listed below are just listed (artist and album name) without favorite songs.

\section{Electronic}

\begin{itemize}
	\item Wintergatan - Wintergatan
	\item Detektivbyran - Wemland
	\item Gorillaz - Demon Days
	\item DEF CON 25 - The Official Soundtrack
	\item Favourite Hardstyle Music
	\item The Grand Sound - Trance
\end{itemize}

\section{Hip Hop}

\begin{itemize}
	\item Alligatoah - Triebwerke
	\item Watsky - Cardboard Castles
	\item Jan Böhmermann - Ich Hab Polizei
	\item The Jazz Hop Café - Jazz Hop \#3
	\item Chillhop Records - Chillhop Essentials - Fall 2017
\end{itemize}

\section{Indie}

\begin{itemize}
	\item Eivor - Room
	\item Pomme - En cavale 
\end{itemize}

\section{Orchestral \& Instrumental}

\begin{itemize}
	\item Adrian von Ziegler - Starchaser
	\item Adrian von Ziegler - The Celtic Collection
	\item An Evening In Rivendell 
	\item A Night in Rivendell
	\item Havasi - Symphonic II.
	\item Two Steps From Hell - Classics I
	\item Two Steps From Hell - Classics II
	\item Wim Mertens - Struggle For Pleasure
	\item Yann Tiersen - Goodbye Lenin
\end{itemize}

\section{Pop}

\begin{itemize}
	\item Abba - 18 Hits
	\item Abba - Waterloo
	\item EAV - Best Of
	\item Rock'n'Pop - Christmas
	\item Sarah Brightman - A Winter Symphony
	\item The Dome 43
	\item The Dome 50
	\item The Dome 49
	\item Gregorian - Masters Of Chant Chapter VI
\end{itemize}

\section{Reggae}

\begin{itemize}
	\item Damian Marley - Welcome To Jamrock
\end{itemize}

\section{Soundtrack}

\begin{itemize}
	\item Dreamfall Chapters 
	\item Dreamfall
	\item Endless Legend
	\item Gothic 2
	\item Der Herr Der Ringe - Die Gefährten
	\item Der Herr Der Ringe - Die Zwei Türme
	\item Der Herr Der Ringe - Die Rückkehr Des Königs
	\item Der Hobbit - An Unexpected Journey
	\item Der Hobbit - Desolation Of Smaug
	\item Der Hobbit - Battle Of The Five Armies
	\item Leinwandträume
	\item Mass Effect 2
	\item Mass Effect 3
	\item Outlander
	\item Shadowrun Hong Kong
	\item The Banner Saga
	\item The Witcher 3 
	\item Transistor
	\item Whiplash
	\item Game Of Thrones - Season 6
	\item Flesh And Bone (Adam Crystal)
	\item Diamond City Radio - Music Inspired by Fallout 4
\end{itemize}

\cleardoublepage
\section{Klassik \& Jazz}

\begin{itemize}
	\item 100 Meisterwerke der Klassik
	\item Bach - Toccata \& Fuge
	\item Buddy Rich - Blues Caravan
	\item David Garret - Rock Symphonies
	\item Keith Jarret - Creation
	\item Lang Lang - Liszt
	\item Lang Lang - Live in Vienna
	\item Ludovico Einaudi - Divenire
	\item Ludovico Einaudi - Elements
	\item Magic Moments - In The Spitit Of Jazz
	\item Marcin Patrzalek - Hush
	\item Orgelsax - Concerto Europeo
	\item Orgelsax - Ich öffne die Tür weit am Abend
	\item Piano Collection (25 CDs)
	\item Piano Nocturnes
	\item Piano Perlen
	\item Piano Poesie
	\item Schöne Weihnacht
	\item Sommernacht
	\item Swing With Cicero
	\item The New Sound Of Classic
	\item Träumerei
	\item Zia - Many And Great Are Thy Things
\end{itemize}

\section{Other Albums}

\subsubsection{Bloodhound Gang}

%=============
\begin{minipage}[t]{0.25\textwidth}\vspace{0pt}
\captionsetup{type=figure}
\includegraphics[width=\textwidth]{Images/cover.png}
\caption*{Show Us Your Hits (2010)}
\end{minipage}
\begin{minipage}[t]{0.25\textwidth}\vspace{0pt}
\begin{itemize}[nosep,leftmargin=1em,labelwidth=*,align=left]
	\setlength{\itemsep}{0pt}
	\item Uhn Tiss Uhn Tiss Uhn Tiss
\end{itemize}
\end{minipage}

\subsubsection{L'Orchestra Cinématique}

%=============
\begin{minipage}[t]{0.25\textwidth}\vspace{0pt}
\captionsetup{type=figure}
\includegraphics[width=\textwidth]{Images/cover.png}
\caption*{Epic Christmas (2017)}
\end{minipage}
\begin{minipage}[t]{0.25\textwidth}\vspace{0pt}
\begin{itemize}[nosep,leftmargin=1em,labelwidth=*,align=left]
	\setlength{\itemsep}{0pt}
	\item Oh Come Oh Come Emmanuel
\end{itemize}
\end{minipage}

\subsubsection{Hans Zimmer}

%=============
\begin{minipage}[t]{0.25\textwidth}\vspace{0pt}
\captionsetup{type=figure}
\includegraphics[width=\textwidth]{Images/cover.png}
\caption*{Live In Prague (2017)}
\end{minipage}
\begin{minipage}[t]{0.25\textwidth}\vspace{0pt}
\begin{itemize}[nosep,leftmargin=1em,labelwidth=*,align=left]
	\setlength{\itemsep}{0pt}
	\item Interstellar Medley
\end{itemize}
\end{minipage}

\subsubsection{Ivan Torrent}

%=============
\begin{minipage}[t]{0.25\textwidth}\vspace{0pt}
\captionsetup{type=figure}
\includegraphics[width=\textwidth]{Images/cover.png}
\caption*{Reverie - The Compilation Album(2014)}
\end{minipage}
\begin{minipage}[t]{0.25\textwidth}\vspace{0pt}
\begin{itemize}[nosep,leftmargin=1em,labelwidth=*,align=left]
	\setlength{\itemsep}{0pt}
	\item Forbidden Love
\end{itemize}
\end{minipage}


\subsubsection{Eddie Van Der Meer}

%=============
\begin{minipage}[t]{0.25\textwidth}\vspace{0pt}
\captionsetup{type=figure}
\includegraphics[width=\textwidth]{Images/cover.png}
\caption*{Cover Songs \#7 (2017)}
\end{minipage}
\begin{minipage}[t]{0.25\textwidth}\vspace{0pt}
\begin{itemize}[nosep,leftmargin=1em,labelwidth=*,align=left]
	\setlength{\itemsep}{0pt}
	\item New Rules
\end{itemize}
\end{minipage}

\subsubsection{Divinity Original Sin 2}

%=============
\begin{minipage}[t]{0.25\textwidth}\vspace{0pt}
\captionsetup{type=figure}
\includegraphics[width=\textwidth]{Images/cover.png}
\caption*{Divinity Original Sin 2 (2017)}
\end{minipage}
\begin{minipage}[t]{0.25\textwidth}\vspace{0pt}
\begin{itemize}[nosep,leftmargin=1em,labelwidth=*,align=left]
	\setlength{\itemsep}{0pt}
	\item Main Theme
\end{itemize}
\end{minipage}


\subsubsection{Rundfunktanzorchester Ehrenfeld \& Jan Böhmermann}

%=============
\begin{minipage}[t]{0.25\textwidth}\vspace{0pt}
\captionsetup{type=figure}
\includegraphics[width=\textwidth]{Images/cover.png}
\caption*{Neo Magazin Royale: Live in Concert (2016)}
\end{minipage}
\begin{minipage}[t]{0.25\textwidth}\vspace{0pt}
\begin{itemize}[nosep,leftmargin=1em,labelwidth=*,align=left]
	\setlength{\itemsep}{0pt}
	\item Baby Got Laugengebäck
\end{itemize}
\end{minipage}

\subsubsection{Polizistensohn aka Jan Böhmermann}

%=============
\begin{minipage}[t]{0.25\textwidth}\vspace{0pt}
\captionsetup{type=figure}
\includegraphics[width=\textwidth]{Images/cover.png}
\caption*{Recht Kommt EP (2018)}
\end{minipage}
\begin{minipage}[t]{0.25\textwidth}\vspace{0pt}
\begin{itemize}[nosep,leftmargin=1em,labelwidth=*,align=left]
	\setlength{\itemsep}{0pt}
	\item Recht Kommt
\end{itemize}
\end{minipage}

\subsubsection{Dua Lipa}

%=============
\begin{minipage}[t]{0.25\textwidth}\vspace{0pt}
\captionsetup{type=figure}
\includegraphics[width=\textwidth]{Images/cover.png}
\caption*{Be The One (EP) (2015)}
\end{minipage}
\begin{minipage}[t]{0.25\textwidth}\vspace{0pt}
\begin{itemize}[nosep,leftmargin=1em,labelwidth=*,align=left]
	\setlength{\itemsep}{0pt}
	\item Last Dance
\end{itemize}
\end{minipage}

\subsubsection{Aurora}

%=============
\begin{minipage}[t]{0.25\textwidth}\vspace{0pt}
\captionsetup{type=figure}
\includegraphics[width=\textwidth]{Images/cover.png}
\caption*{Running With The Wolves (EP)(2015)}
\end{minipage}
\begin{minipage}[t]{0.25\textwidth}\vspace{0pt}
\begin{itemize}[nosep,leftmargin=1em,labelwidth=*,align=left]
	\setlength{\itemsep}{0pt}
	\item Runaway
\end{itemize}
\end{minipage}

\subsubsection{Twenty One Pilots}

%=============
\begin{minipage}[t]{0.25\textwidth}\vspace{0pt}
\captionsetup{type=figure}
\includegraphics[width=\textwidth]{Images/cover.png}
\caption*{Blurryface (2015)}
\end{minipage}
\begin{minipage}[t]{0.25\textwidth}\vspace{0pt}
\begin{itemize}[nosep,leftmargin=1em,labelwidth=*,align=left]
	\setlength{\itemsep}{0pt}
	\item Ride
	\item Stressed Out
\end{itemize}
\end{minipage}

\subsubsection{Paramore}

%=============
\begin{minipage}[t]{0.25\textwidth}\vspace{0pt}
\captionsetup{type=figure}
\includegraphics[width=\textwidth]{Images/cover.png}
\caption*{That's What You Get (2008)}
\end{minipage}
\begin{minipage}[t]{0.25\textwidth}\vspace{0pt}
\begin{itemize}[nosep,leftmargin=1em,labelwidth=*,align=left]
	\setlength{\itemsep}{0pt}
	\item That’s What You Get
\end{itemize}
\end{minipage}

\subsubsection{Kontra K}

%=============
\begin{minipage}[t]{0.25\textwidth}\vspace{0pt}
\captionsetup{type=figure}
\includegraphics[width=\textwidth]{Images/cover.png}
\caption*{Aus Dem Schatten Ins Licht (2015)}
\end{minipage}
\begin{minipage}[t]{0.25\textwidth}\vspace{0pt}
\begin{itemize}[nosep,leftmargin=1em,labelwidth=*,align=left]
	\setlength{\itemsep}{0pt}
	\item Erfolg Ist Kein Glück
	\item Spring
	\item Kampfgeist 2
\end{itemize}
\end{minipage}

\subsubsection{Bliss N Eso}

%=============
\begin{minipage}[t]{0.25\textwidth}\vspace{0pt}
\captionsetup{type=figure}
\includegraphics[width=\textwidth]{Images/cover.png}
\caption*{Off The Grid(2017)}
\end{minipage}
\begin{minipage}[t]{0.25\textwidth}\vspace{0pt}
\begin{itemize}[nosep,leftmargin=1em,labelwidth=*,align=left]
	\setlength{\itemsep}{0pt}
	\item Tear The Roof Off
	\item Moments
\end{itemize}
\end{minipage}

\subsubsection{K.I.Z}

%=============
\begin{minipage}[t]{0.25\textwidth}\vspace{0pt}
\captionsetup{type=figure}
\includegraphics[width=\textwidth]{Images/cover.png}
\caption*{Sexismus Gegen Rechts (2009)}
\end{minipage}
\begin{minipage}[t]{0.25\textwidth}\vspace{0pt}
\begin{itemize}[nosep,leftmargin=1em,labelwidth=*,align=left]
	\setlength{\itemsep}{0pt}
	\item Selbstjustiz
	\item Halbstark
	\item Das System (Die kleinen Dinge) (feat. Sido)
\end{itemize}
\end{minipage}

\subsubsection{Watsky}

%=============
\begin{minipage}[t]{0.25\textwidth}\vspace{0pt}
\captionsetup{type=figure}
\includegraphics[width=\textwidth]{Images/cover.png}
\caption*{Nothing Like The First Time (2012)}
\end{minipage}
\begin{minipage}[t]{0.25\textwidth}\vspace{0pt}
\begin{itemize}[nosep,leftmargin=1em,labelwidth=*,align=left]
	\setlength{\itemsep}{0pt}
	\item Wounded Healer
	\item IDGAF
\end{itemize}
\end{minipage}

\subsubsection{Pertubator}

%=============
\begin{minipage}[t]{0.25\textwidth}\vspace{0pt}
\captionsetup{type=figure}
\includegraphics[width=\textwidth]{Images/cover.png}
\caption*{The Uncanny Valley (2016)}
\end{minipage}
\begin{minipage}[t]{0.25\textwidth}\vspace{0pt}
\begin{itemize}[nosep,leftmargin=1em,labelwidth=*,align=left]
	\setlength{\itemsep}{0pt}
	\item Sentinent
\end{itemize}
\end{minipage}

\subsubsection{Parov Stelar}

%=============
\begin{minipage}[t]{0.25\textwidth}\vspace{0pt}
\captionsetup{type=figure}
\includegraphics[width=\textwidth]{Images/cover.png}
\caption*{The Paris Swing Box (2010)}
\end{minipage}
\begin{minipage}[t]{0.25\textwidth}\vspace{0pt}
\begin{itemize}[nosep,leftmargin=1em,labelwidth=*,align=left]
	\setlength{\itemsep}{0pt}
	\item Booty Swing
\end{itemize}
\end{minipage}

\subsubsection{Stranger Things}

%=============
\begin{minipage}[t]{0.25\textwidth}\vspace{0pt}
\captionsetup{type=figure}
\includegraphics[width=\textwidth]{Images/cover.png}
\caption*{Stranger Things: Music From The Netflix Original Series (Double Vinyl) (2017)}
\end{minipage}
\begin{minipage}[t]{0.25\textwidth}\vspace{0pt}
\begin{itemize}[nosep,leftmargin=1em,labelwidth=*,align=left]
	\setlength{\itemsep}{0pt}
	\item Runaway
	\item Rock You Like A Hurricane
	\item Africa
\end{itemize}
\end{minipage}

\subsubsection{Mitch Murder}

%=============
\begin{minipage}[t]{0.25\textwidth}\vspace{0pt}
\captionsetup{type=figure}
\includegraphics[width=\textwidth]{Images/cover.png}
\caption*{Selection 5 (2018)}
\end{minipage}
\begin{minipage}[t]{0.25\textwidth}\vspace{0pt}
\begin{itemize}[nosep,leftmargin=1em,labelwidth=*,align=left]
	\setlength{\itemsep}{0pt}
	\item The Line
\end{itemize}
\end{minipage}

\subsubsection{Danger Mode}

%=============
\begin{minipage}[t]{0.25\textwidth}\vspace{0pt}
\captionsetup{type=figure}
\includegraphics[width=\textwidth]{Images/cover.png}
\caption*{Activation (2015)}
\end{minipage}
\begin{minipage}[t]{0.25\textwidth}\vspace{0pt}
\begin{itemize}[nosep,leftmargin=1em,labelwidth=*,align=left]
	\setlength{\itemsep}{0pt}
	\item High Velocity
\end{itemize}
\end{minipage}

\subsubsection{OSC}

%=============
\begin{minipage}[t]{0.25\textwidth}\vspace{0pt}
\captionsetup{type=figure}
\includegraphics[width=\textwidth]{Images/cover.png}
\caption*{Girls On Bike (2017)}
\end{minipage}
\begin{minipage}[t]{0.25\textwidth}\vspace{0pt}
\begin{itemize}[nosep,leftmargin=1em,labelwidth=*,align=left]
	\setlength{\itemsep}{0pt}
	\item Boys Fall Easy
\end{itemize}
\end{minipage}

\subsubsection{Zenon Records}

%=============
\begin{minipage}[t]{0.25\textwidth}\vspace{0pt}
\captionsetup{type=figure}
\includegraphics[width=\textwidth]{Images/cover.png}
\caption*{Selections 2018 (2018)}
\end{minipage}
\begin{minipage}[t]{0.25\textwidth}\vspace{0pt}
\begin{itemize}[nosep,leftmargin=1em,labelwidth=*,align=left]
	\setlength{\itemsep}{0pt}
	\item	The Waters of Lethe
\end{itemize}
\end{minipage}

\subsubsection{Merkaba Music}

%=============
\begin{minipage}[t]{0.25\textwidth}\vspace{0pt}
\captionsetup{type=figure}
\includegraphics[width=\textwidth]{Images/cover.png}
\caption*{100th Compilation (2019)}
\end{minipage}
\begin{minipage}[t]{0.25\textwidth}\vspace{0pt}
\begin{itemize}[nosep,leftmargin=1em,labelwidth=*,align=left]
	\setlength{\itemsep}{0pt}
	\item Deep Space
\end{itemize}
\end{minipage}

\subsubsection{Evan Marc + Steve Hillage}

%=============
\begin{minipage}[t]{0.25\textwidth}\vspace{0pt}
\captionsetup{type=figure}
\includegraphics[width=\textwidth]{Images/cover.png}
\caption*{Dreamtime Submersible (208)}
\end{minipage}
\begin{minipage}[t]{0.25\textwidth}\vspace{0pt}
\begin{itemize}[nosep,leftmargin=1em,labelwidth=*,align=left]
	\setlength{\itemsep}{0pt}
	\item Theta Phase
\end{itemize}
\end{minipage}


\subsection{Cover \& Mixtape}

\subsubsection{Game Of Thrones}

%=============
\begin{minipage}[t]{0.25\textwidth}\vspace{0pt}
\captionsetup{type=figure}
\includegraphics[width=\textwidth]{Images/cover.png}
\caption*{Catch The Throne (2014)}
\end{minipage}
\begin{minipage}[t]{0.25\textwidth}\vspace{0pt}
\begin{itemize}[nosep,leftmargin=1em,labelwidth=*,align=left]
	\setlength{\itemsep}{0pt}
	\item Born to Rule
\end{itemize}
\end{minipage}
%=============
\begin{minipage}[t]{0.25\textwidth}\vspace{0pt}
\captionsetup{type=figure}
\includegraphics[width=\textwidth]{Images/cover.png}
\caption*{Catch The Throne II (2015)}
\end{minipage}
\begin{minipage}[t]{0.25\textwidth}\vspace{0pt}
\begin{itemize}[nosep,leftmargin=1em,labelwidth=*,align=left]
	\setlength{\itemsep}{0pt}
	\item White Walker
	\item Soror Irrumator
	\item Loyalty
\end{itemize}
\end{minipage}

\subsubsection{Vladimir Zelentsov}

%=============
\begin{minipage}[t]{0.25\textwidth}\vspace{0pt}
\captionsetup{type=figure}
\includegraphics[width=\textwidth]{Images/cover.png}
\caption*{Guitar Covers And More (2015)}
\end{minipage}
\begin{minipage}[t]{0.25\textwidth}\vspace{0pt}
\begin{itemize}[nosep,leftmargin=1em,labelwidth=*,align=left]
	\setlength{\itemsep}{0pt}
	\item Freestyler
\end{itemize}
\end{minipage}

\subsection{Free}

\subsubsection{Two Bears High-Fiving}

%=============
\begin{minipage}[t]{0.25\textwidth}\vspace{0pt}
	\captionsetup{type=figure}
	\includegraphics[width=\textwidth]{Images/cover.png}
	\caption*{Button Mashing (Instrumental Album) (2013)}
\end{minipage}
\begin{minipage}[t]{0.25\textwidth}\vspace{0pt}
	\begin{itemize}[nosep,leftmargin=1em,labelwidth=*,align=left]
		\setlength{\itemsep}{0pt}
		\item Into The Wilderness
	\end{itemize}
\end{minipage}
%=============
\begin{minipage}[t]{0.25\textwidth}\vspace{0pt}
	\captionsetup{type=figure}
	\includegraphics[width=\textwidth]{Images/cover.png}
	\caption*{VGMashup (2012)}
\end{minipage}
\begin{minipage}[t]{0.25\textwidth}\vspace{0pt}
	\begin{itemize}[nosep,leftmargin=1em,labelwidth=*,align=left]
		\setlength{\itemsep}{0pt}
		\item Common - Testify (Tales of Vesperia)
	\end{itemize}
\end{minipage}
%=============
\begin{minipage}[t]{0.25\textwidth}\vspace{0pt}
	\captionsetup{type=figure}
	\includegraphics[width=\textwidth]{Images/cover.png}
	\caption*{Button Mashing (2013)}
\end{minipage}
\begin{minipage}[t]{0.25\textwidth}\vspace{0pt}
	\begin{itemize}[nosep,leftmargin=1em,labelwidth=*,align=left]
		\setlength{\itemsep}{0pt}
		\item Rakim - Guess Who's Back (Persona 4 - Corner of Memories)
	\end{itemize}
\end{minipage}
%=============
\begin{minipage}[t]{0.25\textwidth}\vspace{0pt}
	\captionsetup{type=figure}
	\includegraphics[width=\textwidth]{Images/cover.png}
	\caption*{3-1 (2013)}
\end{minipage}
\begin{minipage}[t]{0.25\textwidth}\vspace{0pt}
	\begin{itemize}[nosep,leftmargin=1em,labelwidth=*,align=left]
		\setlength{\itemsep}{0pt}
		\item Childish Gambino - Firefly (Darksiders II)
	\end{itemize}
\end{minipage}
%=============
\begin{minipage}[t]{0.25\textwidth}\vspace{0pt}
	\captionsetup{type=figure}
	\includegraphics[width=\textwidth]{Images/cover.png}
	\caption*{Four Leaf Hova (2014)}
\end{minipage}
\begin{minipage}[t]{0.25\textwidth}\vspace{0pt}
	\begin{itemize}[nosep,leftmargin=1em,labelwidth=*,align=left]
		\setlength{\itemsep}{0pt}
		\item Moment of Clarity (Wiosna)
	\end{itemize}
\end{minipage}
%=============
\begin{minipage}[t]{0.25\textwidth}\vspace{0pt}
	\captionsetup{type=figure}
	\includegraphics[width=\textwidth]{Images/cover.png}
	\caption*{Katy Scary (2014)}
\end{minipage}
\begin{minipage}[t]{0.25\textwidth}\vspace{0pt}
	\begin{itemize}[nosep,leftmargin=1em,labelwidth=*,align=left]
		\setlength{\itemsep}{0pt}
		\item Wide Awake (Please Love Me...Once More)
	\end{itemize}
\end{minipage}


\section{Hip Hop \& R'n'B Mixtapes von Mixtapemonkey}

Chiddy Bang - The Swelly Express\\
Kid Cudi - Dat Kid From Cleveland\\
Kid Cudi - Kid Named Cudi\\
Kid Cudi - Rap Hard\\
Tech N9ne - Bad Season\\
Chance The Rapper - Acid Rap\\
Chance The Rapper - Coloring Book\\
Childish Gambino - STN MTN\\
Drake - So Far Gone\\
Frank Ocean - Nostalgia Ultra\\
Gucci Mane - Trapology\\
G-Unit - The Lost Flash Drive\\
Jayden Tilley - Youngblood\\
Lil Dicky - So Hard\\
Lil Uzi Vert - Lil Uzi Vs The World\\
Tupac - Tupac Duets\\
Wiz Khalifa - Kush Oj\\
Wiz Khalifa - Kush Oj 7 Year Anniversary EP\\
Angel Haze - Classick\\
Cardi B - Gangsta Bitch Music Vol 1\\
Cassie - Rocka Bye Baby\\
Georgia Reign - DopeboyzLuvMe\\
Honey Cocaine - Thug\_Love\\
Iggy Azalea - Glory EP\\
Jhene Aiko - Sailing Souls\\
Kamaiyah - A Good Night In The Ghetto\\
Kehlani - Cloud 19\\
Keke Palmer - Keke Palmer\\
Khrystal - QLC\\
Kreayshawn - Young Rich Flashy\\
Lexii Alijai - feelless\\
Marsha Ambrosius - Late Nights Earlier Mornings\\
Mila J - Covergirl\\
Mila J - Milaulongtime\\
Mila J - Westside\\
Shanell - 4 Christmas\\
Shanell - Midnight Mimosas\\
Telana - New Age Soul\\
Tennille - 10FDOOM\\
Tennille - A Bronx Tale\\
Tinashe - Amethyst\\
Tinashe - BlackWater\\
Tinashe - In Case We Die\\
Tinashe - Reverie\\
Tink - Winters Diary 4\\

\cleardoublepage
\chapter{Single Songs}\label{singles}

\section{Metal}

8Kids - Kann mich jemand hören\\
A Day To Remember - All I Want\\
Alex Schmeia - Hide and Seek\\
Alex Schmeia - Mayhem\\
Alex Schmeia - Reborn\\
Alex Schmeia - Too Late\\
Alex Starbard - D Minor Backing Track\\
Alex Starbard - Reflections\\
Alien Weaponry - Holding My Breath\\
All That Remains - The Thunder Rolls\\
All That Remains - What If I Was Nothing\\
Alyona Vargasova - Journey Through the Milky Way\\
Ambleside - Dear Mother\\
Amon Amarth - Crack the Sky\\
Amorphis - Bad Blood\\
Amorphis - House Of Sleep\\
Amorphis - The Bee\\
Amorphis - White Night\\
Amorphis - Wrong Direction\\
Anathema - Springfield\\
Andy James - The Wind That Shakes the Heart\\
Angel Vivaldi \& Andy James - Wave of Synergy\\
Ankor - Rockstar (Cover)\\
Any Given Day - Arise\\
Arch Enemy - The World Is Yours\\
Architects - Doomsday\\
Architects - Even If You Win, You're Still A Rat\\
Architects - Holy Hell\\
Architects - These Colours Don't Run\\
Archspire - Remote Tumour Seeker\\
Arthur Sowinski - Sad Backing Track in D Minor\\
Arthur Sowinski - Sad Backing Track in E Mino\\
As I Lay Dying - 94 Hours\\
As I Lay Dying - An Ocean Between Us\\
As I Lay Dying - My Own Grave\\
As I Lay Dying - The Sound Of Truth\\
As I May - No Way Back\\
At The Gates - Slaughter of the Soul\\
At The Gates - To Drink from the Night Itself\\
Auri - Night 13\\
Avenged Sevenfold - This Means War\\
Bad Religion - 21st Century (Digital Boy)\\
Bad Religion - Fuck You\\
Baroness - Shock Me\\
Beartooth - In Between\\
Behemoth - Chant For Ezkaton 2000 E.V.\\
Behemoth - God Dog\\
Behemoth - O Father O Satan O Sun!\\
Behemoth - Ov Fire And The Void\\
Being As An Ocean - This Loneliness Won't Be the Death of Me\\
Be'Lakor - Countless Skies\\
Be'Lakor - The Smoke Of Many Fires\\
Belphegor - Baphomet\\
Beyond The Black - Night Will Fade\\
Black Label Society - A Love Unreal\\
Blessthefall - Open Water [feat. Lights]\\
blink-182 - Adam's Song\\
blink-182 - I Miss You\\
blink-182 - She's Out Of Her Mind\\
blink-182 - What's My Age Again\\
Bloodred Hourglass - Where the Sinners Crawl\\
Bombus - I Call You Over\\
Bring Me The Horizon - Shadow Moses\\
Bring Me The Horizon - Sleepwalking\\
Brutus - Fire\\
Brutus - Sugar Dragon\\
Brutus - War\\
Buckethead - Big Sur Moon\\
Buckethead - Coma\\
Buckethead - Soothsayer (dedicated to Aunt Suzie)\\
Buckethead - Waiting Hare\\
Burning Witches - Black Widow\\
Caliban - This Oath\\
Callejon - Snake Mountain (Live)\\
Callejon - Utopia\\
Carach Angren - Charles Francis Coghlan\\
Catamenia - The Forests of Tomorrow\\
Children Of Bodom - Are You Dead Yet\\
Children Of Bodom - Needled 24-7\\
Civil War - Tombstone\\
Code Orange - Bleeding In The Blur [Explicit]\\
Code Orange - Forever\\
Cradle Of Filth - Blackest Magick In Practice\\
Cradle Of Filth - Heartbreak And Seance\\
Crown The Empire - Memories Of A Broken Heart\\
Cytotoxin - Abysm Nucleus\\
Dark Fortress - Ylem\\
Dark Tranquillity - The Science of Noise\\
Dawn Of Disease - Ascension Gate\\
Deep Purple - Sometimes I Feel Like Screaming\\
Deftones - My Own Summer (Shove It)\\
Delain - Fire With Fire\\
Der Weg Einer Freiheit - Letzte Sonne\\
Deserted Fear - Open Their Gates\\
Dethklok - Awaken\\
Devin Townsend Project - Deadhead (Live)\\
Diablo Blvd - Sing From The Gallows\\
Dick Dale - Miserlou\\
Die Apokalyptischen Reiter - Auf und nieder\\
Die Apokalyptischen Reiter - Der Rote Reiter\\
Dimmu Borgir - Gateways\\
Dool - Vantablack\\
Draconian - Stellar Tombs\\
Dropkick Murphys - Rose Tattoo\\
Dropkick Murphys - The State of Massachusetts\\
Echosmith - Tell Her You Love Her\\
Eisbrecher - Miststück\\
Eisregen - Elektro Hexe\\
Eisregen - Panzerschokolade\\
Eldamar - The Border Of Eldamar\\
Elevated Jam Tracks - Atmospheric Metal Ballad - G Minor\\
Elevated Jam Tracks - Wild Majestic Metal - E Minor\\
Eluveitie - Epona\\
Eluveitie - Neverland\\
Eluveitie - Rebirth\\
Emperor - I Am The Black Wizards\\
Equilibrium - Blut im Auge\\
Escape The Fate - Broken Heart\\
Evarose - All The Things She Said\\
Evarose - Flatline\\
Evergreen Terrace - Chaney Can't Quite Riff Like Helmet's Page Hamilton\\
Fallujah - Sanctuary\\
Fit For An Autopsy - Absolute Hope Absolute Hell\\
Fit For An Autopsy - Black Mammoth\\
Fit For An Autopsy - When the Bulbs Burn Out\\
Five Finger Death Punch - I Refuse\\
Fjoergyn - What a wonderful world\\
Foo Fighters - The Pretender\\
Frantic Amber - Soar\\
Frog Leap Studios - Africa Outro\\
Gary Moore - The Loner\\
Ghost - Dance Macabre\\
Ghost - Ghuleh Zombie Queen\\
Ghost - He Is\\
Ghost - Jigolo Har Megiddo\\
Ghost - Miasma\\
Ghost - Pro Memoria\\
Ghost - Rats\\
Gojira - Clone\\
Gojira - Space Time\\
Gojira - Oroborus\\
Grimner - Eldhjärta\\
Haggard - Awaking the Centuries\\
Halestorm - Black Vultures\\
Halestorm - Love Bites (So Do I)\\
Hatebreed - Honor Never Dies\\
Honeymoon Disease - Higher\\
Hungry Lights - Fothcrah\\
Hypocrisy - Eraser\\
Icon For Hire - Get Well\\
Icon For Hire - Make a Move\\
Immortal - All Shall Fall\\
In Extremo - Liam (Gälische Version)\\
In Flames - I Am Above\\
In Flames - (This Is Our) House\\
In Flames - We Will Remember\\
Infected Rain - Orphan Soul\\
Insomnium - Inertia\\
Insomnium - Through The Shadow\\
Ithilien - Walk Away\\
J.B.O. - Panzer Dance\\
Jinjer - Just Another\\
Joe Satriani - Midnight\\
Kardashev - Iota\\
Kmac2021 - The Ting Goes Djent\\
Knorkator - Alter Mann\\
Knorkator - Ding Inne Schnauze\\
Knorkator - Liebeslied\\
Knorkator - Rette Sich Wer Kann\\
Knorkator - Warum\\
Knorkator - Wir Werden Alle Sterben\\
Korpiklaani - Henkselipoika\\
Korpiklaani - Ieva's Polka\\
Korpiklaani - Rauta\\
Kreator - Satan Is Real\\
Kvelertak - Heksebrann\\
Lacuna Coil - Blood, Tears, Dust\\
Leprous - The Price\\
Lifelover - Androider\\
Lifelover - Kärlek, Becksvart Melankoli (Love, Pitch Black Melancholy)\\
Lord Of The Lost - Morgana\\
Make a Change... Kill Yourself - Sjælefred\\
Make Them Suffer - Save Yourself\\
Mechina - Progenitor\\
Melodic Metal Backing Track - E Minor (Extended Version)\\
Meshuggah - Bleed\\
Meshuggah - Future Breed Machine\\
Metallica - Call Of Ktulu\\
Metallica - Master Of Puppets\\
Metallica - Moth Into Flame\\
Michalina Malisz - Martyr\\
Minor Threat - Out of Step\\
MOL - Bruma\\
Mono Inc. - The Banks Of Eden\\
Montreal - Auf der faulen Haut\\
Mr Hurley und die Pulveraffen - Ach ja\\
Muse - Stockholm Syndrome\\
Music Is Win - Every Guitar Technique in One Solo\\
Music Is Win - I Wrote This Song in 60 Minutes\\
Myrkur - Ulvinde\\
Nachtblut - Antik\\
Nanowar Of Steel - Norwegian Reggaeton\\
Ne Obliviscaris - Painters of the Tempest, Pt. 2 (Triptych Lux)\\
Nekrogoblikon - We Need a Gimmick [Explicit]\\
Nightwish - While Your Lips Are Still Red\\
Noctem - Eidolon\\
Nothing More - Don't Stop\\
Nothing More - This Is The Time (Ballast)\\
Numenorean - Adore\\
Numenorean - Regret\\
Numenorean - Coma\\
Obscurity - Bergischer Hammer\\
Omnium Gatherum - Nail\\
Omnium Gatherum - New Dynamic\\
Omnium Gatherum - New World Shadows\\
Omnium Gatherum - Soul Journeys\\
Omnium Gatherum - Watcher Of The Skies\\
Orbit Culture - Halloween Theme by John Carpenter\\
Orphaned Land - Like Orpheus\\
Papa Roach - Help\\
Parasite Inc. - The Pulse of the Dead [Explicit]\\
Parkway Drive - A Deathless Song (feat. Jenna McDougall)\\
Parkway Drive - Horizons\\
Parkway Drive - Idols and Anchors\\
Parkway Drive - Prey\\
Parkway Drive - Shadow Boxing\\
Parkway Drive - Smoke 'Em If You Got 'Em\\
Parkway Drive - Wishing Wells\\
Perkele - Heart Full of Pride\\
Petur Ben - Svarthamar\\
Pieter Daarth Project - R.D.F.\\
Pieter Daarth Project - Touching The Void\\
Pink Floyd - Hey You\\
Pixies - Where Is My Mind\\
Powerwolf - Demons Are A Girl's Best Friend\\
Primordial - Wield Lightning to Split the Sun\\
Prophets Of Rage - Unfuck The World [Explicit]\\
Rage Against The Machine - Know Your Enemy (Remastered)\\
Rage Of Light - I Can, I Will\\
Rammstein - Frühling In Paris\\
Raubtier - Lat Napalmen Regna\\
Raunchy - Somewhere Along The Road\\
Rings Of Saturn - The Macrocosm\\
Rise Of The Northstar - Demonstrating My Saiya Style\\
Rise Of The Northstar - What The Fuck\\
Rivers Of Nihil - The Silent Life\\
Rob Scallon - Rain (Live \& Acoustic)\\
Rotting Christ - The Raven\\
Royal Republic - Underwear\\
Sabaton - The Attack Of The Dead Men\\
Saor - Guardians\\
Satyricon - K.I.N.G.\\
Satyricon - Mother North\\
Satyricon - Phoenix\\
Satyricon - The Infinity Of Time And Space\\
Scars On Broadway - Lives\\
Semisonic - Closing Time\\
Shylmagoghnar - I Am the Abyss\\
Sick Of It All - Step Down\\
Slayer - Raining Blood (Album Version)\\
Soilwork - Distortion Sleep\\
Soilwork - Rejection Role\\
Soilwork - Tongue\\
Steve Vai - For the Love of God (Live)\\
Stick To Your Guns - Amber\\
Storm Seeker - Destined Course\\
Suamenlejjona - Mahtisonni\\
Sum 41 - Pieces\\
Sum 41 - Sick Of Everyone\\
Sum 41 - Still Waiting\\
Sum 41 - With Me\\
Swallow The Sun - Firelights\\
Swallow The Sun - With You Came the Whole of the World's Tears\\
Swiss und Die Anderen - Kuhle Typen feat. Die Atzen\\
Ten Second Songs - Chop Suey\\
The Agonist - Take Me To Church\\
The Amity Affliction - I Bring The Weather With Me\\
The Amity Affliction - Pittsburgh (No Intro)\\
The Amorettes - Talk Nerdy to Me\\
The Oath - Silk Road\\
The Ocean - Permian The Great Dying\\
The Offspring - Self-Esteem\\
The Offspring - You're Gonna Go Far Kid\\
The Picturebooks - I Need That Oooh\\
The Pretty Reckless - House on a Hill\\
Thirty Seconds To Mars - A Beautiful Lie\\
Thousand Leaves - Kissing the Tears\\
Thundermother - It's Just A Tease\\
Thy Art Is Murder - Death Squad Anthem\\
Tribulation - Nightbound\\
Tribulation - The Lament\\
Trivium - Beyond Oblivion\\
Trivium - The Crusade\\
Trivium - The Heart From Your Hate\\
Turisas - Rasputin\\
Van Halen - Hot For Teacher\\
Varg - Rotkäppchen\\
Venues - We Are One\\
Vitalism - Gradus\\
W.A.S.P. - Miss You\\
Waxx - Turn Up\\
We Butter The Bread With Butter - Ohne Herz\\
While She Sleeps - Elephant\\
Wither Away - Hazel Eyes\\
Wolfheart - Routa, Pt.2\\
Wolfheart - The Hunt\\
Woods Of Ypres - I Was Buried In Mount Pleasant Cemetery\\
Year Of The Goat - Avaritia\\
Yngwie Malmsteen - Arpeggios From Hell (Bonus)\\
Zeal And Ardor - Built on Ashes\\
Zeal And Ardor - Don't You Dare\\
Zeal And Ardor - Fire of Motion\\

\section{Rock \& Pop}

257ers - Auseinanda\\
257ers - Holland\\
257ers - Warum\\
Adrian von Ziegler - Ótroðinn\\
Against The Current - Another You (Another Way)\\
Against The Current - Legends Never Die\\
Alan Walker - Faded\\
Alex Cameron - Big Enough\\
Alicia Keys - Empire State of Mind (Part II) Broken Down\\
Aliotta Haynes Jeremiah - Lake Shore Drive\\
Alligatoah - Du bist schön\\
Alligatoah - Lass liegen\\
Amy Winehouse - Fuck Me Pumps\\
Anavae - Are We Alone\\
Andrey Vinogradov - Medieval Tune\\
Anna Burch - 2 Cool 2 Care\\
anna RF - Why\\
anna RF feat Naadistan -  Tum Hi Ho\\
Archive - Bullets\\
Audio88 - Direkter Vergleich\\
Audio88 - Ein Besserer Mensch\\
Audio88 \& Yassin - Die Erde ist eine Scheide\\
Audio88 \& Yassin - Halleluja\\
Audio88 \& Yassin - Gnade (feat. Nico KIZ)\\
Audio88 \& Yassin - Regenschirm\\
Audio88 \& Yassin - Rettet die Wale und so\\
Audio88 \& Yassin - Schellen\\
Audio88 \& Yassin - Über Liebe\\
Aurora - The Seed\\
Azedia - Thunder \& Lightning\\
Bag Raiders - Shooting Stars\\
Band Maid - the non-fiction days\\
Beatsteaks - L auf der Stirn (feat. Deichkind)\\
Bestie - Excuse Me\\
Billie Eilish - bad guy\\
Billie Eilish - bury a friend\\
Billie Eilish - lovely\\
Billie Eilish - ocean eyes\\
Billie Eilish - when the party's over\\
Billy Joel - Uptown Girl\\
Birdy - Wings\\
Bliss n Eso - Tear The Roof Off (feat. Watsky)\\
Bloodhound Gang - Along Comes Mary\\
Bob Dylan - Blowin' in the Wind\\
Bob Marley - No Woman, No Cry\\
Bonnie Tyler - Total Eclipse of the Heart\\
Boy - Little Numbers\\
Camila Cabello - Havana\\
Camilla Cabello - Something's Gotta Give\\
Cantina Band Ringtone\\
Caravan Palace - Lone Digger\\
Casper - Im Ascheregen\\
Cat Stevens - Father And Son\\
Chiddy Bang - Opposite Of Adults\\
Childish Gambino - This Is America\\
Chris Rea - Driving Home For Christmas\\
Cocktail Shakers - Girl from Petaluma\\
Coolio - Gangsta's Paradise\\
Corey Hart - Sunglasses At Night\\
Crashing Atlas - Ascend\\
Cuelebre - Fodder for the Raven\\
Daft Punk - Within\\
David Bowie - Life On Mars\\
David Guetta - Memories (Feat. Kid Cudi)\\
Dead Sara - Weatherman\\
Deichkind - Bück dich hoch\\
Deichkind - Denken Sie groß\\
Deichkind - Der Flohmarkt ruft [feat. Herr Spiegelei]\\
Deichkind - Die Welt ist fertig\\
Deichkind - Hauptsache nichts mit Menschen\\
Deichkind - Leider geil\\
Deichkind - Mehr als lebensgefährlich\\
Deichkind - Porzellan und Elefanten\\
Deichkind - Richtig Gutes Zeug\\
Deichkind - So'ne Musik\\
Deichkind - Wer Sagt Denn Das\\
Dendemann - Stumpf Ist Trumpf 3.0\\
Dexter - Dies das (feat. Audio88 \& Yassin)\\
Diamante - Had Enough\\
Die Antwoord - Ugly Boy\\
Dire Straits - Sultans Of Swing\\
Dorothy - Down To The Bottom\\
Dr. Dre - Still D.R.E. [feat. Snoop Dogg]\\
Dynatron - Pulse Power\\
Dzivia - Uźniasieńnie\\
Dzivia - Voryva\\
Earth Wind And Fire - September\\
EAV - Fata Morgana\\
Eddie Van Der Meer - Unravel - Tokyo Ghoul OP 1\\
Electric Light Orchestra - Mr. Blue Sky\\
Elise Trouw - Burn\\
Elvis - Can't Help Falling in Love\\
Eminem - Lose Yourself\\
Eminem - Not Afraid\\
Ennio Morricone - The good, the bad and the ugly\\
Era - Ameno (Album Version)\\
Eric Clapton - Layla\\
Eric Clapton - Layla (Live in San Diego)\\
E.S. Posthumus - Unstoppable\\
Estas Tonne - The Song of the Butterfly\\
Estas Tonne - The Song of the Golden Dragon\\
Euriell - City of the Dead\\
Faber - Wem du's heute kannst besorgen\\
First To Eleven - New Rules\\
Fleurie - Hurricane\\
Florence + The Machine - Jenny of Oldstones (Game of Thrones)\\
Fools Garden - Lemon Tree\\
Fort Minor - Remember the Name\\
Fort Minor - Where'd You Go\\
Fytch - In These Shadows (feat. Carmen Forbes)\\
Fytch - Winter Wind (feat. Carmen Forbes)\\
Game Of Thrones - The Night King\\
Game Of Thrones - The Rains Of Castomere\\
Garmarna - Herr Mannelig\\
Gary Jules - Mad World (feat. Michael Andrews)\\
George Michael - Careless Whisper\\
Globus - Diem Ex Dei\\
Globus - Preliator\\
Gotye - Somebody That I Used To Know\\
Grissini Project - No Time for Caution\\
Grissini Project - Lilium\\
Grits - Here We Go\\
Grits - Ooh Ahh (My Life Be Like)\\
Grossstadtgeflüster - Fickt-Euch-Allee [Explicit]\\
Grossstadtgeflüster - Weil das morgen noch so ist\\
Guns N' Roses - November Rain (Album Version)\\
Gunship - Dark All Day (feat. Tim Cappello \& Indiana)\\
Gute Arbeit Originals - CtrlShift Sommer\\
GZUZ - Warum\\
Haiyti - Sunny Driveby\\
Halflives - Burn\\
Halocene - Good for You\\
Halsey - Heaven In Hiding\\
Hans Zimmer - Davy Jones\\
Holly Mae Henry - More Than Nothing\\
Hunger Games - Everybody Wants To Rule The World (feat Lorde)\\
Hunger Games - Safe \& Sound [feat. The Civil Wars \& Taylor Swift]\\
Hunger Games - The Hanging Tree [feat. Jennifer Lawrence]\\
Inglebirds - Wadadadang\\
Ice Cube - It Was A Good Day\\
Imagine Dragons - Warriors\\
Jim Pandzko feat. Jan Böhmermann - Menschen Leben Tanzen Welt\\
Joe Cocker - You Can Leave Your Hat On\\
John Butler - Ocean\\
John Denver - Take Me Home, Country Roads\\
John Lennon - Imagine\\
Jon Lajoie - Everyday Normal Guy\\
José Gonzales - Crosses\\
Juju \& Said - Berliner Schnauze\\
Kilez More - Alles Bleibt Gleich (feat. Die Bandbreite \& Morgaine)\\
K.I.Z - Abteilungsleiter Der Liebe\\
K.I.Z - Fremdgehen (Album Version)\\
K.I.Z - Glück gehabt\\
Kadebostany - Early Morning Dreams (Kled Mone Remix)\\
Kid Cudi - Mojo So Dope\\
Kid Cudi - Mr. Rager\\
Kid Cudi - Pursuit Of Happiness\\
Kid Cudi - Soundtrack 2 My Life\\
Kid Cudi - Up Up \& Away\\
Kitty Pryde - Okay Cupid\\
KIZ - Glück gehabt\\
Kollegah - Einer von Millionen (feat. Motrip)\\
Kontra K - Erfolg ist kein Glück\\
Kontrust - The Butterfly Defect\\
Kraftklub - Dein Lied\\
La Casa Del Papel - Bella Ciao\\
Laboratorium Pieśni - Sztoj pa moru\\
Lana Del Rey - Born To Die (Album Version)\\
Larkin Poe - Sea of Faces\\
Leo - Africa (feat. Hannah Boulton \& Rabea Massaad)\\
Logic - 1-800-273-8255\\
Logic - Ballin\\
Lord Of The Rings - The Battle Of The Pelennor Fields\\
Luca Stricagnoli - Now We Are Free\\
Maitre Gims - J'me tire\\
Malukah - I Follow the Moon\\
Marcin Przybyłowicz - Lullaby Of Woe\\
Marcin Przybyłowicz - Wolven Storm (English)\\
Marina - To Be Human\\
Markus Junnikkala - Even Death May Die\\
Marshmello \& Anne-Marie - Friends\\
Marteria - Kids (2 Finger an den Kopf)\\
Martin Garrix - Animals (Original Mix)\\
Marvin Gaye - Ain't No Mountain High Enough\\
MediMeister - Prince of Obermehl-Air\\
Medimeisterschaften Bonn - BonnAmour\\
Medimeisterschaften Bonn - Napoleon Bonnerparty\\
Medimeisterschaften Freiburg - \#Nurkittel\\
Medimeisterschaften Göttingen - Swinging Heart\\
Medimeisterschaften Jena - Jenandertaler\\
Medimeisterschaften Rostock - Woodstock Peace \& Love\\
Men at Work - Down Under\\
Merrigan - The Golden Hill\\
Metro Last Light - Behind the Red Curtain\\
Metro Last Light - Echoes of the Past\\
Metro Last Light - Reminiscence\\
Metro Last Light - The Farewell\\
Metro Last Light - Vessel of Sin\\
MGMT - Kids\\
Michael Jackson - They Don't Care About Us\\
Miike Snow - Genghis Khan\\
Möchtegang - Gf\_\_\_t letscht Nacht\\
Möchtegang - So andersch\\
Money Boy - Swaghetti Yolonese\\
Monty Python - Always Look On The Bright Side Of Life\\
Morgaine - Für Eine Bessere Welt\\
Morlockk Dilemma - Der Elfenbeinturm (feat. Audio88)\\
Morlockk Dilemma \& Hiob - Bastard Homosapiens\\
Morlockk Dilemma \& Hiob - Kapitalismus Jetzt\\
Mr. Big - Wild World\\
NF - Let You Down\\
NF - The Search\\
Nirvana - Smells Like Teen Spirit\\
Ö La Palöma Boys - Ö La Palöma\\
Of Monsters And Men - Little Talks\\
OK KID - Gute Menschen\\
Omnia - Fee Ra Huri\\
Omnimar - Boom Boom\\
Omnimar - Reason\\
Orgonite - Habibi Yaeni\\
Orgonite - Hamsa Xamca\\
Orgonite - Xamca\\
Otava Yo - Cossacks Lezginka\\
Owen Dennis - Gary vs. David\\
Owl City - Fireflies\\
Pale Waves - Television Romance\\
Patty Gurdy - Gurdy's Green\\
Patty Gurdy - Gurdy's Green\\
Patty Gurdy - The Longing (Storm Seeker Cover)\\
Peter Gundry - Don't Wake Me Just Yet\\
Pink Floyd - Another Brick In The Wall, Pt. 2\\
Polizistensohn - Blasserdünnerjunge macht sein Job\\
Post Malone - Congratulations\\
Princess Chelsea - Cigarette Duet\\
Prinz Pi - Kompass ohne Norden\\
Puddles Pity Party - Where Is My Mind\\
Pvris - Chandelier\\
Queen - Bohemian Rhapsody\\
Radiohead - Creep\\
Redbone - Come and Get Your Love (Single Edit)\\
Red Hot Chili Peppers - Dark Necessities\\
Red Hot Chili Peppers - Snow (Hey Oh)\\
Regular Show - Garys Synthesizer\\
Romano - Immun\\
Romano - Köpenick\\
Rummelsnuff - Bratwurstzange (Remix von Lord Of The Lost)\\
Scandal - Departure\\
Schandmaul - Dudelzack\\
Scott McKenzie - San Francisco\\
SDP - Merkste selber, wa!\\
Selena Gomez - Hands To Myself\\
Shireen - Umai\\
Sia - Chandelier\\
Sido - Der Tanz [feat. K.I.Z]\\
Sido - Spring rauf\\
Sigrid - Strangers\\
Silver - Wham Bam Shang-A-Lang\\
Silver Convention - Fly Robin Fly\\
Simon And Garfunkel - Scarborough Fair\\
Sina - Twenty-One Eleven (Feat. Mark Moody)\\
Skillet - Awake And Alive (Album Version)\\
SSIO - Schon wieder Sonntag\\
Stefan Raab - Wir Kiffen!\\
Stromae - Papaoutai\\
Sunflower Bean - I Was A Fool\\
Syd Matters - Obstacles\\
Taylor Davis - Fake Love\\
Taylor Davis - Game Of Thrones Theme\\
TeraBrite - Confident\\
Terabrite- The Greatest\\
The Animals - House of the Rising Sun\\
The Beach Boys - Wouldn't It Be Nice (Stereo Mix)\\
The Beatles - Hey Jude\\
The Beatles - While My Guitar Gently Weeps (Remastered)\\
The Buggles - The Plastic Age\\
The Clash - Should I Stay or Should I Go\\
The HU - Yuve Yuve Yu\\
The HU - Wolf Totem\\
The Jackson 5 - I Want You Back (SPK Mix)\\
The Jackson 5 - I Want You Back\\
The Oath - Silk Road\\
The Pineapple Thief - The Final Thing on My Mind\\
The Proclaimers - I'm Gonna Be (500 Miles)\\
The Red Army Choir - National Anthem of the Ussr\\
The Regrettes - Seashore [Explicit]\\
The Rolling Stones - Paint It Black\\
The Walking Dead - The Parting Glass\\
The XX - Intro\\
Tina Guo - Wonder Woman Main Theme\\
Tobias Rauscher - Still Awake\\
Tonight Alive - Breakdown\\
Trailerpark - Sterben kannst du überall [Explicit]\\
Trove Lo - Habits\\
twenty one pilots - Car Radio\\
twenty one pilots - Friend, Please\\
twenty one pilots - Heathens\\
twenty one pilots - Ride\\
Vikings - Floki Appears to Kill Athelstan\\
Ville Valo \& Natalia Avelon - Summer Wine\\
Wagakki Band - Strong Fate\\
Wallows - Scrawny\\
Watsky - Whoa Whoa Whoa\\
Waxx \& Pomme - Hotline Bling\\
Wild Arms - Into The Wilderness\\
Yes - Roundabout (Remastered Version)\\
Yung Larry - Lauch\\
Zugezogen Maskulin - Alle gegen Alle\\
Zugezogen Maskulin - Plattenbau O.S.T\\
Zugezogen Maskulin - Was für eine Zeit\\

\section{Klassik}

Albert Schönberger - O du fröhliche (Freie Orgelimprovisation)\\
Alice Sara Ott - Prélude in D Flat Major (Raindrop), Op.28, No.15\\
Ernst-Erich Stender - Improvisation über O du fröhliche\\
John Keys - O Come O Come Emmanuel (Veni Emmanuel) (Organ)\\
Liszt - Totentanz\\
Orchestre Montreal - Saint-Saens Danse Macabre, Op.40, R.171\\
Parley Belnap - Festive Trumpet Tune (David German)\\
Prélude in G minor, Op. 23 5 Alla marcia\\
Rachmaninoff - Idil Biret - Op. 3 No. 2. Prelude in C-Sharp Minor\\
Rachmaninoff - Morceaux de fantaisie, Op. 3 No. 2 in C-Sharp Minor, Prelude\\
Rachmaninoff - The Isle of the Dead, Op. 29\\
Rachmaninov - Prélude in C sharp minor Op.3 No.2\\
Tatyana Ryzhkova - Dreams of a Russian Summer (Dedicated to Tatyana Ryzhkova)\\
Valentina Lisitsa - Beethoven Piano Sonata No.14 In C Sharp Minor, Op.27 No.2 - Moonlight - 3. Pr\\
Yuja Wang - Saint-Saens Danse macabre, Op.40\\

\section{DEMO Google Drive/ Facebook}

Kardashev - Neverbreath (DEMO Album)

\section{Kostenlose Songs von Youtube etc.}

Pieter Daarth Project - Touching The Void\\
Regular Show - Garys Synthesizer\\
All That Remains - The Thunder Rolls\\
Wild Arms - Into The Wilderness (Original)\\
Matthias Rascher - Dance With Me (Michael Schütz)\\
Achozen - Deuces\\
Achozen ft. Killah Priest and Shukura Holliday - Immaculate\\

\section{Kostenlose Songs von Bandcamp etc.}

son kas - Wasserleichentreiben\\
MediMeister - Prince of Obermehl-Air\\
Witt Lowry - Kindest Regards\\
Witt Lowry - Go Big or Go Home ft. Trippz Michaud\\
Witt Lowry - Rescue\\
Witt Lowry - Wake Up\\
Witt Lowry - Youth\\
Witt Lowry - Witty's Acapella\\
Witt Lowry - Move On\\
Witt Lowry - Used To You\\
Witt Lowry - I Could Be\\
Witt Lowry - Dinner For Two\\
Witt Lowry - Higher Ground\\
Witt Lowry - Lay Here\\
Witt Lowry - Leave ft. Trippz Michaud\\
The Doo - Ascend\\
The Doo - Eclipse\\
The Doo - Horizons\\

\chapter{Backing Tracks}\label{btracks}

\subsection{Death Culture Studio}

\url*{https://www.youtube.com/channel/UC5i2WkUKnkAp5\_fHTuNyrrQ}\\
\ \\
DEATH METAL DRUM TRACK \#5\\
DEATH METAL DRUM TRACK \#4\\
DEATH METAL DRUM TRACK \#3\\
DEATH METAL DRUM TRACK \#2\\
DEATH METAL DRUM TRACK \#1\\
HEAVY METAL DRUM TRACK \#1\\

\subsection{GuitarHero0650}

\url*{https://www.youtube.com/channel/UCZiOxGzuoDlyWVdpMoP4O6g}\\
\ \\
Melodic Metal Backing Track - E Minor (Extended Version)\\
Metalcore Backing Track - D Minor\\
Metalcore Backing Track \#2 - D Standard\\
Rock Metal \#1 - C\\

\subsection{Arthur Sowinski}

\url*{https://sowinskibackingtracks.bandcamp.com}\\
\ \\
Arthur Sowinski - Sad Backing Track in D Minor\\
Arthur Sowinski - Sad Backing Track in E Minor\\

\subsection{Tore Fagerheim}

\url*{https://metalguitarstuff.bandcamp.com}\\
\ \\
Metal Guitar Stuff - Backing Tracks - How To Impress Girls With The Guitar - TABS\\
Metal Guitar Stuff - Backing Tracks - How To Impress Girls With The Guitar 2 - TABS\\
Metal Guitar Stuff - Backing Tracks - Into The Void - TABS\\
Metal Guitar Stuff - Backing Tracks - Last Chance\\
Metal Guitar Stuff - Backing Tracks - Original Song - Desecreation - TABS\\
Metal Guitar Stuff - Backing Tracks - Original Song - Ruins\\
Metal Guitar Stuff - Backing Tracks - The Fallen EP\\
Metal Guitar Stuff - Backing Tracks - A Minor 170 BPM Metal - Rock - Guitar Backing Track\\
Metal Guitar Stuff - Backing Tracks - Challenges\\
Metal Guitar Stuff - Backing Tracks - Challenges (Extended Edition)\\
Metal Guitar Stuff - Backing Tracks - Departure\\
Metal Guitar Stuff - Backing Tracks - E minor - B phrygian - Intermediate Level Backing Track\\
Metal Guitar Stuff - Backing Tracks - Mechanical Malfunction\\
Metal Guitar Stuff - Backing Tracks - Stading Ground\\
Metal Guitar Stuff - Backing Tracks - The Uprising\\
Tore Fagerheim - A Minor - Metal Guitar Backing Track\\
Tore Fagerheim - A Minor Power Ballad Clean Version\\
Tore Fagerheim - A Minor Power Ballad Guitar Backing Track\\
Tore Fagerheim - B Minor - Metal Guitar Backing Track\\
Tore Fagerheim - B Minor Melodic Death Metal - Melodeath Guitar Backing Track\\
Tore Fagerheim - B Minor Modern Metal Guitar Backing Track -7 String-\\
Tore Fagerheim - D Minor Metalcore Killswitch Engage Style Backing Track\\
Tore Fagerheim - E Minor - Heavy Rock - 80s Metal Guitar Backing Track\\
Tore Fagerheim - E Minor - Heavy Rock - Metal Guitar Backing Track\\
Tore Fagerheim - E Minor 80's Power Ballad Guitar Backing Track - Hard Rock Metal -\\
Tore Fagerheim - E Minor Acoustic Power Ballad Guitar Backing Track\\
Tore Fagerheim - E Minor Epic Power Ballad Guitar Backing Track\\
Tore Fagerheim - E Minor Guns N' Roses Style Guitar Backing Track\\
Tore Fagerheim - E Minor Modern - Classic Metal Guitar Backing Track\\
Tore Fagerheim - E Minor Modern Metal Sad Guitar Backing Track\\
Tore Fagerheim - E Minor Sad Guitar Backing Track Acoustic Ballad\\


%
\chapter{Personal Favorites}\label{fav}

\section{Albums}

\begin{enumerate}
	\item Orbit Culture - Rasen
	\item In Flames - Whoracle
	\item Heaven Shall Burn - Wanderer
	\item Raunchy - Wasteland Discotheque
	\item Gojira - From Mars To Sirius
	\item Dark Tranquillity - The Gallery
	\item Uada - Devoid Of Light
	\item Eluveitie - Origins
	\item Raubtier - Skriet Vran Vildmarken
	\item Amon Amarth - Surtur Rising
	\item Kvelertak - Meir
	\item Vindland - Hanter Savet
	\item Finntroll - Nifelvind
	\item Equilibrium - Turis Fratyr
	\item Gormathon - Following The Beast
	\item Blind Guardian - Imaginations Through The Looking Glass (DVD)
	\item Linkin Park - Minutes To Midnight
	\item Nightwish - End Of An Era (DVD)
	\item Epica - Retrospect (DVD)
\end{enumerate}

\newpage
\section{Songs}

\begin{enumerate}
	\item Orbit Culture - Sun Of All
	\item In Flames - The New World
	\item Heaven Shall Burn - Beyond Redemption
	\item Gojira - The Art Of Dying
	\item Raunchy - Somewhere Along The Road
	\item Dark Tranquillity - Punish My Heaven
	\item Amon Amarth - The Last Stand Of Frey
	\item Kvelertak - Apenbaring
	\item Gormathon - Land Of The Lost
	\item Fit For An Autopsy - Flatlining
	\item Architects - Holy Hell
	\item Cattle Decapitation - Manufactured Extinction
	\item At The Gates - Slaughter Of The Soul
	\item Paradise Lost - Beneath Broken Earth
	\item Moonspell - Extinct
	\item Mol - Bruma
	\item Vindland - Morlusenn
	\item Amorphis - House Of Sleep
	\item Numenorean - Regret
	\item Year Of The Goat - Avaritia
	\item Satyricon - Phoenix
	\item Solstafir - Fjara
	\item Uada - Devoid Of Light
	\item Shylmagoghnar - I Am The Abyss
	\item Fjoergyn - What A Wonderful World
	\item Be'lakor - The Smoke Of Many Fires
	\item Raubtier - En Hjältes Väg
	\item Omnium Gatherum - Ophidian Sunrise
	\item Insomnium - One For Sorrow
	\item Lindemann - Steh Auf
	\item Soilwork - The Ride Majestic (Aspire Angelic)
	\item Sabaton - Midway
	\item Kardashev - Beside Cliffs and Chasms
	\item Bloodred Hourglass - Times We Had
	\item While She Sleeps - Revolt
	\item Cytotoxin - Chaos Chascade
	\item Rings Of Saturn - Macrocosmos 
	\item Parkway Drive - Idols and Anchors
	\item Tempel - Afterlife
	\item Diablo Blvd. - Sing From The Gallows 
	\item Pieter Daarth Project - Touching The Void
	\item Rivers Of Nihil - The Silent Life
	\item Hypocrisy - Eraser
	\item Pain - Shut Your Mouth
	\item Mors Principium Est - Masquerade
	\item Parasite Inc. - The Pulse Of The Dead
	\item Tribulation - The Lament
	\item Trivium - The Crusade
	\item Blackbriar - I'd Rather Burn	
	\item Eluveitie - Neverland
	\item Skalmold - Vanaheimur
	\item Korpiklaani - Ämmänhauta
	\item Mono Inc. - The Banks Of Eden
	\item Carach Angren - When Crows Tick On Windows 
	\item Behemoth - Chant for Ezkaton 2000 e.v.
	\item Rotting Christ - The Raven
	\item Avenged Sevenfold - This Means War
	\item Callejon - Snake Mountain
	\item Ghost - From The Pinnacle To The Pit
	\item Finntroll - Under Bergets Rot
	\item Cradle Of Filth - Blackest Magick In Pracktice
	\item If These Trees Could Talk - Berlin
	\item Billy Talent - Cure For The Enemy
	\item Green day - Boulevard Of Broken Dreams
	\item Sum41 - We're All To Blame
	\item Blink182 - I Miss You
	\item Five Finger Death Punch - Wrong Side Of Heaven
	\item The Offspring - You're Gonna Go Far Kid
	\item Linkin Park - What I've Done
	\item System Of A Down - Soldier Side
	\item Blind Guardian - The Bards Song
	\item Evanescence - My Immortal
	\item Nightwish - Over The Hills And Far Away
	\item Within Temptation - Candles
	\item Epica - The Essence Of Silence
	\item Rise Against - Paper Wings
	\item Equilibrium - Blut Im Auge
	\item Fejd - Bed För Din Själ 		
	\item In Extremo - Liam
	\item Rammstein - Seemann
	\item Lord Of The Lost - Morgana
	\item Mechina - Progenitor
	\item Devin Townsend Project - Deadhead 
	\item Death - Without Judgement
	\item Deep Purple - Perfect Strangers
	\item Lacrimas Profundere - Antiadore
	\item Dawn Of Disease - Ascension Gate
	\item Belzebubs - Blackened Call
	\item Alter Bridge - Cry Of Achilles
	\item Volbeat - Always, wu
\end{enumerate}


\end{document}