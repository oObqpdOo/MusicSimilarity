%==============================================================

\section{Big Data processing}\label{bdf}

After accumulating the data and presenting various methods to extract and process different audio features, the following section describes the data analysis and computation of the similarities with the Big Data processing framework Apache Spark \cite{spark}. Later on chapter \ref{bds1} deals with the implementation of the various similarity measurements while chapter \ref{bds2} deals with the handling of larger amounts of data, runtime analysis and the combination of multiple similarity measurements.

\subsection{Hadoop}

With the ever growing availability of huge amounts of high dimensional data the need for toolkits and efficient algorithms to handle these grew as well over the past years. Search engine providers like Google and Yahoo firstly ran into the problem in the early 2000s \cite[p. 6]{sparkbook1}.

\subsubsection{MapReduce}

MapReduce as a programming paradigm was introduced by google as an answer to the problem of internet scale data and dates back to 2004 \cite{mapreduce1}. Figure \ref{mapred} shows the basic scheme of a MapReduce program. The core idea is to split a problem into many independent tasks.

\FloatBarrier
\begin{figure}[htbp]
	\centering
	%Image based on: https://commons.wikimedia.org/wiki/File:Mapreduce.png
	\begin{tikzpicture}[node distance = 4cm][every node/.style={thick}]
	  \colorlet{coul0}{orange!20} \colorlet{coul1}{blue!20} \colorlet{coul2}{red!20} \colorlet{coul3}{green!20}
	  \tikzstyle{edge}=[->, very thick]
	  \draw[thick, fill=violet!30] (-1, -2) rectangle node[rotate=90] {\textbf{Input data}} (0,2);
	  \foreach \i in {0,1,2,3} {
	    \node[draw, fill=coul\i, xshift=2em] (data\i) at (1.5, 1.5 - \i) {Input};
	    \node[ellipse, draw, fill=cyan!20, xshift=2em] (map\i) at (3.5, 1.5 - \i) {\textsf{Map}};
	    \draw[edge] (0,0) -- (data\i.west);
	    \draw[edge] (data\i) -- (map\i);
	  }
	  \node[draw, minimum height=2cm, fill=purple!30, xshift=7em] (resultat) at (10, 0) {\textbf{Results}};
	  \foreach \i in {0,1,2} {
	    \node[draw, fill=yellow!20, minimum width=2cm, xshift=4em] (paire\i) at (5.5, 1.5 - \i*1.5) {\begin{minipage}{1cm}Tuples \centering $\langle k,v \rangle$\end{minipage}};
	    \node[ellipse, draw, fill=cyan!20, xshift=6em] (reduce\i) at (7.5, 1.5 - \i*1.5) {\textsf{Reduce}};
	    \draw[edge] (paire\i) -- (reduce\i);
	    \draw[edge] (reduce\i.east) -- (resultat);
	  }
	  %paire
	  \draw[edge] (map0.east) -- (paire0.west); \draw[edge] (map0.east) -- (paire1.west);
	  \draw[edge] (map1.east) -- (paire0.west); \draw[edge] (map1.east) -- (paire2.west);
	  \draw[edge] (map2.east) -- (paire1.west); \draw[edge] (map2.east) -- (paire0.west);
	  \draw[edge] (map3.east) -- (paire1.west); \draw[edge] (map3.east) -- (paire2.west);
	\end{tikzpicture}
	\caption{MapReduce \cite{mapred1im}}
	\label{mapred}
\end{figure}
\FloatBarrier

replication rate\\

\subsubsection{Hadoop}

General Purpose Hardware/ commodity hardware\\
\ \\
Error Management\\
\ \\
Data locality\\
\ \\
Other Big Data Toolkits: SMACK; lambda; spark\\

\subsection{Spark}

SQL vs NoSQL\\
\ \\
user defined function\\
\ \\
cluster configuration file\\
\ \\
streaming\\
\ \\
caching\\

\subsubsection{Spark and RDDs}

\subsubsection{Spark DataFrame}

\subsubsection{lazy evaluation}

\subsubsection{min and max value aggregation}

\subsubsection{Hadoop vs Spark Memory Management}

\FloatBarrier
\begin{figure}[htbp]
	\centering
	\begin{tikzpicture}
	\node[server](server 1){};
	\node[server, right of= server 1](server 2){};
	\node[server, right of= server 2](server 3){};
	\node[rack switch, above of=server 2,xshift=0.1cm,yshift=0.3cm]
	  (rack switch 1){};
	\draw[thick,darkgray!10!gray] (server 1.north)--(rack switch 1);
	\draw[thick,darkgray!10!gray] (server 2.north)--(rack switch 1);
	\draw[thick,darkgray!10!gray] (server 3.north)--(rack switch 1);
	% = = = = = = = = = = = = = = = =
	% Labels
	% = = = = = = = = = = = = = = = =
	\node[xshift=-1.55cm,yshift=0.2cm,left of = server 3,align=left](lev1)
	  {Computing Servers};
	\node[xshift=-0.55cm,yshift=0.3cm,above of = lev1,align=left](lev2)
	  {Access Layer};
	\end{tikzpicture}
	\caption{Data locality}
	\label{dataloc}
\end{figure}
\FloatBarrier

\subsubsection{Spark application UI}

\subsection{Data locality and parallelization}

\noindent\textit{\textbf{high data locality, full parallelizable, very low replication rate\\}}
\noindent\textit{\textbf{insert data locality scheme here\\}}

\subsection{Combining different measurements}

\noindent\textit{\textbf{weighted arithmetic mean of different distance measurements\\}}

\begin{equation} \label{eq:distance}
dist = \frac{\sum_{m = 0}^{M - 1}{w_m \cdot d_m}}{\sum_{m = 0}^{M - 1}{w_m}}
\end{equation}

\noindent\textit{\textbf{statistic prescaling to have mean = 0.5 and variance 0.5?\\}}


