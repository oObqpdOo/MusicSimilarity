%==============================================================

\chapter{Big Data Framework Spark}\label{bds1}

After accumulating the Data and presenting various methods to extract different audio features, the following chapters describe the data analysis with Apache Spark \cite{spark}. Chapter \ref{bds1} deals with the implementation of the various similarity measurements while chapter \ref{bds2} deals with the handling of larger amounts of data, runtime analysis and the combination of multiple similarity measurements. 

\section{Apache Hadoop and Spark} 

First of all a small introduction to Big Data processing with Spark is given.
With the ever growing availability of huge amounts of high dimensional data the need for toolkits and efficient algorithms to handle these grew as well over the past years. Map-Reduce as a 

\subsection{Hadoop and Map Reduce}

\subsection{Spark and RDDs}

\subsection{Spark DataFrame}

\subsection{Performance Comparison of RDD vs DataFrame}

\section{Data aggregation}

To work with the features a few transformations have to be done first. 

\begin{pythonCode}
song = "music/TURCA1.wav"
chroma = sc.textFile("features/out[0-9]*.notes")
chroma = chroma.map(lambda x: x.split(';'))
chroma = chroma.map(lambda x: (x[0], x[1], x[2], x[3].replace("10",'K').replace("11",'L').replace("0",'A').replace("1",'B').replace("2",'C').replace("3",'D').replace("4",'E').replace("5",'F').replace("6",'G').replace("7",'H').replace("8",'I').replace("9",'J')))
chroma = chroma.map(lambda x: (x[0], x[1], x[2], x[3].replace(',','').replace(' ','')))
df = spark.createDataFrame(chroma, ["id", "key", "scale", "notes"])
\end{pythonCode}

\section{Euclidean Distance}

\begin{pythonCode}
distance_udf = F.udf(lambda x: float(distance.euclidean(x, comparator_value)), FloatType())
result = df_vec.withColumn('distances', distance_udf(F.col('features')))
result = result.select("id", "distances").orderBy('distances', ascending=True)
result = result.rdd.flatMap(list).collect()
\end{pythonCode}

\section{Bucketed Random Projection}

\begin{pythonCode}
brp = BucketedRandomProjectionLSH(inputCol="features", outputCol="hashes", seed=12345, bucketLength=1.0)
model = brp.fit(df_vec)
comparator_value = Vectors.dense(comparator[0])
result = model.approxNearestNeighbors(df_vec, comparator_value, df_vec.count()).collect()
rf = spark.createDataFrame(result)
result = rf.select("id", "distCol").rdd.flatMap(list).collect()
\end{pythonCode}


\section{Cross-correlation}

\begin{pythonCode}
def chroma_cross_correlate(chroma1, chroma2):
    corr = sc.signal.correlate2d(chroma1, chroma2, mode='full') 
    transposed_chroma = np.transpose(corr)
    mean_line = transposed_chroma[12]
    return np.max(mean_line)
distance_udf = F.udf(lambda x: float(chroma_cross_correlate(x, comparator_value)), DoubleType())
result = df_vec.withColumn('distances', distance_udf(F.col('chroma')))
result = result.select("id", "distances").orderBy('distances', ascending=False)
result = result.rdd.flatMap(list).collect()
\end{pythonCode}


\section{Kullback-Leibler Divergence}

\begin{pythonCode}
def symmetric_kullback_leibler(vec1, vec2):
    mean1 = np.empty([13, 1])
    mean1 = vec1[0:13]
    cov1 = np.empty([13,13])
    cov1 = vec1[13:].reshape(13, 13)
    mean2 = np.empty([13, 1])
    mean2 = vec2[0:13]
    cov2 = np.empty([13,13])
    cov2 = vec2[13:].reshape(13, 13)
    d = 13
    div = 0.25 * (np.trace(cov1 * np.linalg.inv(cov2)) + np.trace(cov2 * np.linalg.inv(cov1)) + np.trace( (np.linalg.inv(cov1) + np.linalg.inv(cov2)) * (mean1 - mean2)**2) - 2*d)
    return div
distance_udf = F.udf(lambda x: float(symmetric_kullback_leibler(x, comparator_value)), DoubleType())
result = df_vec.withColumn('distances', distance_udf(F.col('features')))
result = result.select("id", "distances").orderBy('distances', ascending=True)
result = result.rdd.flatMap(list).collect()
\end{pythonCode}


\section{Jensen-Shannon Divergence}

\begin{pythonCode}
def jensen_shannon(vec1, vec2):
    mean1 = np.empty([13, 1])
    mean1 = vec1[0:13]
    cov1 = np.empty([13,13])
    cov1 = vec1[13:].reshape(13, 13)
    mean2 = np.empty([13, 1])
    mean2 = vec2[0:13]
    cov2 = np.empty([13,13])
    cov2 = vec2[13:].reshape(13, 13)
    mean_m = 0.5 * (mean1 + mean2)
    cov_m = 0.5 * (cov1 + mean1 * np.transpose(mean1)) + 0.5 * (cov2 + mean2 * np.transpose(mean2)) - (mean_m * np.transpose(mean_m))
    div = 0.5 * np.log(np.linalg.det(cov_m)) - 0.25 * np.log(np.linalg.det(cov1)) - 0.25 * np.log(np.linalg.det(cov2))  
    if np.isnan(div):
        div = np.inf
    return div
distance_udf = F.udf(lambda x: float(jensen_shannon(x, comparator_value)), DoubleType())
result = df_vec.withColumn('distances', distance_udf(F.col('features')))
result = result.select("id", "distances").orderBy('distances', ascending=True)
result = result.rdd.flatMap(list).collect()

\end{pythonCode}

\textit{Problem mit den "skyrocketing determinanten" -> Lösung: cholesky zerlegung, geht nicht - nicht positiv definit}\cite[p.45]{schnitzer1}

\section{Levenshtein distance}

\begin{pythonCode}
def get_neighbors_notes(song):
    filterDF = df.filter(df.id == song)
    filterDF.first()
    comparator_value = filterDF.collect()[0][3] 
    print comparator_value
    df_merged = df.withColumn("compare", lit(comparator_value))
    df_levenshtein = df_merged.withColumn("word1_word2_levenshtein", levenshtein(col("notes"), col("compare")))
    df_levenshtein.sort(col("word1_word2_levenshtein").asc()).show()
\end{pythonCode}

\section{Andere MapReduce tasks}

\subsection{Alternating Least Squares}

\subsection{TF-IDF weights}

\subsection{DIMSUM all-pairs similarity}

\section{Combining different measurements}

\textit{\textbf{Pre- merge feature sets, broadcast comparator, descending importance pre-filtering}}

\textit{\textbf{statistic normalization of similarities}}

\section{performance}

\textit{\textbf{high data locality, full parallelizable, very low replication rate}}

\textit{\textbf{insert data locality scheme here}}

Very small 100 Song Dataset plus 20 Cover Songs to evaluate and compare similarity metrics. Full similarity matrices. 
High performance/ throughput tests with full 12000 song testset to evaluate full load performance.\\


